\documentclass[a4paper]{scrreport}

\usepackage[T1]{fontenc}
\usepackage[utf8]{inputenc}
\usepackage[nswissgerman]{babel}
\usepackage[babel,german=swiss]{csquotes}
\usepackage[onehalfspacing]{setspace}
\usepackage[style=verbose-ibid]{biblatex} % -ibid


% Literaturverzeichnis hinzufügen
\addbibresource{bibliography.bib}


% Einstellungen für Darstellung der Fussnoten
\usepackage[bottom,norule,hang,perpage]{footmisc}
\setlength{\footnotemargin}{2mm} % Spacing in Fussnoten zwischen der Zahl und dem Text
% \renewcommand{\footnoterule}{\rule{45mm}{0.1mm}{\vspace*{5mm}}} % Trennlinie über Footer


% Einstellunge für Zitierart
\renewcommand*{\labelnamepunct}{\addcolon\addspace} % Interpunktion in Zitaten zwischen Autor und Titel
\renewcommand{\newunitpunct}{\addcomma\addspace} % Zitatelemente mit einem Komma trennen (nicht der standardmässig Punkt)
\renewcommand{\subtitlepunct}{\addperiod\addspace} % Interpunktion in Zitaten zwischen Titel und Untertitel


% Spacing des Papiers
\usepackage{geometry}
\geometry{a4paper,top=30mm,right=27mm,bottom=30mm,left=27mm}


% Überschriftentiefe die im Inhaltsverzeichnis nummeriert werden
\setcounter{secnumdepth}{3}


% Überschriftentiefe die im Inhaltsverzeichnis angezeigt werden
\setcounter{tocdepth}{3}


% Abbildungen (figure) innererhalb einer Section durchnummerieren
% https://tex.stackexchange.com/a/425603
% \let\counterwithout\relax
% \let\counterwithin\relax
% \usepackage{chngcntr}
% \counterwithin{figure}{section}
% \counterwithin{table}{section}


% Sections durchgehend nummerieren und nicht von Chapters unterbrechen
\usepackage{chngcntr} % chngcntr bereits eingebunden
\counterwithout{section}{chapter}


% Package um Links im PDF-Dokument zu verwenden
\usepackage{url}
\usepackage[hidelinks]{hyperref}


% Package um Bilder (PNG) und PDFs als figures einzubinden
\usepackage{graphicx}


% Placeholder for figures: \includegraphics[width=.5\textwidth]{example-image-a}
% https://tex.stackexchange.com/a/413186
\usepackage{mwe}


% Package um Symbole für Musik bereitzustellen: \flat \sharp
% \usepackage{textcomp}


% Package um Symbole Noten bereitzustellen: \quarternote \twonotes
% \usepackage{wasysym}


% Package um Generalbassbezifferungen im Fließtext besser zu verwenden
% https://tex.stackexchange.com/a/267644
% https://tex.stackexchange.com/a/267618
% \usepackage{amsmath} % \overset{above}{main}


% Command für inline Code im Fließtext
\newcommand{\code}{\texttt}


% Package und Command um Text einzukreisen für Bassstufen
% https://tex.stackexchange.com/a/199056
\usepackage{tikz}
\newcommand*\circled[1]{\tikz[baseline=(char.base)]{\node[anchor=text, shape=circle,draw, inner sep=0pt, minimum size=1.2em] (char) {#1\strut};}}


% Git Hash der Version erkennen und auf \gitAbbrevHash legen
\usepackage{xstring}
\usepackage{catchfile}
\CatchFileDef{\gitHash}{.git/refs/heads/master}{}
\newcommand{\gitAbbrevHash}{\StrLeft{\gitHash}{7}}


% Alle Quellen im Literaturverzeichnis anzeigen, auch nicht verwendete
\nocite{*}


% Einstellungen für das Stichwortverzeichnis
% \usepackage{makeidx}
% \makeindex
% \renewcommand{\indexname}{Stichwortverzeichnis}


% Nummerierung von Sections ohne die Zählung von Chapter
\renewcommand*\thesection{\arabic{section}}


% Spacings in Tabellen
\setlength{\tabcolsep}{1em} % horizontal padding
{\renewcommand{\arraystretch}{1.5}% vertical padding


% Fussnoten in Tabellen aktivieren mit dem Command \tablefootnote
\usepackage{tablefootnote}


% Zeilenumbrüche in Tabellen erlauben mit \thead{Mixo-\\lydisch} und |makecell{}
% https://tex.stackexchange.com/a/176780
\usepackage{makecell}


% Fügt Fraktur Schriftarten hinzu: \textfrak{Fraktur} \textswab{Schwabacher} \textgoth{Gothic}
% \usepackage{yfonts}


% "Abbildung ..." unter den figures umbenennen
\usepackage[figurename=Notenbeispiel]{caption}
\renewcaptionname{nswissgerman}{\listfigurename}{Verzeichnis der Notenbeispiele}


% Gestrichelte horizontale Linien in Tabellen mit \hdashline
\usepackage{arydshln}


% Abstände im Inhaltsverzeichnis zwischen Zahl und Titel anpassen
\usepackage{tocloft}
\setlength{\cftsecnumwidth}{1.5em}
\setlength{\cftsubsecnumwidth}{2.1em}
\setlength{\cftsubsubsecnumwidth}{2.75em}


% Sicherstellen, dass figures in der section ausgegeben werden zu der sie gehören
\usepackage[section]{placeins}




\title{Gemeinsam musizieren Lernen}
\subtitle{Master Musikpädagogik}
\subject{Masterthesis}
\titlehead{
	\hfill
	\includegraphics{img/Logo_Musikhochschule.pdf}
}
\author{
	Katharina Jäckle\\
	Hochschule für Musik Freiburg
}
\dedication{
	{
		\Huge
		Gemeinsam musizieren Lernen\\
	}
	\vspace{0.75cm}
	{
		\Large
		Das gemeinsame Musizieren Lernen\\
		oder auch gemeinsam das Musizieren Lernen\\
	}
	% \vspace{1.5cm}
	% {
	% 	\normalsize
	% 	Ausserdem noch eine kleine Danksagung \\
	% 	oder weitere kurze Unteruntertitel hier\\
	% }
	\vspace{12cm}
	{
		\footnotesize
		Text erstellt mit \LaTeX und git.\\
		\small{Version: \gitAbbrevHash}
	}
}
\publishers{
	Gutachter: Andreas Doerne\\
}
\date{\today}



\begin{document}

	\maketitle

	\tableofcontents

    \addchap{Vorwort}

Mein persönlicher Hintergrund aus dem ich heraus dieses Thema gewählt habe, war
zum einen die Situation meiner Privatschüler*innen, zum anderen die Situation
der Instrumentalschüler*innen am Birklehof, einem Privatinternat an dem ich
unterrichte und zu guter Letzt der Einfluss von Corona auf unsere soziale
Umwelt. Meine Privatschüler*innen erhalten einmal wöchentlich meinen Unterricht,
darüber hinaus sind sie aber nicht ansatzweise in eine musikalische Umgebung
eingebettet. Drei Ausnahmen sind eine Schülerin, deren Mutter selbst
Geigenlehrerin ist, wo zu Hause viel von der Familie aus musiziert wird und zwei
Medizin-Studenten, die mit großer Begeisterung im Uni-Orchester spielen. Aber
die Zeiten von Corona gehen nicht spurlos an dem Instrumentalunterricht vorüber.
Die Lockdowns, die beispielweise alle denkbaren Ensembleproben zu einem
schwierigen Unterfangen gemacht haben, die Abstandsregelen und auch das Verbot
von Veranstaltungen in größeren Gruppen, wie Konzerte, hat vor allem eines sehr
deutlich gezeigt: der Mensch braucht sein soziales Umfeld sowie die soziale
Interaktion um sich weiter zu entwickeln. Wie wichtig dieser Bestandteil auch
für unsere musikalischen Erfahrungen ist, werde ich in meiner Masterarbeit näher
beleuchten.

Ein herzliches Dankeschön geht an die Einflüsse, Anregungen und den Input meiner
Professoren Wolfgang Lessing und Andreas Doerne, sowie Michael Stecher die
vorallem im musikpädagogischen Bereich prägend für mich waren und meinen
Horizont immer weiter wachsen lassen haben. Für mein künstlerisches Dasein, war
die Studienzeit genauso anregend, da viele, um nicht zu sagen alle Fragen die
aufgeworfen wurden, auch mit meiner Person als ausübende Künstlerin eng
verbunden waren. Außerdem bedanke ich mich bei meinem Kommilitonen Timo Langpap,
der zurecht in diesem Zusammenhang sagen würde, dass auch wir
Instrumentallehrer*innen unsere eigene künstlerische Persönlichkeit pflegen und
in den Instrumentalunterricht einbringen müssen. Wir haben viel Co-Working für
Projekte inner- und außerhalb des Studiums durchgeführt, uns immer wieder
ausgetauscht und diskutiert. Dank Herr Prof. Lessing haben wir außerdem an dem
Hochschulwettbewerb der deutschen Musikhochschulen teilgenommen und mit unserem
Projekt \emph{Digitale Kettenkomposition} einen zweiten Preis erhalten. In dem
Projekt haben wir Schüler aus verschiedenen Orten, mit verschiedenen
Instrumenten und Genres musikalisch miteinander verknüpft und damit die ein oder
andere Lockdown-Welle für unsere Schüler erträglicher machen können. 

Ich bedanke mich bei den Korrekturlesern und für die Unterstützung in der
Nutzung des Programms LaTex bei Wolfgang Drescher.




\vspace{0.5cm}

\begin{flushright}
	{
		\small
		Freiburg im Breisgau, September 2021\\
		--- Katharina Jäckle }
\end{flushright}

	\addchap{Einleitung}

"Musik ist Teil der Symbolwelt des Menschen: Mitteilung, Kommunikation,
Interaktion." \autocite[91]{doerne:umfassend_musizieren}

Das gemeinsame Musizieren zieht sich bei meinem Lebenslauf wie ein roter Faden
durch die Laufbahn. Als Instrumentallehrerin habe ich allerdings auch
beobachtet, wie es für Schüler ist, die lediglich den
Einzel-Instruemntalunterricht kennen. Dabei ist mir aufgefallen, dass ein
wesentlicher Bestandteil fehlt: das
gemeinsame Musizieren. Wenn man alleine den Aspekt des
Differenziellen Lernens betrachtet wird schnell klar, dass Einzelunterricht
alleine nicht ausreicht um einen Schüler motiviert am Instrument zu halten und
ihn weiter zu bringen. Da sich auf der sozialen Ebene im Miteinander so viel
ereignet, kann eine einzelne Lehrkraft nicht alle Ebenen abdecken und bedienen.
Oft ist in unserem Musikschulsystem das gemeinsame Musizieren aber gar nicht
vorgesehen, sondern vielmehr ein optionales Zusatzangebot, in Form von einem
Mitwirken im Orchester, im Chor
oder in der Band. Dieses Zusatzangebot ist aber auch nicht immer gegeben;
besonders bei kleineren Institutionen oder auch bei Privatschülern, sind die
Instrumentallehrer mit ihren Schülern auf sich alleine gestellt. 
Das "gemeinsame Musizieren" will gelernt sein, genau so wie 
aber auch "gemeinsam Musizieren Lernen" möglich ist. Was ist die bessere
Variante und schließt das Eine das Andere aus? Zunächst werfe ich einen Blick in
die Theorie und werde die verschiedenen Aspekte die Gruppenprozessen mit sich
bringen, herauszuarbeiten.
Wie das gemeinsame Musizieren in anderen Ländern wie
Venezuela oder auch Japan funktioniert und was in welchem System der Fall ist,
werde ich in dieser Arbeit im darauffolgenden Schritt
genauer beleuchten. Anschließend wird betrachtet, wie der gängige Weg eines
Instrumentalschülers in Deutschland aussieht. Des Weiteren werde ich die verschiedenen
Systeme in Hinblick auf das sozialen Miteinander und ihren Zusammenhang mit
dem Instrumentalunterricht analyiseren. Dabei arbeite ich die Quintessenzen der
jeweiligen Systeme heraus, und warum das gemeinsame Musizieren so
wichtig ist. Ziel ist es, daraus ein mögliches Szenario für ein weiter
verbreitetes und eingebettetes 
"gemeinsam Musizieren Lernen" in Deutschland abzuleiten.


Da die Vielfalt an Terminologien bezüglich der verschiedenen
Geschlechter heutzutage sehr groß ist, werde ich fortfolgend in meiner Masterarbeit
absatzweise wechselnd männlich und weiblich gendern. Ich weise an dieser Stelle
daraufhin, dass sowohl alle männlichen, weiblichen und auch diversen
Geschlechter mit einbezogen werden und gemeint sind. 

	\addchap{Gemeinsames Musizieren}

Das gemeinsame musizieren lernen ist ein wichtiger Prozess, der in der
Instrumentalausbildung ein fester Bestandteil sein sollte. Schließlich müssen
die Instrumentalschüler ersteinmal an die Gruppenprozesse herangeführt werden um
zu lernen wie man (musikalisch) interagieren kann. Es schließt allerdings nicht
aus, dass man dabei gemeinsam das Musizieren lernt. Im besten Fall wird nämlich
genau das erreicht. Dabei sollte das Musizieren im Vordergrund stehen und nicht
die technischen Hürden, die ein Instrumentalschüler überwinden muss um dorthin
zu kommen. Insofern sollten am
besten beide Varianten in der Praxis ineinander greifen, was aber nicht
ausschließt, dass ein Schüler zusätzlich noch Einzelunterricht erhält.

Bei dem Gemeinsamen Musizieren spielen viele verschiedene Faktoren eine Rolle.
Im Folgenden werde ich deswegen verschiedene Aspekte beleuchten und erörtern
inwieweit sie die Schülerinnen weiterbringen und für sie gewinnbringend sein
können. Außerdem werfe ich aber auch einen Blick auf die Aufgabe der
Instrumentallehrer um darauf einzugehen, wie man verschiedene Gruppen und
Situationen wirkungsvoll leiten kann. Alle Aspekte spielen sich auf der sozialen
interaktiven Ebene ab, im Wechselspiel zwischen dem Individuum und der Gruppe.
Dafür ist ein erster Blick in die Gruppenprozesse sinnvoll, sowie mögliche
Aktionsformen für Gruppen, die Hierarchie innerhalb einer Gruppe, die
Kommunikation, Motivation und das psychosoziale Lernen, welches in der Gruppe
stattfindet wird betrachtet, sowie die Bedeutung von "Gruppenarbeit" für den
Schüler und die Bedeutung für den Lehrer.

















%"Musik als Kommunikationsmittel" S. 17 %doerne?

%Aristoteles: "der Schlüssel zu einem glücklichen Leben besteht im Miteinander,
%das Leben in der Gemeinschaft unter Freunden." (S. 9 "Die Kunst zu unterrichten)

%%"Sie muss gelernt und ein Leben lang geübt werden." (auch S. 9 %Doerne?)


%"Jedes Menschen Persönlichkeit, das bedeutet: seine Fähigkeiten, seine Art zu
%denken und zu fühlen, wird durch Umstände und Umgebung geschnitzt und
%gemeißelt." (S. 20 shinichi Suzuki)

%Andreas Doerne schreibt sehr zutreffend "(...) dass Musik in ihrem Wesen Mitteilung, Kommunikation, Gestaltung, das
%Vermitteln einer Botschaft ist, die in Sternstunden über das Erleben der
%Einzelnen hinausreicht und sie miteinander verbinden kann." S. 11 Doerne?
%(andreas doerne?) 



\section{Gruppenprozesse}


- Individualität in der Gruppe 
"(...)Individualität zeigt sich gerade auch im Umgang mit anderen,
Individualität entwickelt sich im Vergleich zu und in ABgrenzung von anderen
oder auch dadurch, dass Positionen in einer Gruppe besetzt werden."

- unterschiedliche Alters- und Lernniveaus (manchmal) evtl. S. 95

- Lernen / Differenzielles Lernen? 
"Ein Großteil musikalischen Lernens findet über Imitation statt (...) S. 98

- Hierarchie
fortgeschrittenere Schüler zu Mentoren erklären und sie damit mitverantwortlich
für deren musikalische Entwicklung und den Fortschritt des Ensembles zu machen.
S. 95

- Altersgruppen/Erwachsene
"Wiedererlangen sozialer Integration scheint ein wesentliches Motiv
musikalischer Betätigung bei Erwachsenen (...). " s. 118

- "Selbsttätigkeit" (die Kunst zu unterrichten)
"Erst wenn es eine Notwendigkeit für einen Schüler gebe, eine Arbeit wirklich
selbst zu vollbringen, würde er ganzheitlich mit seinen Gefühlen, Trieben und
Willenskraft in die Tätigkeit involviert. (S. 63 bzw Eigl von Kerschensteiner)

-Gruppe als Korrektiv

\subsection*{Bedeutung für den Schüler*in}

Welche Unterrichtsform für welchen Schüler besonders geeignet ist, ist ein sehr
individuelles Unterfangen. Während der eine Schüler die ganze Aufmerksamkeit
eines Lehrers genießt, kann sich der andere im Einzelunterricht unwohl fühlen.
Wer ein ser kommunikativer Mensch ist, ist hingegen vermutlich in den
Gruppenunterrichten besser aufgehoben. Oftmals ändern sich die Unterrichtsformen
für einen Schüler allerdings auch, da es gängig ist in Gruppenunterrichten zu
starten. Die Form eines Kleingruppenunterrichts bei Anfängern bietet sich auch
an, siehe Zitat \autocite[220]{Busch:grundwissen_Instrumentalpädagogik}
Im Endeffekt ist es aber im fortgeschrittenen Stadium am besten, wenn die
Instrumentalschüler sowohl Einzelunterricht haben, als auch in eine
Musiziergruppe eingebunden sind. 

"Man erfährt beim Musizieren soziale Resonanz(...)" S.28

"Eine Steigerung musikalischer Kommunikation erfährt der Schüler im
Zusammenspiel mit dem Lehrer, im Gruppen- oder Ensembleunterricht. (...) Die Individualität des eigenen Musizierens entwickelt sich in
Beziehung zu
anderen."  S.99 Punkt 5

"Das Gefühl von Zugehörigkeit und Zusammenklang beim Spielen im Ensemble."


\subsection*{Bedeutung für den Lehrer}

"Die Besonderheiten in der instrumentalpädagogischen Arbeit mit Gruppen
resultiert aus der Tatsache, dass mehrere Lernende zeitlgleich so unterrichtet
werden müssen, dass ein gemeinsamer Lernprozess stattfindet."
\autocite[221]{Busch:grundwissen_Instrumentalpädagogik} Die Herausforderung
besteht also darin, allen Lernenden weiterzuhelfen, auch wenn jeder Einzelne mit
individuellen Schwierigkeiten zu kämpfen hat. (Der Vorteil in der Gruppe ist aber
auch, dass sich die Lernenden untereinander Lösungsmöglichkeiten abschauen nud
auch gegenseitig helfen können.) 
Je nach Gruppengröße muss der Lehrer verschiedene Aufgaben übernehmen,
zu der
auch eine unterschiedliche Vorbereitungsnotwendigkeit hinzukommt. Während der
der Lehrende im Einzelunterricht unmittelbar auf die Impulse des Schülers
eingehen kann, so muss der Lehrende bei Kleingruppen neben klaren Vorgaben die
Schüler zu einem von- und miteinander Lernen anregen und sich selbst auch einmal
zurücknehmen. Im Großgruppenunterricht ist es essentiell, dass der Lehrer im
Vorfeld eine Struktur und Stücke festlegt um die Inhalte gut anleiten zu können.\autocite[220]{Busch:grundwissen_Instrumentalpädagogik}
Meistens hat man in dieser Konstellation die Situation eines Frontalunterrichts,
was beim Klassenmusizieren von den Instrumentalpädagogen verlangt wird, während die
Instrumentalpädagogen mit diesem Szenario oftmals nicht vertraut sind. Bei sehr
großen Gruppen, wie beispielweise im
Klassenmusizieren ist es allerdings auch keine Seltenheit, dass zwei oder
mehrere Lehrende in unterschiedlichen Konstellationen zusammenarbeiten. Dies
bringt auch wieder neue Herausforderungen mit sich, so müssen sich die Lehrenden
gut untereinander absprechen und auch gegenseitig aufeinander eingehen und
flexibel reagieren können. Dafür sind klare Absprachen im Voraus sehr dienlich,
so kann zum einen vereinbart werden, dass der eine Lehrer den Unterricht leitet,
während der andere beobachtet und assistiert. Es gibt aber auch die Möglichkeit,
dass die Lehrenden sich unterinander abwechseln und je nach inhaltlichem
Schwerpunkt ablösen, sodass jeder das Gebiet übernehmen kann, in dem er einen
besonderen Zugang hat. Es gibt eine dritte Möglichkeit die gesamte Gruppe
simultan zu unterrichten und die vierte Möglichkeit die Gruppe der Lernenden
aufzuteilen und im selben oder in getrennten Räumen parallel zu unterrichten.
\autocite{cook s. 461} 

"Für den Instrumentallehrer stellt Gruppen- oder Ensembleunterricht eine höhere
Herausforderung dar, da er seine Aufmerksamkeit (gerecht) und flexibel
verteilen, differenziertere Spielanweisungen geben und eine musikalisch und
menschlich komplexer Situtation steuern muss. Andereseits ergeben sich im Umgang
mit der Gruppendynamik auch andere Möglichkeiten der Motivation und Förderung
musikalischen Lernens." S. 99 Punkt 6 (doerne umfassend musizieren?)
Wie man mit dieser Herausforderung allerdings besser zurechtkommen kann,
beschreiben Barbaraba Busch und Barbara Metzger sehr zutreffend und auch
umfassend in dem Kapitel "Leitgedanken für die Arbeit mit Gruppen" in dem Buch
Grundwissen Instrumentalpädagogik. 


- (differenzielles Lernen)
"Von einem bestimmen Alter an ist den Heranwachsenden die Rückmeldung der
Gleichaltrigen sowieso wichtiger als die des Lehrers, dann bewirtk die Kritik
eines Pultnachbarn manchmal eher eine Veränderung oder eine intensivere
Auseinandersetzung mit einem Stück als die des Dirigenten "Da vorn." S. 95


Evtl. ins Fazit? ((Die Musikhochschule müssten in diesem Punkt noch etwas besser ausbilden, da
jeder Instrumentalpädagoge darauf vorbereitet sein sollte und es so wichtig ist.
S. 94 oberer Abschnitt ))

"(...) oft ziemlich eng ausgebildeten Instrumentalpädagogen(...) S. 119






	\addchap{Der gängige Weg}

Wie in Deutschland der gängige Werdegang eines Instrumentalschülers aussieht,
soll im Folgenden beschrieben werden. Viele Schüler*innen kommen in ihrem Leben
ausschließlich in den Genuss des Einzel-Instrumentalunterrichts, aber es gibt in
der musikalischen Früherziehung, sowie auch ergänzend zum Einzelunterricht auch
hierzulande einige Angebote, die Schüler*innen über den Einzelunterricht hinaus
wahrnehmen können und auch teilweise nutzen.


\section{Musikalische Früherziehung}

Das Arbeiten mit Gruppen ist ein grundlegendes Merkmal der Elementaren
Musikpädagogik (EMP). Wir können aus der EMP auch für andere Musiziergruppen
einige wichtige Bausteine ableiten, zumal sich das Elementare Musizieren an
Menschen jeden Alters wendet.
\autocite[226]{busch:grundwissen_instrumentalpaedagogik} Oft wird sie jedoch
besonders für die musikalische Früherziehung genutzt und richtet sich bei vielen
Angeboten an junge Familien oder Kindergartenkinder, ähnlich wie bei Suzukis
Philosophie. Es ist  für die musikalische, soziale und psychische Entwicklung
für die jüngste Generation unserer Gesellschaft von großer Bedeutun schon im
frühen Alter die Gruppenprozesse sowie das Musizieren kennen zu lernen. Wieviel sie sich in den
ersten Lebensjahren aneignen zeigt sich alleine darin, dass sie ungefähr im
Alter von vier Jahren der Muttersprache mächtig sind. Auf die Musik übertragen
sind sich alle großen Musikpädagogen von Zoltan Kodaly bis Shinichi Suzuki
einig, dass ein Kind unbedingt so früh wie möglich mit der Musik in Kontakt
kommen muss. \enquote{In diesem Sinne gilt es, ein Kontinuum der musikalischen
Muttersprache zu entwickeln.}
\autocite[45]{ernst:die_zukunftsfaehige_musikschule} Nicht nur für die
Gruppenprozesse, sondern vor allem auch für das natürliche heranführen an die
Musik, ist die Musikalische Früherziehung besonders wichtig. Der Ort dieser
musikalischen Basisbildung sollte nicht erst in der Musikschule sein, sondern
bereits im Kindergarten oder vorher in Krabbelgruppen o.Ä.
\autocite[43]{ernst:die_zukunftsfaehige_musikschule}
Das Musizieren im Kindergarten ist zum Einen für jedes Kind zugänglich und
findet zum Anderen somit auch im Alltag integriert statt. Im Anschluss können
die Schulkinder außerdem durch eine bessere Orientierung über die Instrumente,
das Instrument auswählen, das ihnen am meisten zusagt. Diese Orientierungsphase
ist nicht zu unterschätzen und besonders heutzutage wichtig, wo die Kinder nicht
mehr vielfältig musizierend, sondern einförmig konsumierend mit der Musik
aufwachsen.\autocite[37]{ernst:die_zukunftsfaehige_musikschule}

Neben dem Umgang mit Gruppen werden im Folgenden die weiteren Grundgedanken der
EMP dargelegt. Die wichtige entwicklungspsychologische Erkenntnis nach Oerter
und Lehmann besagt, dass jeder Mensch musikalisch
ist.\autocite[88]{musikalische_begabung} Geht man von diesem Grundgedanken aus,
so soll die EMP ermöglichen, dieses musikalische Potenzial zu entwickeln, wobei
die Entfaltung der musikalischen Ausdrucksfähigkeit die Persönlichkeitsbildung
des Menschen unterstützt. Das gelingt durch das musikalische Gestalten, wozu
verschiedene Handlungsbereiche zählen, wie das sensorische Sensibilisieren, die
Bewegung, das Singen, Sprechen, das Lernen die Zeit zu strukturieren,
(Elementar-  und Körper-)Instrumente spielen, das Visualisieren sowie das Hören.
\autocite[227]{busch:grundwissen_instrumentalpaedagogik} Als grundlegendes
Musikverständnis bilden in der EMP Musik und Bewegung stets eine Einheit.

Zwei übergeordnete Erziehungsziele sind es, die psychologischen und sozialen
Ziele zu schulen. Was die beiden verbindet und vereint ist die Entwicklung von
Selbstwahrnehmung und sozialer Beziehungsfähigkeit, durch die zunehmend die
Ich-Identität ausgebildet wird. Zu den psychologischen Zielen gehört es,
Geborgenheit zu erfahren und die Selbstwahrnehmung zu fördern. Außerdem fördert
die EMP die \emph{Ich-Erweiterung} durch Identifikationsprozesse mit
verschiedenen Lebewesen, Naturerscheinungen und unbelebten Objekten. Es geht des
weiteren bei den psychologischen Zielen darum, Ideen zu entwickeln und
persönlich herausgefordert zu werden, in dem sich die Schüler*innen präsentieren
müssen und dadurch Unsicherheiten entstehen können. Es gilt zu Lernen Spannungen
auszuhalten, das Durchhaltevermögen zu steigern und Befriedigungsaufschub
ertragen zu lernen. Ein weiterer Aspekt ist es, die Initiative zu ergreifen und
Entscheidungen zu fällen, Beziehungen zur Gruppe oder einem Partner aufzubauen
und als letzten Aspekt auch das Zurücktreten und Verzichten und Selbstkontrolle
zu lernen. Dies sind alles psychologische Effekte, die im Umgang mit der Gruppe
besonders geschult und gelernt werden können; besser als im Einzel-
Instrumentalunterricht. Und doch beziehen sie sich sehr stark auf das Individuum
selbst.

Die sozialen Aspekte hingegen beziehen sich, wie schon erwähnt, auf die Gruppe.
Hier geht es darum zu lernen, die Anderen wahrzunhemen. Dazu zählt auch
aufeinander zu reagieren, in Kommunikation zueinander zu treten und aufeinander
Rücksicht zu nehmen. Das Abwarten und eine Reihenfolge einhalten, wird auch
geschult, indem sich die Gruppe selbst regulieren lernt und ein Gruppengefühl
entwickelt wird. Das funktioniert wiederum nur, wenn die Schüler*innen anfangen
sich mitverantwortlich zu fühlen und auch Ideen Anderer akzeptieren.

Um sich diesen Zielen zu nähern, geht die EMP spielerich mit Individual- und
Kollektivgestaltung um und versucht durch vielfältige Aktionsformen verschiedene
Interaktionen zu ermöglichen. Dadurch geschieht die soziale Sensibilisierung, zu
der noch die auditive Sensibilisierung hinzukommt. 

Die Sozialform der EMP ist wie eingangs erwähnt immer die Gruppe, die aber
wiederum verschiedene Spielräume für verschiedene Aktionsformen ermöglicht. Alle
Formen basieren auf dem beziehungsorientierten Arbeiten, in dem \enquote{über
die Beziehungen, die zur Musik aufgebaut werden sollen, [...] auch die Kontakte
mit anderen Gruppenmitgliedern im Fokus.}
\autocite[10]{dartsch:kern_des_musizierens} stehen. Mit dem ganzheitlichen
Menschen im Blick soll neben der angestrebten Persönlichkeitsbildung die soziale
Kommunikation ausgebaut werden. Fest steht, dass viele Kinder, die eine
musikalische Früherziehung miterlebt haben, gut mit Musik kommunizieren können
und damit sowohl im oft notwendigen Gruppenunterricht, aber auch bei
musikalischen Dialogen mit der Lehrkraft ausdrucksfähiger sind.


%S. 25%Andreas oder erstes Buch? s. 25


%"(...) sind viele der damit verbundenen Techniken und Methoden für den
%"Ernsthaften" Instrumentalunterricht durchaus hilfreich." S. 25 Anselm Ernst S.
%37 ff




%"Für die Musikschule reklamieren wir den Erziehungs- und Bildungsauftrag, in
%allen Schichten der Gesellschaft so umfassend wie möglich eine Musizierkultur
%aufzubauen" (S. 37/38) Anselm Ernst




\section{Instrumentalunterricht}
Der Instrumentalunterricht bietet viele verschiedene Formen vom Einzelunterricht (EU),
über den Gruppenunterricht (GU), bis hin zu dem sogenannten "Multidimensionalen
Unterricht". Man könnte auch soweit gehen, dass Band- Chor- oder Orchesterproben
mit einem Lehrer zu dem normalen Instrumentalunterricht zählen. Als Erstes
verknüpfen wir im europäischen System jedoch mit dem klassischen
Instrumentalunterricht den Einzelunterricht, wo ein Lehrer einen einzelnen
Schüler im Normalfall zwischen dreißig und neunzig Minuten unterrichtet. Dazu
gehe ich im Folgenden noch mehr ein. Was den Gruppenunterricht ausmacht, wird
auch näher beleuchtet werden. Fest steht auf jeden Fall, dass im Idealfall das
Lehren und Lernen flexibel gestaltet werden und sich im besten Fall der EU und
GU ergänzen. In unserem Musikschulsystem ist leider noch zu wenig Raum für die
Kombination aus individuellem- und gemeinschaftlichen Lernen. So schreibt Anselm
Ernst in seinem Buch Die zukunftsfähige Musikschule: "Wer genügend Raum und
Zeit hat, kann sich mit Ruhe, Geduld und Gelassenheit den Menschen und den
Sachen widmen". \autocite[84]{ernst:die_zukunftsfaehige_musikschule}
Bei allen Unterrichtsformen, so unterschiedlich sie sein mögen und mit ihren
jeweiligen Vor- und Nachteilen soll es darum gehen, dass der Instrumentalschüler
das für ihn am besten geeignete Lernumfeld geboten bekommt. Dazu sollte es auf
alle Fälle variabel sein und keine starre einseitige Unterrichtsform, sondern
eine Wechselwirkung aus allen Möglichkeiten, aus denen wir Musiker schöpfen
können. 


\subsection{Einzelunterricht}
Im Einzelunterricht (EU) erhält der Instrumentalschüler die volle Aufmerksamkeit des
Lehrers. Dies hat seinen Vorteil darin, dass der Lehrer auf der persönlicheren
Ebene viel mehr auf den Schüler eingehen kann, sowie spieltechnisch in einem auf
den Schüler abgestimmten Tempo arbeiten kann. Außerdem erlaubt der EU, tiefer
und detaillierter auf das ein oder andere Thema einzugehen, zum Beispiel in der
Spieltechnik oder Körperhaltung. Die volle Aufmerksamkeit des Lehrers kann jedoch auch genauso gut
zum Nachteil werden, da der Schüler ausschließlich von der Lehrperson seine
Anweisungen erhält und der Lernweg ausschließlich eindimensional stattfindet.
Der Schüler kann sich zwar auch etwas vom Lehrer abschauen und auf der
differenziellen Ebene etwas Unausgeprochenes aus dem Unterricht mitnehmen, aber
der Lehrer alleine kann gar nicht soviel anbieten wie eine Gruppe von
Mitschülern. Gerade für Schüler, die ein Problem mit der Hierarchie haben, ist
möglicherweise der Einzelunterricht nicht das optimale Lernumfeld. Andere
Schüler hingegen genießen es, endlich einmal die volle Aufmerksamkeit einer
Erwachsenen Person und gegebenenfalls sogar Vertrauensperson zu bekommen. Diese
Erfahrung habe ich besonders bei meinen Internatsschülern gemacht, sowie bei
Instrumentalschülern, die zu Hause viele Geschwister haben. 

\subsection{Gruppenunterricht} 
Beim Gruppenunterricht muss man zwischen Klein- und Großgruppen unterscheiden.
In einem Kleingruppenunterricht sprechen wir von zwei bis fünf Schülern, während
ein Großgruppenunterricht alles darüber hinaus beschreibt.
\autocite[219]{Busch:grundwissen_Instrumentalpädagogik} Laut Anselm Ernst gilt ein Gruppenunterricht ab
drei oder mehr Schülern, da
man
erst ab drei Schülern von einer Gruppe sprechen kann.
\autocite[79]{ernst:die_zukunftsfaehige_musikschule}
Allerdings gibt es zusätzlich noch das Szenario von zwei Schülern, die gemeinsam
Unterricht haben und somit weder dem Einzel-, Klein- noch dem Gruppenunterricht
zugeordnet werden können. Hier sprechen wir also dann vom
Partnerunterricht.\autocite[219]{Busch:grundwissen_Instrumentalpädagogik} Bei zwei Schülern dominiert der
Lehrer trotz allem
noch das Geschehen, aber nichts desto trotz können die Schüler gemeinsam Duos
spielen, ohne, dass die Lehrperson direkt als Duo-Partner involviert ist, wie im
(EU). "Während im Großgruppenunterricht und somit auch im Klassenmusizieren bzw.
Klassenunterricht dem Erwerb spieltechnisch-musikalischer Fertigkeiten Priorität
zukommt, ist das Spielen im Ensemble hiervon anzugrenzen, da dies das
gemeinschaftliche Musizieren fokussiert und in der Regel einen vorausgehenden
oder parallel stattfindenden Instrumentalunterricht voraussetzt."\autocite[219]{Busch:grundwissen_Instrumentalpädagogik}

Die Vorteile des Gruppenunterrichts sind sehr vielseitig. Abgesehen davon, dass
gemeinsames Musizieren einfach mehr Spaß macht und da kann ich aus Erfahrung von
meinen Schülern sprechen, die förmlich danach lechzen, wieder gemeinsam zu
musizieren. Ein weiterer Aspekt ist, dadurch, dass das soziale
Lernen in den Vordergrund rückt, verschmelzen das soziale und musikalische
Lernen. Außerdem ensteht bei den Schüler durch das soziale Miteinander, woraus
auch Freundschaften entstehen können, eine stärkere Bindung zur Musik. Auch die Übemotivation ist höher, da sich die Schüler vor den anderen
zum Einen nicht blamieren wollen und zum Anderen auch eine Verantwortung für das
Mittragen des Unterrichts übernehmen. Außerdem können die Schüler die Stücke
gemeinsam Üben, sodass sie auch beim Üben nicht alleingelassen sind und das Üben
selbst auch zusammen Üben lernen. In einem Alter, wo das Instrumentalspiel
vermeintlich uncool werden könnte, lernen die Schüler aber auch in der
Gemeinschaft, dass es auch andere Personen gibt, die ein Instrument spielen
wodurch sich die Schüler in ihrer musikalischen Bestätigung gegenseitig
bestärken. \autocite{ernst:die_zukunftsfaehige_musikschule}

Im Vergleich zum Musikunterrich in der Schule oder auch dem sogenannten
Klassenmusizieren, muss man außerdem den Vorteil sehen, dass die
Instrumentallehrer sehr flexibel sind in ihrer Gestaltung, da sie sich an keine
Lehrpläne halten müssen. Dies gilt besinders, wenn das Musizieren im Rahmen von
der Musikschule stattfindet. Für diese Freiheit sollten die Instrumentallehrer
dankbar sein.



\subsection{Klassenmusizieren}
Eine Form des Gruppenunterrichtes in Großgruppen stellt das Klassenmusizieren
dar. Es findet oft in sogenannten Bläser- oder Streicherklassen
statt, sodass die Schüler je nach Angebot der Schule auf die Familie der
Streich- oder eben Blasinstrumente festgelegt werden. Das Klassenmusizieren ist allderings
nicht zu verwechseln mit dem normalen Musikunterricht, welcher trotzdem noch
abgehalten wird.
Im Idealfall läuft es so
ab, dass die Schüler die verschiedenen Instrumente die zur Auswahl stehen kennen
und anschließend eines auswählen dürfen. Ziel ist es auch hier wieder, dass die
Kinder im gemeinsamen Musizieren, zu dem auch das Singen, Solmisieren und 
Rhythmussspielchen gehören, einen ganzheitlichen, praktischen Weg zur Musik
finden. \autocite[91]{ernst:die_zukunftsfaehige_musikschule}
Dadurch, dass das Klassenmusizieren im normalen Unterrichtsalltag integriert
wird, lernen die Schüler das Instrument, wie jedes andere Fach, welches sie
unterrichtet bekommen als selbstverständlich kennen. Dadurch, dass es im
Stundenplan integriert ist, ist es kein besonderes außerschulisches Angebot,
welches nur von wohlhabenden Eltern ermöglicht werden kann und für jeden
zugänglich. So haben die Schüler auch die Gelegenheit, die Musik wie eine
weitere Sprache dazuzulernen. Im Unterschied zum beispielsweise
Englisch-Unterricht, sind sie aber nicht auf sich alleine gestellt, vor allem
dann, wenn sie bei Prüfungen ihre eigene Leistung unter Beweis stellen müssen.
"Beim instrumentalen Klassenunterricht jedoch verwandelt sich die Klasse in der
Schüler/innen in ein Ensemble. Sie lernen gemeinsam und musizieren gemeinsam." \autocite[92]{ernst:die_zukunftsfaehige_musikschule}
Wie Anselm Ernst außerdem sehr zutreffend beschreibt, verbinden sich die
Einzelleistungen und addieren sich nicht. Dass die Schüler sich als Gemeinschaft
erleben ist der zentrale Wert dieses Musizierunterrichts!

"Es wächst die Zahl der Kinder, die über Formen des Klassenmusizierens an
Grundschulen (...)) gewonnen werden." %/S. 118 welches buch? evtl. Doerne?/


JeKids: meine Annahme, dass es Quatsch ist, einem Schüler im Klassenmusizieren
ein Isntrument beizubringen.




\subsection{Multidimensionaler Unterricht} 
Das Konzeot des Multidimensionalen Unterrichts weitet die Unterrichtsformen in
alle denkbaren Strukturen aus. Das Konzept stammt von Gerhard Wolters und
beschreibt eine gute Möglichkeit, wie wir in unseren Musikschulen den
Multidimensionalen Unterricht als eine wunderbare Ergänzung zum Einzelunterricht
etablieren könnten. Er bezieht sieben Dimensionen ein, die sehr unterschiedliche
Richtungen einbeziehen. \autocite[86ff]{ernst:die_zukunftsfaehige_musikschule} Damit
ist der zuvor beschriebene Gruppenunterricht gemeint. Wolters
spricht von der Partnerschaft zwischen zwei und mehr Personen. Hier steht das
Voneinander Lernen im Vordergrung. Die zweite Dimension bezieht sich auf den
Unterricht in mehreren Räumen. Das bedeutet aber auch, dass die Lehrperson nicht
immer anwesend sein muss, während eines Unterrichtes. Egal ob es sich um eine
Gruppe oder einen Einzelunterricht handelt. Sehr revolutionär ist auch der
Gedanke der flexiblen Unterrichtszeiten. "Flexible (längere oder kürzere)
Unterrichtszeiten passen sich an die individuellen Lernbereitschaften,
Lernrhythmen, Lerntempi usw. an."
\autocite[87]{ernst:die_zukunftsfaehige_musikschule} Außerdem hat es den
positiven Nebeneffekt, dass der Instrumentalschüler am Ende sogar längere
Unterrichtszeiten hat, als in dem gewohnten EU. Eine neue Dimension ist der
Unterricht mit mehreren Lerhkräften. Hier können die Interaktionen der
Lehrkräfte untereinander angeregt werden, was das Unterrichten für die
Instrumentallehrer durch die gegenseitigen Anregungen interessanter gestaltet.
Und auch die Schüler profitieren davon, da jede Lehrkraft seine eigenen Stärken
mitbringt und die Schüler direkt von verschiedenen Stärken profitieren können.
Was es in Kinderorchestern bereits gibt, wird in Gruppenunterrichten, wenn sie
denn angeboten werden, sehr selten praktiziert, ist das Lernen mit Partnern
unterschieldichen Alters. "Intelektuell, körperlich, emotional und sozial sind
gleichaltrige Kinder/Jugendliche sher unterschiedlich entwickelt, so dass auch
in dieser Dimension Flexibilität erforderlich ist. \autocite[87]{ernst:die_zukunftsfaehige_musikschule}
Die sechste Dimension schließt sich dem unterschiedlichen Alter an, denn Wolters
schreibt, dass auch Unterricht von Schülern in unterschiedlichen Lernniveaus
stattfinden soll. Zum einen enspricht es der Lebensrealtität und zum anderen
kann man die Schüler in diesem Zusammenhang wunderbar dazu anleiten, sich
gegenseitig zu helfen, ihnen beibringen für einander ein Vorbild zu sein, sich
gegenseitig zu motivieren und auch Verantwortung zu übernehmen. "Von
Gleichaltrigen oder Älteren lernen Kinder/Jugendliche oft besser als von der
Lehrperson." \autocite[87]{ernst:die_zukunftsfaehige_musikschule} Die letzte
Dimension beschreibt den Unterricht mit verschiedenen Instrumenten. Dadurch
"öffnet sich der Gruppenunterricht zum Ensembleunterricht."
\autocite[87]{ernst:die_zukunftsfaehige_musikschule}

\section{Ensembles}

Nach dem Einzelunterricht, manchmal erst im späteren Leben und manchmal schon in
der frühen Kindheit besuchen Instrumentalist*innen verschiedene Ensembles in
Ergänzung zum Instrumentalunterricht, oder auch unabhängig davon. Es gibt einige
Punkte, die alle Ensembles gemeinsam haben. So beudeutet das Wort
\emph{Ensemble} zunächst einmal \enquote{eine Gruppe von Dingen, die eng
zusammengehören.}\autocite{wikipedia:gruppe} Wenn man die Bedeutung weiter auf
die Kunst beziehen möchte, so steht in der Definition: \autocite{eine Gruppe von
Künstlern -wie Schauspieler, oder Musiker, die gemeinsam etwas vortragen.}
\autocite{wikipedia:gruppe} Das interessante an dieser Beschreibung ist, dass
das Wort \emph{vortragen} impliziert, dass diese Gruppe auch auf der Bühne oder
in einem anderen öffentlichen Rahmen mit ihrem Ensemble auftritt und das
Erarbeitete präsentiert. Betrachten wir den Einzel- Instrumentalunterricht, so
ist es nicht unbedingt selbstverständlich, dass der/die Schüler*in das
Erarbeitete öffentlich vorträgt. Gerade bei Privatschüler*innen werden oft keine
Vorspiele durchgeführt, da es ohne eine Institution im Hintergrund, die darauf
Wert legt, nicht immer dazu kommt, zumal auch nicht immer die räumlichen
Möglichkeiten dafür geschaffen sind. Wie sieht es jedoch mit dem
Gruppenunterricht aus? Er findet meist doch im Zusammenhang mit einer
Institution statt, sei es dem Kindergarten, der Schule oder einer Musikschule.
Interessanterweise ist es dort, wenn auch teilweise unausgepsrochen,
selbstverständlich, dass die Gruppen bei einem besonderen Anlass oder auch bei
einem dafür geschaffenen Vorspiel, ihre Stücke darbietet. Andreas Doerne
beschreibt in seinem Buch \emph{Umfassend Musizieren} das Ensemble als
\enquote{eine musikalische Ganzehit, die sich durch ein gleichberechtigtes,
dialogisches Miteinander und Zusammenspiel ihrer einzelnen Teile (Mitglieder)
konstituiert.} \autocite[62]{doerne:umfassend_musizieren} Diese Definition
trifft aber meiner Meinung nach nicht in allen Punkten auf ein Ensemble zu. So
werden wir später feststellen, dass beispielsweise in einem Orchester der
Dirigent die Spielimpulse gibt und auch auch vorgibt, wie zu spielen ist,
weshalb es in diesem Moment kein gleichberechtigtes Miteinander mehr ist. Auch
das dialogische Miteinander ist nicht in jedem Ensemble erwünscht und
selbstverständlich. Wenn wir uns beispielweise die Suzuki-Talenterziehung
anschauen, werden wir sehr schnell feststellen, dass die Lehrkraft ganz deutlich
in der überordneten dominierenden Position ist, in der das dialogische
Miteinander auf einer sehr untergeorneten Ebene nur selten erwünscht ist. Nichts
desto trotz ist das \enquote{Ensemblespiel (ist) Kommunikation mit musikalischen
Körpern und Klängen.} \autocite[62]{doerne:umfassend_musizieren} Sich dies zu
vergegenwärtigen ist sehr hilfreich, da man schnell feststellt welche zentrale
Rolle dabei die Kommunikation spielt. Die Kommunikation wiederum ist sehr eng
mit der Psyche eines jeden verknüpft, wenn er Sitautionen für sich selbst
auswertet und mit anderen abgleicht und in vielerlei anderer Hinsicht. Die
soziale Komponente ist meiner Meinung nach das wesentliche Grundelement eines
Ensembles. Welche sozialen Eigenschaften innerhalb eines Ensembles eine Rolle
spielen ist im folgenden Zitat sehr zutreffend geschildert: \enquote{Das
Ensemble ist durchdrungen vom gegenseitigen Zuhören und Mitteilen, Sehen und
Gesehenwerden, Agieren und Reagieren, Führen und Geführtwerden, Hervorbringen
und Abnehmen von Spielimpulsen, insgesamt also vom wechselseitigen Geben und
Nehmen.} \autocite[62]{doerne:umfassend_musizieren} Diese sozialen Komponenten
finden wir auch in allen folgenden Formationen wieder. Außerhalb des Musizierens
treffen sich alle Ensembles zu mehr oder weniger regelmäßigen Proben und
veranstalten auch Probenwochenenden, an denen alle über die Musik hinaus
Zusammenkommen. Bei größeren Formationen gibt es oft auch außerhalb der Musik
einen Vorstand, der für die organisatorischen Fragen verantwortlich ist und bei
Problemen versucht die Gruppe zusammen zu halten. Das ist fast vergleichbar mit
unserer Gesellschaft, in der sich das Ganze nur in anderen Dimensionen abspielt.


\subsection{Orchester} 

Ein Sinfonieorchester ist, sofern man es überhaupt noch zu einem Ensemble zählen
kann, womöglich die größte Gruppierung von Musizierenden die zusammenspielen.
Aufgrund der Größe, trägt eine klare Hierarchie zur kleineren Unterteilung
innerhalb dieses großen Organismus bei. Aufgrund der Größe ist ein Dirigent, bei
dem vorne der Klang zusammenläuft eine wichtige Schlüsselfigur des Orchesters,
da er die Musik zusammenführen und regulieren kann. Die einzelnen Spieler*innen
sind ja mit ihrer eigenen Stimme beschäftigt und sitzen teilweise sehr weit von
anderen Stimmen entfernt, sodass die Verantwortung bei dem/der Dirigent*in
liegt, dies zusammenzubringen. Sowohl im Timing als auch in der Dynamik. Das hat
zur Folge, dass die Aufmerksamkeit der einzelnen Spieler*innen neben den
Stimmführer*innen auf den/die Dirigent*in gerichtet ist, auf die Gestaltung des
eigenen Parts, auf das Verfolgen der Mitspieler*innen und auf den
Orchesterklang. \autocite[56]{doerne:umfassend_musizieren} Wie wir sehen, ist
die Aufmerksamkeit also auf sehr viele verschiedene Felder gelenkt, die sich in
ihren Schwerpunkten immer wieder verschiebt. Das ist für mich persönlich das
Spannende am Orchesterspiel, was so viel Freude bereitet und weshalb es einfach
nie langweilig wird und immer spannend ist!

Für Insrumentalist*innen aller Art und jeden Niveaus gibt es Schulorchester,
Kinderorchester, Jugendorchester, Studentenorchester und Laienorchester. Oft
sind diese jedoch in größeren Städten und in ländlicheren Gegenden nicht ganz so
weit verbreitet. Außerdem ist für ein Sinfonieorchester vorausgesetzt, dass man
ein Instrument erlernt hat, welches in der typischen Bestzung eines Orchesters
vorgesehen ist. 

\subsection{Musikverein}
Die Musikvereine sind hingegen oft vermehrt in ländlicheren Regionen zu Hause.
Es gibt bei ihnen drei verschiedene Instrumental-Kategorien. Zum einen die
Akkordeonvereine, die Blasmusikvereine, sowieo die Fasnachtsgruppierungen. In
den Akkordeonvereinen spielen zwar alle erst einmal das selbe Instrument, aber
doch in ihren unterschiedlichen Stimmen und auch leicht unterschiedlich in der
Funktionsweise, da die Bässe beispielweise verstärkt sind. Manchmal gibt es noch
Trompeter und Schlagzeuger dazu, aber das ist eher die Ausnahme. Bei den
Fasnachtsgruppen, kommen die Angehörigen einer Zunft zusammen und musizieren
gemeinsam. Die Hauptinstrumente hierbei sind Trommeln, Piccolo-Flöten und
Blechblasinstrumente. Sie arbeiten meist nur einmal im Jahr und auf den
Fasnachts-Anlass hin. Im Blasmusikverein sind ähnlich wie im Orchester
verschiedene Instrumentengruppen vertreten aus der sich heraus eine klare
Hierarchie innerhalb der Gruppen bildet. Außerdem gibt es auch eine/n
Dirigent*in, welcher die Proben anleitet. Im Vergleich zum Orchester sind aber
nur Blasinstrumente, die gespielt werden und ein Schlagzeug zuzüglich. Das
Schöne am Musikverein ist die heterogene Zusammensetzung der Mispieler*innen.
Hier kommen Jung und Alt zusammen, Anfänger und Fortgeschrittene musizieren
gemeinsam. Außer den Landespolizeiorchestern oder beim Militär, gibt es bei
Blasmusikvereinen fast nur Laiengruppierungen. Dadurch, dass also viele
Musiker*innen in ihrer Freizeit im Verein spielen, gehört neben dem eigentlichen
Musizieren auch das Vereinsleben dazu, welches ich von allen hier aufgezählten
Ensembles als das am stärksten ausgeprägteste einordnen würde. Das soziale
Miteinander entsteht somit durch die Musik, aber wird auch darüber hinaus
bewusst gepflegt und gestärkt. Da die Blasmusikvereine eher in ländlicheren
Gegenden verortet werden, spielen sie dort häufig zu besonderen Anlässen im Ort
oder aber auch im Zusammenhang mit Fasnacht zu bestimmten Umzügen. Die Konzerte
sind demnach mit einem festlichen Charakter verbunden, aber trotz allem mit
einem adequaten Grad in der Erwartung an die Leistung der Spieler*innen. Oft
sind Blasmusikvereins-Konzerte auch in einem lockeren Rahmen, wo das Publikum
parallel zum Konzert Essen und Trinken bekommt. Das lockert zum einen die
Athmosphäre und wird dadurch sowohl für das Publikum als auch für die
Musizierenden zu einem gemeinschaftlichen, geselligen Zusammenkommen, welches
von Musik begleitet wird. Wenn man in die Historie zurückblickt, gibt es viele
Hauskonzerte und auch Opernvorstellungen, während denen das Publikum essen,
trinken und laut sein durfte. In diesen Momenten nimmt die Musik eine andere
Rolle ein, als in den Konzertsälen, in der das Publikum die ernste Musik mit
klaren Applausregeln zu honorieren und ansonsten still zu sein hat. Außerdem ist
sie auf diese Art und Weise sehr natürlich in den Alltag mit eingebunden,
während ein Orchesterkonzertbesuch direkt ein spezieller besonderer Anlass ist.
Auch die Literatur ist in den Musikvereinen näher an den Bedürfnissen der
breiten Gesellschaft, da oft Filmmusik oder andere typische, bekannte
Gassenhauer gespielt werden. Durch die Athmosphäre der Musik wird das Publikum
dadurch auch anders und womöglich besser erreicht, als bspw. bei ernster
klassischer Musik. 


\subsection{Chor}
Wie beim Orchester haben wir beim Chor auch einen Chorleiter, welcher die
Aufwärmübungen und die Probenarbeit mit dem Ensemble anleitet. Parameter, wie
sich in den Klang seiner Gruppe einzuordnen sind auch ähnlich zu anderen
Ensembles, sowie das gemeinsame gleichzeitige und möglichst synchrone Agieren.
Technisch kann man sich jedoch nicht ganz soviel abschauen, da das Singen und
wie es funktioniert nicht so offensichtlich ist, wie das spielen eines
Instrumentes, wo jede Haltung imitiert werden kann. Die Körperhaltung, der Stand
an sich kann beim Singen zwar imitiert werden aber darüber hinaus gilt es seine
Stimmbänder zum klingen zu bringen, was eine sehr individuelle und wenig
sichtbare Angelegenheit darstellt. Musikalisch kann man hingegen viel dazulernen
und gerade die technische Einschränkung die Laien oftmals mit ihren Instrumenten
haben, sind beim Singen etwas weniger eine Barriere. Das ist sicher auch ein
Grund dafür, warum es sehr viele Laienchöre gibt, egal ob in einer größeren
Stadt oder auf dem Dorf. Die Voraussetzungen um in einem Chor mitzusingen sind
oft sehr offen und man muss nie aktiven Gesangsunterricht gemacht haben, um in
einem Laienchor mitwirken zu dürfen. Damit ist das Chor-Ensembles das, was den
einfachsten Zugang zur Musik ermöglicht. 


\subsection{Band}
Mit einer Band assoziiert man direkt verschiedene Genres außerhalb der
klassischen Szene und ein soziales Miteinander, welches im Vergleich zu vielen
anderen Ensemblegruppen nicht unbedingt von einer einzigen Person angeleitet
wird. Im Zusammenhang mit dieser Ensemblearbeit verbinde ich auch das kreative
Arbeiten, die Improvisation und das forschen und ausprobieren, wie man seine
eigenen Stücke mit der Gruppe schreiben, komponieren und umsetzten kann. Da eine
Band aber meist auch nur aus fünf bis zehn Personen besteht, ist diese
Interaktion und der Austausch auch möglich, ohne dass es zu komplex wird. Bei
einer Band sind die Spieler*innen meistens mit verschiedenen Instrumenten am
Musizieren, was ein wichtiger Unterschied zu Orchestern, Chören und
Blasmusikvereinen ist, wo wir zumindest immer eine Gruppering von gleichen
Instrumenten haben. Dadurch, dass jede/r für sein/ihr Instrument verantwortlich
ist und sich die Mitspieler*innen nur mit dem eigenen Instrument auskennen, kann
man sich nur über die Sprache der Musik gegenseitig weiterhelfen. Nicht aber
über konkrete technische Anweisungen. Diese Hürden muss man entweder selbst
überwinden, oder eine Lehrkraft um Hilfe bitten, wenn die Musik als Sprache
selbst nicht mehr weiterhilft. Aber es steht auch viel mehr die Musik selbst im
Vordergrund und es ist zweitrangig, wie man zu seinem Ziel gelangt. Durch die
kreative Herangehensweise müssen die Spieler*innen außerdem eine klare
Vorstellung entwickeln, was sie wie haben möchten und diese wiederum mit ihrem
Mitspielern kommunizieren. Oft haben Musizierende diese klare Vorstellung nicht
von Anfang an, aber es zeichnet Bandproben aus, dass man gemeinsam auf die Suche
danach geht. Dieser Ideenaustausch ist sehr wertvoll und besonders zum
Einzelunterricht eine wunderbare Ergänzung.



\subsection{Sonstige Formationen}
Denkt man an die Instrumente, die auf den ersten Blick nicht für Ensembles bzw.
Gruppenunterrichte geeignet sind, wie zum Beispiel das Klavier oder die Gitarre,
so kommen mir trotzdem und zum Glück doch einige Gruppierungen aus meinem Umfeld
in den Sinn, die sich über die gängigen Formationen nicht abschrecken lassen und
neugierig genug sind, sich ein Ensemble zu suchen. Mit dem Klavier gibt es im
klassischen Sektor wenigstens noch die sparte der Klavier-Kammermusik, von
Klaviertrio bis hin zum Klavierquintett, in anderen Genres ist es aus Bands
nicht wegzudenken. Aber darüber hinaus wird es eher wieder schwieriger und auch
bei den jungen Klavierschüler*innen, ist das Ensemblespiel leider ein noch viel
zu sehr vernachlässigter Baustein. Wenn man bedenkt welche ganzen sozialen und
auch musikalischen Kompetenzen im Ensemblspiel geschult und gelernt werden, so
muss ich sagen, dass es in Zukunft unbedingt auch für Klavierschüler
Möglichkeiten und Räume zum gemeinsamen Musizieren geschaffen werden müssen!
Betrachten wir aber die sparte der neuen Musik, so stelle ich fest, dass gerade
hier immer wieder untypische, neue und besondere Besetzungen von Fagott über
Klavier, Schlagzeug und beispielsweise Geige immer gängiger werden und besonders
beliebt sind. Allerdings sind die Anforderungen an das Niveau in der Neuen Musik
meist leider so hoch, dass sie für Instrumentale Anfänger*innen oft nicht
spielbar sind. Bei dem Klavier kommt außerdem noch hinzu, dass man aufgrund der
Größe des Instrumentes leider kaum mehr als zwei Klaviere bzw. Flügel in einem
Raum stehen haben. Wenn man aber an die Familie der Keyboards denkt, sind diese
wiederum sehr vielseitig einsetzbar.

Vergleichen wir das Klavier beispielsweise mit der Gitarre, die es in der
klassischen Musikszene als Ensembleinstrument auch nicht leicht hat, so kann man
wenigstens mit mehreren Schülern ein Gitarrenensemble zusammenstellen und
gemeinsam musizieren. Dass die Gitarristen aber auch Pianisten in den anderen
Genres ihren Platz meist in einer Band haben, vor allem auch im Populären
Musikbereich, ist nicht außer Acht zu lassen. Auch in anderen Kulturen spielt
die Gitarre eine sehr ausgeprägte Rolle. Das Schlagzeug ist ein weiteres
Instrument, welches es im Ensemble nicht immer leicht hat. Immerhin hat jedes
Orchester, jede Band und jeder Musikverein einen Schlagzeuger. Aber mit seinen
eigenen Schlagzeugkolleg*innen zusammen zu spielen, ist außerhalb des
Musikstudiums nicht unbedingt üblich. 

Wir können daraus mitnehmen, dass die Instrumente in den verschiedenen Genres
andere Funktionen und Einsatzmöglichkeiten haben. Deshalb sollten
Instrumentalschüler*innen an verschiedene Genres herangeführt werden um einen
möglichst breiten Horizont zu haben und durch diesen die Möglichkeiten sehen,
die ihr Instrument auch im Zusammenhang von Ensembles bietet. 







	\addchap{Analyse verschiedener Systeme}


\section{Jedem Kind sein Instrument}


"Es wächst die Zahl der Kinder, die über Formen des Klassenmusizierens an
Grundschulen (...)) gewonnen werden." S. 118 

JeKids: meine Annahme, dass es Quatsch ist, einem Schüler im Klassenmusizieren
ein Isntrument beizubringen.

Instrumentaler Klassenunterricht
Anselm Ernst S. 91 ff

\section{El Sistema}

\subsection{Gründer José Antonio Abreu}
\enquote{Ursprünglich wurde Musik von einer Minderheit für eine Minderheit gemacht. Dann
wurde es zur Kunst einer Minderheit für die Mehrheit, und jetzt stehen wir am
Anfang eines neuen Zeitalters, in dem Musik das Vorhaben einer MEhrheit für die
Mehrheit ist.} \autocite[5]{kaufmann:el_sistema}
Das sind die Worte des Visionärs, der El Sistema in die Welt gerufen hat. Aus
dem Zitat geht deutlich hervor, dass er die Musik vor allem für die Mehrheit
zugänglich machen will und das erreicht El Sistema ohne Weiteres, da es
weltweiten Anklang findet und internaionale Wellen schlägt.
Abreu stammt aus einer sehr musikaffinen Familie, angefangen bei seinen
Großeltern, die ihre Vorliebe zu der Opernwelt aus Italien mitbrachten. Auch
Abreus Mutter und Vater selbst musizierten, wo sowohl die klassische Musik der
Großeltern aus Europa eine Rolle sppielte, als auch die lateinamerikaniche
Volksmusik, die sie umgab. Mit neun Jahren begann Abreu das Klavierspiel zu
lernen. Seine erste Musiklehrerin prägte ihn sehr. Vor allem ihr Leitsatz
"Niemand ist unmusikalisch, jeder kann ein Instrument erlernen - es gibt keine
Disqualifikation durch Unbegabtheit, sehr wohl aber ein Fördern und Fordern in
unterschiedlicher Ausprägung." \autocite[20]{kaufmann:el_sistema} Außerdem
schaffte es die Klavierlehrerin ihren Schülern mit auf den Weg zu geben, dass
man als Pianist nicht zwangsläufig ein Einzelgänger sein muss und sich durch die
Arbeit mit Chören integrieren, bei Theateraufführungen und eigentlich bei jeder
Art künstlerischer Betätigung mitwirken kann. Neben dem
Klavier lernte Abreu aber auch bald noch Geige.
Aufgewachsen in Barquisimeto, "das sich selbst als die musikalische Haupftstadt
Venezuelas bezeichnet(...)." \autocite[22]{kaufmann:el_sistema} hatte Abreu
durch eine sehr gute Musikschule die Möglichkeit Geige bei einem aus Riga
stammenden Geigenlehrer zu lernen. Außerdem gab es an dieser Musikschule auch
ein Orchester, in dem er bald mitspielen durfte, obwohl er das Geigenspiel
gerade erst dabei war zu lernen. Er wurde aber von seinen Mitspieler und
besonders vone einer anderen Geigerin so gut an
die Hand genommen, dass er keine frustrierenden Erfahrungen sammeln musste.
Dieses Prinzip wurde auch bei der Gründung seiner Kinder- und Jugendorchester
später zu einem wichtigen Teil der Methode. Später studierte Abreu Klavier,
ORgel, Cembalo, Dirigieren
und Komposition, sowie Ökonomie in Caracas. In der
Hauptstadt ist er auch zum ersten Mal mit dem Elend der Vorstadt in den
sogenannten Barrios konfrontiert. "Doch auch die Zustände innerhalb des
professionellen Musikbetriebs in Caracas, die sich ihm durch seine Arbeit mit
dem Orquesta Sinfónica Venezuela und in seinen Kammermusikprogrammen
erschließen, berühren ihn zutief: Der Musikbetrieb verweigert dem eigenen
NAchwuchs die berufliche Ausübung und bevorzugt eingewanderte Musiker für die
Positionen in Orchestern."\autocite[28]{kaufmann:el_sistema} Nach seinen
umfassenden Studien und teilweise auch schon parallel beginnt Abreu auch zu
arbeiten und wird später zum Vorsitzenden der Wirtschaftskommission in der
Finanzabteilung des Abgeordnetenhauses. Dennoch bleiben ihm die schwierigen
Umstände des Landes und vor allem der Kinder immer noch belastend im Kopf und
die Lösung des Problems sieht er in einer guten Ausbildung, in der die Menschen
dazu erzogen werden ihr Leben selbstbestimmt und selbstbewusst zu führen.\autocite[31]{kaufmann:el_sistema} 



\subsection{El Sistema}
El Sistema ist eine Orchesterbewegung,
die weltweil viele Wellen schlug und für viele Länder zum kultur- und
sozialpolitischen Vorbild wurde, da es Kindern und Jugendlichen in einem von
schweren sozialen Problemen betroffenem Land durch die Ermöglichung einer
musikalischen Ausbildung eine gesellschaftsrelevante
Initiative schuf. 

Gründung: 
Die Gründungsidee entfaltete sich in Abreu über einen längeren Zeitraum und
wurde durch das Leiten eines Kammermusikensembles, in dem auch seine Schwestern
mitwirkten immer präsenter, zumal sie sich viel darüber austauschten und
philosophierten. Das Ensemble bestand aus acht Musikern, die Abreu also schon
als potentielle Lehrkräfte zu Hand hatte. \autocite[34]{kaufmann:el_sistema}
Was sein Konzept betrifft, so ist das gemeinsame Musizieren der zentrale
Gedanke. "Seine These, dass das Erlernen von Instrumenten am besten im
Zusammenspiel mit anderen und folglich im Orchester erfolge, scheint ihm immer
wieder aufs Neue richtig."\autocite[34]{kaufmann:el_sistema} Er wollte, dass die
Kinder die "selbst die Kraft erfahren dürfen, die sich beim gemeinsamen
Musikmachen entwickelt."\autocite[34]{kaufmann:el_sistema} Mit seinem Plan
wollte er gleich zwei sozialipolitische Themen lösen: er wollte den Kindern
durch die Musik zu einer besseren beruflichen Zukunft verhelfen und betrachtete
gleichzeitig das gemeinsame Musizieren als Werkzeug für eine bessere Bildung und
grundsätzliche Sozialisierung von Kindern vor allem aus den Armenvierteln. Seine
Gründung sollte also ein "Ausgangspunkt für eine strukturelle Veränderung in der
Aullgemeinbildung, in der musikalischen Bildung und im professionellen
Musikangebot Venezuelas" sein. \autocite[38]{kaufmann:el_sistema} Was gerade die
Benachteiligten angeht so war es Abreu ein Anliegen, dass aus den vielen
Niemanden in den Elendsvierteln viele Jemande werden, "die durch das gemeinsame
Musizieren an ihre eigene Zukunftsfähigkeit zu glauben lernen."\autocite[39]{kaufmann:el_sistema}

Die Ausbildung in der klassischen Musik soll den besonders verarmten Kindern aus den Vororten
eine Perspektive verschaffen. Das System in Venezuela sieht so aus, dass die
Kinder vier Stunden Proben und Unterrichte direkt nach der Schule haben und am
Wochenende zusätzliche Probentermine, die variieren. Abreus Ansicht und Vision ist, dass die
Musik als Vermittler von sozialer Entwicklung in der höchsten Form betrachtet
werden sollte, da sie die höchsten Werte wie Solidarität, Harmonie und
wechselseitiges Verständnis überträgt und die Fähigkeit dazu besitzt, eine
Gemeinschaft miteiander zu verbinden und große Gefühle auszudrücken. Das Motto
von der Organisation war "Play and fight", was die Entschlossenheit hinter "El
Sistema" zu stehen zum Ausdruck bringen soll, sowie die Vitalität und
gleichzeitig auch den politisch kritischen Geist zum Ausdruck bringt. Der
heutzutage sehr bekannte Dirigent Gustavo Dudamel wurde mit "El Sistema" groß
und sagt "music saved my life and has saved the lives of thousands of atrisk
children in Venezuela ... like food, like health care, like education, music has
to be a right for every citizen."%\autocite{wikipedia}% Das Besondere an diesem
Das Besondere an dem Projekt ist wie schon angedeuted wird, dass es eben auch
gerade in den Vororten von Caracas die Kinder mit ins Boot holte, die sonst
aufgrund ihrer schwierigen Lebensumstände und Armut keine Gelegenheit gehabt
hätten, mit der (klassischen) Musik in Berührung zu kommen und gleichzeitig
einen Weg aus der Armut, dem Drogenmissbrauch und der Kriminalität heraus zu
finden. Aus El Sistema geht auch das bekannte Simon Bolivar Orchester hervor,
das nach und nach Erfolg erntete und international
immer bekannter wurde, bis es zuletzt in großen Konzertsälen Europas
konzertieren durfte, bekam es auch immer mehr finanzielle Unterstützung von der
venezuelanischen Regierung. Das pädagogische Programm weitete sich aber nicht
nur inlands aus, sondern fand bald in Amerika, Kanada, England, Kolumbien,
Portugal, Phillipinen und Peru seine Nachfolger und führte somit weltweit
hunderttausende von Kindern an die Musik heran. Etwas umstritten ist das Projekt
jedoch auch, worüber Geoffrey Baker neben einem Zeitungsartikel ein ganzes
Buch verfasste. Darin beklagt er den kulturellen Autoritarismus, die
Hyper-Disziplin, die Ausbeutung, den Wettbewerb und die
Geschlechter-Diskriminierung. 

\subsection{Das Lernen in einer musikalischen Praxisgemeinschaft}
Die Ausbildung in El Sistema funktioniert so, dass die Schüler keine Musikschule
besuchen, wie wir es hier in Deutschland kennen, sondern den "Nucleo", also den
den Kern oder das Zentrum der sogenannten "Orchesterbewegung" besuchen. Hier
spielen die Schüler von Beginn an in einem "richtigen" Orchester mit, wo
ernsthaft musiziert wird. Neben den Orchesterproben im Tutti gibt es sowohl
Stimmproben als auch begleitenden Einzelunterricht. Geleitet werden die Proben
von professionellen Musikern aber auch von fortgeschrittenen Instrumentalisten.
\autocite[45]{kaufmann:el_sistema} Der Vorteil dabei ist, dass die
fortgeschritteneren Musiker sich nicht weiter langweilen müssen, sondern im
nächsten Schritt zum Nachdenken über eine sinnvolle Vermittlung ihrer
Fähigkeiten angeregt werden. Außerdem erfahren sie durch die Rolle des Lehrens
auch nochmal neue Kompetenzen und haben die Möglichkeit ihr Instrument weiter zu
ntdecken, indem sie durch die Anfänger womöglich mit Fragestellungen und
Problemen konfrontiert sind, die ihnen selbst bis dahin gar nicht begegnet sind.
Die Anfänger hingegen werden zwar ins kalte Wasser
geworden, aber gleichzeitig auch von der Gruppe gehalten und unterstützt.
%Vom wilden Lernen/Musizieren Lernen-auch außerhalb von Schule und Unterricht,
%Herausgegeben von Natalia Ardila-Mantilla und Peter Röbke.%
Das was die Orchesterbewegung ausmacht ist, dass es eine wirkliche
Praxisgemeinschaft ist, in der die Musik allgegenwärtig ist und fester
Bestandteil des Alltages ist. Diese musikalische Praxisgemeinschaft zeichnet
sich neben ihrer Komplexität durch ihre Heterogenität aus. Die Vielfalt durch
die unterschiedlichen Persönlichkeiten mit ihren jeweiligen Neigungen und
Bedürfnissen ergibt. %vom wilden Lernen s. 161
Außerdem liegt bei dem orchestralen Musizieren der Fokus weniger auf dem
systematischen Erwerb von technischen Fähigkeiten. Viel mehr dient das
Instrument als Mittel zum Zweck und kann mehr wie ein Werkzeug betrachtet
werden. "Lernen ist der Prozess, in dem man die vollständige Teilhabe an dieser
Praxisgemeinschaft erwirbt." Im besten Fall werden die Neuen von den Alten gut
aufgenommen, sodass es zu einem fruchtbaren Wechselspiel zwischen Erfahrenen und
Unerfahrenen kommen kann. Aber vor allem für den Anfänger kann ein großer Sog
entstehen und das Bedürfnis des Dazugehörens ein ausgeprägter Wunsch werden.
Durch die heterogene Gruppierung ergibt sich auch, dass die Praxisgemeinschaft
keine Teilnahmebedingungen bestimmt. Vielmehr versucht jedes Orchestermitglied
so viel und so gut es kann sein Bestes zu geben und beizutragen. Durch die
vielen Proben, die absichtlich viel Zeit der Kinder beansprucht, soll das
Musizieren wie eine zweite Muttersprache werden, die immer präsent ist.%S.162
Außerdem will Abreu durch sein "System" die "die theorielastige und auf den
bekannten Instrumentalunterricht fixtierte Methodik verworfen und durch das
Prinzip des gemeinsamen Musizierens ersetzt (werden.)" \autocite[45]{kaufmann:el_sistema}


Fortschritt des einzelnen Schüler?

\subsection{Übertragung nach Deutschland}


\section{Suzuki}

"Auch in größeren Gruppen finden ein- und zweipolige Interaktionen statt."
(S.30, die kunst zu unterrichten)


\section{Gemeinsames Musizieren im Online-Instrumentalunterricht}

Der digitale Instrumentalunterricht ist ein noch sehr neues Format, das
ausgelöst durch die Corona-Pandemie alle Instrumentallehrer*innen in Deutschland vor neue
Herausforderungen gestellt hat. Durch die Lockdowns waren die Lehrkräfte
gezwungen ihren Schüler*innen das Instrument online weiter beizubringen, während
sämtliche Ensemble-, Band- , Orchester oder Musikvereinsproben erst einmal auf
Eis gelegt waren, da keine größere Personenanzahl zusammentreffen durfte.
Allerdings haben mir alle bekannten Lehrkräfte es geschafft, diese Zeit zu
überbrücken -manche besser und manche schlechter. Der
online-Instrumentalunterricht ist durch die technischen
Möglichkeiten heutzutage gut möglich, aber die technische Ausrüstung spielt
dabei doch eine Rolle. Es gibt viele verschiedene Programme und Medien, die den
Instrumentalunterricht bereichern können. Dazu zählen zum Beispiel die App
\emph{Garage Band}, das Programm \emph{DAW} (Digital Audio Workstation),
verschiedene Hardware-Produkte wie zum Beispiel Mikrofone, Laptops, Kopfhöhrer
usw. Nachdem die Lockdowns wieder vorbei waren und der Instrumentalunterricht
wieder in Präsenz stattfinden konnte, meint man zunächst es hätte sich nichts
geändert. Auch das gemeinsame Musizieren konnte wieder fortgesetzt und
aufgenommen werden. Aber wie mir von vielen Seiten rückgemeldet wurde: die
gemeinsame Musizierpraxis hat den Schüler*innen extrem gefehlt. Dadurch, dass
auch der Schulunterricht phasenweise online ablief, fehlte den Lernenden die
gesamte soziale Interaktion und Kommunikation im realen Leben. Die Musikschulen
sind wieder zu ihrem Präsenzunterricht zurückgekehrt und auch die Ensembles
können wieder ihre Proben aufnehmen. Im Internet sprießen inzwischen aber immer
mehr Online-Musikschulen aus dem Boden, die ausschließlich den
Instrumentalunterricht über die Plattformen online anbieten, dazu zählen zum
Beispiel \emph{kinder-musikunterricht.de}, \emph{musikmachen-online.de} oder
\emph{MyMusicSchool.com}. Bei \emph{MyMusicSchool.com} wird ausschließlich
Online-Einzelunterricht angeboten, während sich die Online-Musikschule
\emph{kinder-musikunterricht.de} vor allem an sehr jungen Kindern und damit der
musikalischen Früherzierhung orientiert, weshalb sich wiederum alle Angebote auf
Gruppenunterricht beziehen. Die Altersspanne der Zielgruppe liegt bei dieser
Musikschule zwischen drei Monate bis hin zum Melodikaunterricht, der ab einem
Alter von sechs Jahren besucht werden darf. \autocite{online_musikschule_emp}
Die Seite \emph{musikmachen-online.de} bietet als eine der wenigen
Online-Musikschulen verschiedene Möglichkeiten zum gemeinsamen Musizieren an.
\autocite{online_musikschule_mmo} Bei allen Gruppenangeboten, die von Hausmusik
über Saxophon-Quartett, Bandproben bis zu Impro-Workshops reichen, wird nie
gleichzeitig online musiziert, da dies rein technisch nicht machbar ist.
Stattdessen arbeiten sie mit Playalongs, zu denen die Schüler*innen für sich in
\emph{Breakout-Rooms} spielen können und wo sie Einzeln Feedback von der
Lehrperson erhalten. Die "Gruppen-Calls" laden dann zum Austausch und zur
gegenseitigen Inspiration ein. \autocite{online_musikschule_mmo}


\subsection{Digitale Kettenkomposition}
Eine weitere Möglichkeit das gemeinsame Musizieren in den
Online-Instrumentalunterricht zu integrieren ist die \emph{Digitale
Kettenkomposition}, die ich, wie im Vorwort bereits angedeutet, zusammen mit
Timo Langpap entworfen habe. Wir haben unsere Instrumentalschüler*innen mit
ihren verschiedenen Instrumenten eigene Musik entwerfen lassen und wie einen
Kettenbrief für die anderen Schüler*innen zugänglich gemacht, sodass sie sich
wiederum ein Stück aussuchen konnten, zu dem sie dann wieder etwas improvisiert
und das dann aufgenommen haben. So entstanden musikalischen Kettenkompositionen,
die bis zu sieben Mitwirkende hatten, während andere Stücke nur von zwei
Schüler*innen gestaltet wurden. Trotz allem war das ganze leider sehr anonym und
die Schüler konnten sich bis zuletzt nicht gegenseitig kennenlernen. Positiv
hingegen war, dass die Schüler*innen neugierig auf die Musik der anderen
Schüler*innen waren und selbst kreativ werden konnten, was bei allen Beteiligten
einen großen Motivationsschub veranlasst hat, da es doch ein größeres
übergeordnetes Projekt und Ziel gab, auf das mehrere Instrumentalist*innen aus
verschiedenen Orten gemeinsam hingearbeitet haben. 

\subsection{Pro und Contra}
Gemeinsames Musizieren im Online-Instrumentalunterricht ist aufgrund der Latenz
nicht möglich, soviel steht fest. Dabei ist es zunächst einmal egal, ob es
Online-Einzel- oder Gruppenunterricht ist. Das wird auch auf der Seite von
\emph{Musik Machen Online} bei allen Gruppenangeboten deutlich. Allerdings ist
der soziale Austausch von Erfahrungen auch schon sehr wertvoll und ein
Online-Gruppenangebot deshalb trotz allem sinnvoll! Leider bieten aber nicht
viele Online-Musikschulen den gemeinsamen Gruppenunterricht an. Die
Online-Musikschule \emph{kinder-musikunterricht.de} macht zwar den Anschein und
bietet ausschließlich Gruppenunterrichte an, zeigt jedoch auf ihrer Website
nicht auf, wie sie den Online-Gruppenunterricht zu gestalten denken und
vermittelt dem Interessenten damit einen ziemlich falschen Eindruck. Außerdem
sehe ich den Online-Instrumentalunterricht gerade in den ersten Lebensjahren
eines Kindes sehr kritisch, da die Kinder nicht vor dem Laptop von einer
Lehrkraft etwas vorgespielt bekommen sollten, sondern es viel wertvoller ist
beispielsweise das soziale Verhalten im Zusammentreffen mit anderen Kindern zu
schulen, womit auf der Inernet-Seite geworben wird. Dieses soziale Verhalten
können Kinder meiner Meinung nach aber nicht online lernen, sondern nur im
realen Austausch und in der Interaktion mit anderen Kindern. Außerdem sollten
die Kleinkinder nach meiner persönlichen Ansicht nicht zu früh vor dem Computer
oder Laptop sitzen, da sie Videocalls noch nicht abstrahieren und einordnen
können. 

Der Großteil der Angebote des Online-Instrumentalunterrichts bietet
ausschließlich Einzelunterricht an. Dieser Online-Markt zestört das gemeinsame
Musizieren, da es wie eingangs erwähnt nicht einmal im Einzelunterricht zwischen
der Lehrkraft und dem/der Schüler*in zustande kommen kann. So kann ein Anfänger
nur sehr schwer in einen \emph{Spielflow} gelangen, geschweige denn eine
Klangvorstellung entwickeln. Ein weietrer Aspekt, der im
Online-Instrumentalunterricht schwierig zu erfassen ist, ist die Einschätzung
von einem neuen Lehrer. Da die Körperhaltung, der Klang der Stimme und viele
weitere Parameter durch das Internet sehr verfälscht werden, ist es viel
schwieriger eine persönliche Beziehung zu einer Lehrperson aufzubauen. Dazu
zählt auch, dass man sein Gegenüber nur in einem bestimmten Ausschnitt
wahrnehmen kann und nicht die Person als Ganze. Auch für die Lehrperson ist es
dann viel schwieriger, einzuschätzen welcher Umgang dem/der Schüler*in am besten
entsprechen würde. Und umgekehrt, kann der/die Lernende manches Gesprochene von
der Lehrkraft ggf. auch nicht richtig deuten und einschätzen.
Diese Anonymität sehe ich
deshalb am Online-Einzelunterricht sehr kritisch.
Das Angebot ist ein weiteres
Beispiel dafür, dass viele Instrumentalpädagog*innen noch nicht begriffen haben,
wie essentiell das gemeinsame Musizieren ist. Andererseits ist die Hemmschwelle
ein Instrument neu zu Lernen etwas herabgesetzt, da man zu Hause in seiner
gewohnten Umgebung und ohne große Fahrtwege ein Instrument lernen kann. Wenn
dadurch mehr Menschen das Musizieren für sich entdecken und es wieder mehr in
den Alltag von der ganzen Gesellschaft integriert ist, profitieren mit etwas Glück
all die verschiedenen Ensembles und bekommen mehr Zuwachs. Dies beglückt im
besten Fall dann wieder die einzelnen Spieler*innen, sodass sich ihre Motivation
wieder stärkt. Online-Einzel-Unterricht in Kombination mit einem Ensemble, in
dem man zusammenkommt und gemeinsam im echten Leben und gleichzeitig (ohne
Latenz) musiziert, schätze ich wiederum als sinnvoll ein, zumal es ein Modell
ist, was zeitgemäß ist. 






	\addchap{Fazit}


- Ausbildung nicht gut genug S. 119/ S. 94
- EMP hilft und bietet Potential/Techniken

	




	\clearpage
	\addcontentsline{toc}{chapter}{Literatur}
	\printbibliography

	% \clearpage
	% \addcontentsline{toc}{chapter}{Stichwortverzeichnis}
	% \printindex

	% \clearpage
	% \addcontentsline{toc}{chapter}{Verzeichnis der Notenbeispiele}
	% \listoffigures

	\section*{Erklärung}

\pagenumbering{gobble}

Hiermit erkläre ich, dass ich diese Arbeit selbständig verfasst habe, keine
anderen als die angegebenen Quellen und Hilfsmittel verwendet habe und alle
Stellen, die wörtlich oder sinngemäß aus veröffentlichten Schriften entnommen
wurden, als solche kenntlich gemacht habe. Darüber hinaus erkläre ich, dass
diese Arbeit nicht, auch nicht auszugsweise, bereits für eine andere Prüfung
angefertigt wurde.


\end{document}
