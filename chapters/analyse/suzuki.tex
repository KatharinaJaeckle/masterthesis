\section{Suzuki}

- Shinichi Suzuki als Person?
Bevor wir Suzukis Methode anschauen, will ich noch kurz auf seine Person
eingehen. Er wuchs in Japan auf, studierte aber seit seinem 23ten Lebensjahr nach dem ersten Weltkrieg (zwischen seinem 20. und 30 Lebensjahr)
in Berlin, wo er später auch seine Ehefrau kennenlernte. Außerdem besaß sein
Vater die Suzuki Geigenfabrik, die sie während und vor allem nach dem zweiten
WEltkrieg mit Mühe und Not über Wasser halten konnten. 1898 geboren erlebte Suzuki beide Weltkriege miterlebt und erst nach
dem
zweiten Weltkrieg, wo er selbst inzwischen ? Jahre alt war, seine
"Talenterziehung" wie er sie nennt gegründet. 

Suzuki lernte das Geigenspiel selbst als Auto-Didakt, als er in der Fabrik
anfing zu arbeiten und gerade mit der Schule fertig war. Wollte zunächst gar
kein Musiker werden, aber eine Klavierlehrerin, die ihn spielen gehört hatte auf
einer Kreuzschifffahrt, war fest davon überzeugt, dass er Musik studieren
sollte.(Fräulein Kodas Ansicht, S. 89)
Anfang 20 (mit 21) bekam er dann seinen ersten richtigen Geigenunterricht, bei der Schwester
von Fräulein Koda (S. 90) 
Mit 22 Jahren fuhr er dann zunächst unter der Vorgabe einer Welterreise, die der
Fürst Tokugawa ihm vorgeschlagen hat nach Deutschland um dort das Geigenstudium zu
beginnen. (S. 92)

(Suzuki lernte in Berlin Einstein kennen)

Suzuki verbrachte 8 Jahre, also von 22-30 in Berlin. 

"Und durch die Talenterziehung, ((die der Muttersprache zugrunde liegt)), die
sich auf Liebe gründet, wird Sinn für Wahrheit, Freude und Schönheit ein
natürlicher Teil des Charakters der Kinder werden." (Szusuki S. 75)

Ich brachte mit Suzuki immer junge Kinder in Verbindung, die gemeinsam mit einem
Lehrer Stücke erarbeiteten. 
Ähnlich wie bei der musikalischen Früherziehung ist es Suzuki auch ein Anliegen,
die Kinder schon im frühen Kindesalter mit der Musik in Berührung zu bringen.
Nicht umsonst sind die jüngsten Schüler 4 bis 5 Jahre alt. Und auch schon vorher
spricht er davon, dass die Kinder wie ein Samenkorn was in der Erde wächst und
gedeiht, ehe man es heraussprießen sieht, sich viel in der Entwicklung vollzieht
und die Einflüsse wie bei einer Pflanze auch schon im frühsten
Entstehungsstadium auf das Kind einwirken. (s. 17 Suzuki)

"Auch in größeren Gruppen finden ein- und zweipolige Interaktionen statt."
(S.30, die kunst zu unterrichten)

Suzuki Methode geht erst einmal davon aus das man Talent nicht hat, sondern
lernen kann. Beispiel Muttersprache (S. 13 shinishi Suzuki)

Dementsprechend muss man schon direkt nach der Geburt anfangen bzw. so früh wie
möglich und das Kind im Lernporzess begleiten. Wie wichtig dsa Umfeld dabei ist
kommt im folgenden Zitat von Shinichi Suzuki gut zum Ausdruck: "Einen Samen, der
in den Boden versenkt ist, können wir nicht sehen, doch Wasser, Temperatur,
Licht und Schatten wirken als täglicher Anreiz." (Shinichi Suzuki S. 17)
Wer ist nun dieses Umfeld? Sind es die Mitspieler, die eigenen Eltern oder
Geschwister, der Lehrer? Es spielt also das gesamte Umfeld zusammen, wozu alle
diese Begleiter eines Instrumentalschülers/Menschen zählen. 

- meine Annahme: je aktiver diese den Lernprozess mitbegleiten, desto
erfolgreicher lernt der Schüler. Wenn jemand aus einer Musikerfamilie kommt,
oder Geschwister hat, die Musizieren, dann bekommt das Kind dort schon sehr viel
mit. (Kapitel mit Baby voller Freude S.20)

"Jedes Menschen Persönlichkeit, das bedeutet: seine Fähigkeiten, seine Art zu
denken und zu fühlen, wird durch Umstände und Umgebung geschnitzt und
gemeißelt." (S. 20 shinichi Suzuki)

Rolle des Lehrers: "von größter Bedeutung im ersten Monat einen guten Lehrer zu
haben.2 (S. 21 Shinichi Suzuki)

Es ist fragwürdig und sicher auch ein kultureller Unterschied/Einfluss, dass Suzuki viele seiner Beobachtungen und Theorien von
Vögeln auf den Menschen überträgt. 

Wolfskinder
"Der springende Punkt ist eben nicht die Erblichkeit, sondern die Umgebung.
Kinder leben, sehen und fühlen, und ihrer Umgebung entsprechend entwickeln sie
ihre Fähigkeiten." (S.24 shinichi suzuki)

"Was in der Umgebung fehlt, wird nicht entwickelt." 
"Gerade die frühste Kidnheit ist kritisch."(s. 25)
"Erst von diesem Zeitpunkt an machen sich die psychologischen Einflüsse geltend,
und zwar von der Umwelt des Kindes ausgehend. Die Bedingungen dieser Umwelt
schaffen den Kern seiner Fähigkeiten." (S. 26, Suzuki )

Umgebung
Wenn Suzuki aber von der alles entscheidenden Umgebung spricht, wie muss diese
dann im besten Fall aussehen? 

Piaget sagt im Gegenteil, dass das Lernen von dem Schüler ausgeht und aus seinem
Innern kommt. In diesem Zusammenhang ist die Umgebung also nicht alles entscheidend.
--> Lessings Text nochmal dazu durchforsten

Im Bezug auf das gemeinsame Musizieren ist dann aber die äußere Umgebung das
Ensemble. Wie viel Einfluss hat dann der Ensemble-Baustein? Und wie groß ist der
Teil den es einnimmt als äußerer Einfluss auf einen Schüler? Fest steht, dass
auch ein Ensemble alleine nur ein Baustein von der Gesamtumgebung eines Schülers
ist. Er ist außerdem im besten Falle eine Ergänzung zum Instrumentalunterricht
und öffnet den musikalischen Horizont des Schülers, aber er ersetzt nicht die
Rolle der Familie, der Schule und des sonstigen Kontextes. 

Suzuki wollte (nach dem Krieg) vorallem kleinere Kinder und Anfänger mit seiner
damals neuen Methode unterrichten "(...)nicht um Genies auszubilden, sondern um
durch Geigespielen die Fähigkeiten der Kinder zu erweitern." (S. 43 Suzuki) So
ist im die ganzheitliche menschliche Ausbildung nämlich ein wichtiges Anliegen
und er der Meinung, dass alle seine Schüler vorallem auch zu "edlen Menschen"
herangezogen wurden, wenn sie durch seine Schule gingen. 

Auswendig spielen: "Meine Schüler müssen das Musikstück auswendig beherrschen
und dürfen sich nicht auf die Noten verlassen." (S. 46 suzuki)

"Fähigkeit durch Üben entwickeln" "Jedermann kann sich selbst ausbilden; er muß
lediglich die wirkungsvollsten Methoden kennen" (Suzuki S. 52)

"Wir müssen unsere Fähigkeiten einfach üben und weitererziehen, das bedeutet,
dieselbe Tätigkeit immer aufs Neue wiederholen, bis sie natürlich, einfach und
leicht vonstatten geht. Das ist das ganze Geheimnis." (Suzuki S. 57)
Nach unseren Erkenntnissen der heutigen Zeit kann man diese banale Wiederholen
allerdings sehr kritisch betrachten. Schließlich geht es auch um die Intention
hinter der Tätigkeit und ...

"(...) aber diese Tatkraft aufzubringen, ist die wahre Schwierigkeit." (Suzuki
S. 60) --> die Tatkraft des wirklich täglichen Wiederholens

"Leistung bedeutet, dass auch Tatkraft und Geduld genau wie die anderen
Fähigkeiten ausgebildet werden müssen." (S.60)

"In diesem Augenblick indes ist es nicht mehr eine Frage der Gewandheit allein,
sondern des Besitzes von Geist und Herz." (S. 59)

In Suzukis Persönlichkeit wird das unererlässliche Streben nach Verbesserung
seiner selbst als Person und in seinen musiaklischen Fähigekeiten ganz deutlich
sichtbar. 

(Evtl diesen Abschnitt ins Fazit)
Wir können daraus schließen, dass die musikalische Ausbildung weit über
Einzelunterrichtseinheit hinausgeht und es auf den gesamten Umgang mit der Musik
und dem Instrument im Umfeld eines Schülers ankommt. Deswegen müssen wir
unbedingt Angebote in unserem System schaffen, die über den Einzelunterricht
hinausgehen! 


Samenkorn
- Wie es sich verwurzelt und dass man ihn nicht einfach herausnehmen und
anschauen kann ob er gewachsen ist Suzuki S. 61

Geigenentdeckung von seinem Vater bzw. in der Fabrik dort S. 76


"Es liegt in unserer Macht, alle Kinder der Welt so zu erziehen, dass sie ein
wenig besser, ein wenig glücklicher werden. Darauf müssen wir hinwirken. Ich
erstrebe nicht mehr als Liebe und Glück für die Menschheit, und das, so glaube
ich, ersehnt jeder in seinem Innersten." (suzuki S. 103)

"Will ein Musiker ein guter Künstler werden, muss er zunächst ein besserer
Mensch werden." (S. 103)

"Kunst ist nicht irgendwo im luftleeren Raum zu finden, ein Kunstwerk ist der
Ausdruck der umfassenden Persönlichkeit, des Gefühls und der Fähigkeiten eines
Menschen." (s. 103)

"Lebenskraft als Erziehung./ Mer Erziehung als Unterricht. (S. 105/106)

"Das Wort "Erziehung enthält in seiner Bedeutung zwei Begriffe. Neben
"unterrichten" bedeutet es auch "Aufziehen", "bilden", das heißt formend
gestalten" (suzuki S. 106)
