\section{Suzuki}

\subsection{Gründer Shinichi Suzuki}
Shinichi Suzuki wurde 1898 in Japan geboren und studierte von seinem 23ten
Lebensjahr an, für acht Jahre Geige in Berlin, wo er später auch seine Ehefrau
kennenlernte. Sein Vater war Besitzer der Suzuki Geigenfabrik. Suzuki erlebte
beide Weltkriege mit und gründeetet nach dem zweiten Weltkrieg im Alter
von 48 Jahren das Talenterziehungs-Institut. 

Interessanterweise lernte Shinichi Suzuki das Geigenspiel selbst zunächst als Auto-Didakt, als er
in der Fabrik anfing zu arbeiten und gerade mit der Schule fertig war. Zunächst
wollte er gar kein Musiker werden, aber eine Klavierlehrerin, die ihn auf einer
Kreuzschifffahrt spielen gehört hatte, war fest davon überzeugt, dass er Musik
studieren sollte. \autocite[89]{suzuki:erziehung_ist_liebe} Diese Parallele gibt
es zu Abreu, der auch besonders durch seine Klavierlehrerin geprägt war. Im Alter von 21
Jahren bekam er schließlich seinen ersten richtigen Geigenunterricht, bei der
Schwester der besagten Klavierlehrerin. Nur zwei Jahre später fuhr er dann
zunächst unter der Vorgabe einer Weltreise, die der Fürst Tokugawa ihm
vorgeschlagen hatte, nach Deutschland um dort das Geigenstudium aufzunhemen.
\autocite[90ff]{suzuki:erziehung_ist_liebe}

\subsection{Suzukis Methode und Philosophie}

Zunächst einmal muss man wissen, dass Suzuki mit der Gründung seines
Talenterziehungs-Insituts, den Kindern über die Musik hinaus zu einer guten
Entwicklung verhlefen möchte und sein Ziel ist \enquote{alle Kinder der Welt so zu erziehen, dass sie
ein wenig besser, ein wenig glücklicher werden. Darauf müssen wir hinwirken. Ich
erstrebe nicht mehr als Liebe und Glück für die Menschheit, und das, so glaube
ich, ersehnt jeder in seinem Innersten.}
\autocite[103]{suzuki:erziehung_ist_liebe}
Er nutzt das Musizieren vielmehr als ein Mittel, die Kinder auf den richtigen
Weg zu bringen. Auch dies ist ein ähnlicher Gedanke, aus dem heraus alle zuvor
beleuchteten Systeme entstanden sind. So sagt Suzuki nämlich auch: \enquote{Will ein Musiker ein guter
Künstler werden, muss er zunächst ein besserer Mensch
werden.}\autocite[103]{suzuki:erziehung_ist_liebe} Beim Erlernen eines
Instrumentes muss der Schüler nach seinem Modell täglich die Tatkraft
aufbringen, das neue Stück zu üben und dies Tatkraft aufzubringen, ist die wahre
Schwierigkeit. %(60) 
\enquote{Wir müssen unsere Fähigkeiten einfach üben und
weitererziehen, das bedeutet, dieselbe Tätigkeit immer aufs Neue wiederholen,
bis sie natürlich, einfach und leicht vonstatten geht. Das ist das ganze
Geheimnis.}\autocite[57]{suzuki:erziehung_ist_liebe} Nach unseren Erkenntnissen der heutigen Zeit kann man
dieses banale Wiederholen allerdings sehr kritisch betrachten. Schließlich geht
es auch um die Intention hinter der Tätigkeit und die sinnvoll verbrachte Zeit
am Instrument. Es gehört aber bei dem
Erlernen eines Instrumentes neben der Tatkraft, die Suzuki nennt, auch dazu, die
Fähigkeit der Geduld mit einzuüben. \enquote{Leistung bedeutet, dass auch Tatkraft und
Geduld genau wie die anderen Fähigkeiten ausgebildet werden müssen.}
\autocite[60]{suzuki:erziehung_ist_liebe}
\enquote{Und durch die Talenterziehung, die sich auf Liebe gründet, wird Sinn für
Wahrheit, Freude und Schönheit ein natürlicher Teil des Charakters der Kinder
werden.} \autocite[75]{suzuki:erziehung_ist_liebe}

Ähnlich wie bei der musikalischen Früherziehung ist es Suzuki ein großes
Anliegen, die Kinder schon im frühen Kindesalter mit der Musik in Berührung zu
bringen. Nicht umsonst sind die jüngsten Schüler in der Talenterziehung drei
Jahre alt. Auch schon vorher spricht er davon, dass sich bei den Kindern viel in
der Entwicklung vollzieht und die Einflüsse auch schon im frühsten
Entstehungsstadium auf das Kind einwirken. Er nennt das Bild eines Samenkorns,
was in der Erde wächst und gedeiht und bei welchem schon sehr viel passiert, ehe
man es heraussprießen sieht. \enquote{Einen Samen, der in den Boden versenkt ist, können
wir nicht sehen, doch Wasser, Temperatur, Licht und Schatten wirken als
täglicher Anreiz.}\autocite[17]{suzuki:erziehung_ist_liebe} Es spielt also das
gesamte Umfeld eine Rolle, wozu alle Begleiter des Schülers zählen. Außerdem
trägt der Lehrer aus den frühsten Jahren eine sehr große Verantwortung. Es ist
\enquote{von größter Bedeutung im ersten Monat einen guten Lehrer zu haben.}
\autocite[21]{suzuki:erziehung_ist_liebe} Nach diesem Bild muss man schon direkt
nach der Geburt beginnen, das Kind mit der Musik in Berührung zu bringen bzw. so
früh wie möglich und das Kind im Lernporzess begleiten, da gerade die frühste
Kindheit kritisch ist. \autocite[25]{suzuki:erziehung_ist_liebe} So gehört zur
Suzuki-Talenterziehung dazu, dass die Kinder die Musikstücke, die sie spielen
und üben sollen, auch außerhalb des Unterrichts viel anhören, oder sie im
Hintergrund zu Hause abgespielt werden, sodass die Stücke in ihrem Alltag
integriert sind. Der Talenterziehung liegt das Prinzip der Muttersprache
zugrunde und Suzuki versucht die Musik so natürlich in das Leben des Kindes zu
integrieren, dass das Kind die Musik wie eine Muttersprache erlernen kann.  So
erhalten nicht die Schüler*innen, sondern die Eltern die ersten Geigenstunden und die
Kinder sollen im Idealfall zu Hause mit dem Instrument spielerisch
in den Kontakt treten. \autocite[75]{suzuki:erziehung_ist_liebe} Es hängt also
viel von der Umgebung des Schülers ab, also von seinen Eltern, ob es Geschwister
hat, die Musizieren, den Mitschülern und allen anderen Begleitern. Suzuki
formuliert ganz drastisch: \enquote{Was in der Umgebung fehlt, wird nicht entwickelt.}
\autocite[25]{suzuki:erziehung_ist_liebe}

Wenn Suzuki aber von der alles entscheidenden Umgebung spricht, wie muss diese
dann im besten Fall aussehen? Kann sich ein Schüler nicht trotzdem weiter
entwickeln, wenn er sich von seinen eigenen Fähigkeiten leiten und führen lässt?
Suzuki vergleicht es mit den berühmten Wolfskindern aus Indien, dass Kinder nur
ihrer Umgebung entsprechend die Fähigkeiten entwickeln können und unter keinen
Umständen darüber hinaus. \autocite[24]{suzuki:erziehung_ist_liebe} Er vertritt
den Standpunkt, dass Talent nicht vererbar ist, sondern alles eine Frage der
Entwicklung der Fähigkeiten ist und geht zunächst einmal davon aus, dass in
jedem Kind das gleiche Potential steckt. \enquote{Der springende Punkt ist nicht die
Erblichkeit, sondern die Umgebung.} \autocite[24]{suzuki:erziehung_ist_liebe}
\enquote{Jedes Menschen Persönlichkeit, das bedeutet: seine Fähigkeiten, seine Art zu
denken und zu fühlen, wird durch Umstände und Umgebung geschnitzt und
gemeißelt.} \autocite[20]{suzuki:erziehung_ist_liebe}

Suzuki wollte nach dem Krieg vorallem kleinere Kinder und Anfänger*innen mit seiner
damals neuen Methode unterrichten \enquote{(...)nicht um Genies auszubilden, sondern um
durch Geigespielen die Fähigkeiten der Kinder zu erweitern.}
\autocite[43]{suzuki:erziehung_ist_liebe} So
ist ihm die ganzheitliche menschliche Ausbildung ein wichtiges Anliegen
und er der Meinung, dass alle seine Schüler vorallem auch zu \emph{edlen Menschen}
herangezogen wurden, wenn sie durch seine Schule gingen. 

%Auswendig spielen: "Meine Schüler müssen das Musikstück auswendig beherrschen
%und dürfen sich nicht auf die Noten verlassen." (S. 46 suzuki)

%"Fähigkeit durch Üben entwickeln" "Jedermann kann sich selbst ausbilden; er muß
%lediglich die wirkungsvollsten Methoden kennen" (Suzuki S. 52)

%In Suzukis Persönlichkeit wird das unererlässliche Streben nach Verbesserung
%seiner selbst als Person und in seinen musiaklischen Fähigekeiten ganz deutlich
%sichtbar.


%Geigenentdeckung von seinem Vater bzw. in der Fabrik dort S. 76

%"Lebenskraft als Erziehung./ Mehr Erziehung als Unterricht. (S. 105/106)

%"Das Wort "Erziehung enthält in seiner Bedeutung zwei Begriffe. Neben
%"unterrichten" bedeutet es auch "Aufziehen", "bilden", das heißt formend
%gestalten" (suzuki S. 106)



\subsection{Das Lernen in einer musikalischen Praxisgemeinschaft}
Das gemeinschaftliche Musizieren steht in Suzukis Talenterziehung im direkten
Zusammenhang mit dem Erlenen des Instruments. Da sogar auch die Eltern mit
eingebunden werden, kommt im Vergleich zu anderen Musiksystemen auch noch dieser
vorgesehene Faktor hinzu. Wie Suzuki in seiner Biografie immer wieder deutlich
zum Ausdruck bringt, spielt die Umgebung im Entwicklungsprozess des Schülers
eine sehr wichtige Rolle. Nach Suzuki könnte man fast schon sagen, das die
Umgebung auch einen Ensemble-Baustein darstellt, mit ihren musikalischen
Auswirkungen auf das Kind. Dieser Baustein ist außerdem im besten Falle eine
Ergänzung zum Instrumentalunterricht und öffnet den musikalischen Horizont des
Schülers. Außerdem lernen die Kinder mindestens einmal in der Woche im
Gruppenunterricht von ihrer Lehrkraft und zusammen mit ihren Mitschüler*innen
neue Stücke. Betrachtet man diesen Gruppenprozess allerdings einmal etwas
genauer, so merkt man, dass die Rolle der Lehrkraft eine ganz klar übergeordnete
ist und kein Raum für jegliche andere Gruppendynamiken gelassen wird. Nach
Suzukis Beschreibung finden \enquote{Auch in größeren Gruppen (finden) ein- und
zweipolige Interaktionen statt.} \autocite[30]{suzuki:erziehung_ist_liebe}
Allerdings bietet seine Art wie er den Gruppenunterricht gestaltet sehr wenig
Raum für Individualität oder Austausch der Schüler untereinander. Wie im Kapitel
des Gemeinsamen Musizierens beschrieben, ist aber gerade an Gruppenprozessen das
Interessante, dass wir über die zweipolige Interaktion hinaus agieren können. Da
im Einzelunterricht die Lehrkraft und der/die Schüler*in auch schon ein- oder
zweipolig interagieren, fehlt hier in der Suzuki Talenterziehung, die
Interaktion, die darüber hinausgeht und gerade die kostbare Ergänzung zum
Einzelunterricht darstellt. Leider wird der/die Schüler*in als Individuum
außerdem komplett in den Hintergrund gestellt und wenig dazu angeregt sich frei
zu entfalten oder nach ihrem/seinem eigenen Weg mit dem Instrument zu forschen.
Kulturell betrachtet und auch wenn man die Gründung der Talenterziehung in den
historischen Kontext einordnet, kann man sich durchaus erklären, wieso kurz nach
dem zweiten Weltkrieg das Bedürfnis nach einem geordneten und von Oben
bestimmten Gruppenprozess ein sehr erwünschtes Modell war. Nicht umsonst hat
Suzuki mit seiner Talenterziehung so große Wellen geschlagen, die bald nach
Amerika überschwappten und heutzutage auch in ganz Europa gängig ist.



\subsection{Pro und Contra}
Die Lehrkraft ist allen Beschreibungen zu Folge immer in ihrer dominanten Rolle
und gibt bei Suzuki im wahrsten Sinne des Wortes den Ton an. So sind die Schüler
der Talenterziehung zwar in einer Gruppe, aber durch die fehlende Interaktion
untereinander, haben sie sehr wenig Gelegenheit bewusst voneinander zu lernen
und in den Austausch zu kommen. Das differenzielle Lernen findet trotzdem statt,
da sie mitbekommen, wer die Stücke schnell zu beherrschen lernt und wer manchmal
etwas länger benötigt oder auch wie die Mitspieler jeweils ihr Instrument halten
und ob es organisch aussieht oder nicht. Was wiederum für die Schüler
untereinander interessant ist, ist wie auch bei El Sistema, dass sie
unterschiedlich fortgeschritten sind mit dem Instrument und somit die weniger
Fortgeschrittenen Schüler von den Anderen etwas abschauen und lernen können.
Aber funktioniert dieser Weg auch umgekehrt? Ein fortgeschrittener Schüler kann
vermutlich mehr an seinem Charakter arbeiten und sich in Tugenden wie Geduld und
stetige Selbstverbesserung üben; außerdem kann er sehen und analysieren, womit
die Mitspieler zu kämpfen haben und woran das liegen mag, darüber hinaus kann
der/die Fortgeschrittene rein instrumental weniger von dem/der Anfänger*in
lernen. Orientiert sich diese*r wiederum an der Lehrkraft, kann e/sie sich doch
noch klanglich oder technisch etwas abschauen. Der Gedanke Suzukis auf die
Einflüsse der Umwelt, ist sehr einleuchtend, problematisch sehe ich jedoch die
Abhängigkeit, die dabei entsteht. Laut Suzuki kann das Kind nur soviel Lernen,
wie es von ihrem Umfeld mitbekommt. Allerdings glaube ich dass Schüler*innen
durchaus das musikalische Potential mit sich bringen auch im späteren Alter und
einem weniger musikaffinen Umfeld die Möglichkeiten haben sich musikalisch zu
entfalten. Auch problematisch sehe ich den Gedanken der quantitativen Zeit am
Instrument, in dem es nur um das fleißige Üben geht und Suzuki hierbei das Thema
\emph{wie} man richtig übt außer acht lässt. Außerdem ist auch die Motivation
ein wichtiger Faktor, der uns mit unserem Instrument deutlich weiterbringen kann
und qualitativ wertvolle Erkenntnisse hervorrufen kann. Natürlich ist ein
gewisses Maß an Kontinuität für das Erlernen notwendig um auch Fortschritte zu
erzielen, die einen motivieren. Aber aus Prinzip seine Zeit am Instrument
abzusitzen trägt dazu sicher nicht dabei. Die Mentalität spielt bei dieser
Herangehensweise sicher eine bedeutende Rolle. Aber trotz allem ist diese
Mentalität und auch die Erziehung in Europa eine Andere. Gerade bei Künstlern
und Musikern ist die Erziehung sehr auf die Ausgestaltung der Individualität
ausgerichtet, mit dem Ziel Musikerpersönlichkeiten auszubilden. Trotz allem sagt
Shinichi Suzuki aber: \enquote{Die Kunst ist nicht irgendwo im luftleeren Raum
zu finden, ein Kunstwerk ist der Ausdruck der umfassenden Persönlichkeit, des
Gefühls und der Fähigkeiten eines Menschen.}
\autocite[103]{suzuki:erziehung_ist_liebe} Die Gefühle und auch Fähigkeiten der
Menschen sind aber so individuell, dass er der Förderung dieser individuellen
Entwicklung meiner Meinung nach nicht gut genug nachkommt, obwohl er gerade eben
diese als die hohe Kunst ansieht. Suzuki thematisiert die Mentalität und betont
immer wieder, wie wichtig es ist, die Menschen zu guten und sich immer weiter
verbessernden Menschen zu erziehen. Außerdem appelliert er an die Verantwortung
der Gesellschaft gegenüber den Kindern, dass jedem geborenen Kind Versorgung und
Erziehung gesichert ist. \autocite[130]{suzuki:erziehung_ist_liebe} 
Interessanterweise wurde die Suzuki-Methode in Deutschland vom Verband deutscher
Musikschulen abgelehnt. \autocite[49]{ernst:die_zukunftsfaehige_musikschule}
