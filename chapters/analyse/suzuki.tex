\section{Suzuki}

\subsection{Suzuki als Person} Shinichi Suzuki als Person/Lebenslauf
Shinichi Suzuki wurde 1898 in Japan geboren und studierte
von seinem 23ten Lebensjahr an, für acht Jahre Geige
in Berlin, wo er später auch seine Ehefrau kennenlernte. Sein
Vater war Besitzer der Suzuki Geigenfabrik. Suzuki erlebte beide Weltkriege mit und erst nach
dem zweiten Weltkrieg, gründete er im Alter von 48 Jahren das
Talenterziehungs-Institut. 

Shinichi Suzuki lernte das Geigenspiel selbst zunächst als Auto-Didakt, als er in der Fabrik
anfing zu arbeiten und gerade mit der Schule fertig war. Zunächst wollte er gar
kein Musiker werden, aber eine Klavierlehrerin, die ihn auf einer
Kreuzschifffahrt spielen gehört hatte, war fest davon überzeugt, dass er Musik studieren
sollte. \autocite[89]{suzuki:erziehung_ist_liebe}
Im Alter von 21 Jahren bekam er schließlich seinen ersten richtigen Geigenunterricht, bei der Schwester
der besagten Klavierlehrerin. Nur zwei Jahre später fuhr er dann zunächst unter der Vorgabe einer Weltreise, die der
Fürst Tokugawa ihm vorgeschlagen hatte, nach Deutschland um dort das Geigenstudium aufzunhemen. \autocite[90ff]{suzuki:erziehung_ist_liebe}

\subsection{Suzukis Methode und Philosophie}

Zunächst einmal muss man wissen, dass Suzuki mit der Gründung seines
Talenterziehungs-Insituts, den Kindern über die Musik hinaus zu einer guten
Entwicklung verhlefen möchte und "alle Kinder der Welt so zu
erziehen, dass sie ein wenig besser, ein wenig glücklicher werden. Darauf müssen wir hinwirken. Ich
erstrebe nicht mehr als Liebe und Glück für die Menschheit, und das, so glaube
ich, ersehnt jeder in seinem Innersten."
\autocite[103]{suzuki:erziehung_ist_liebe}
Er nutzt das Musizieren vielmehr als ein Mittel, die Kinder auf den richtigen
Weg zu bringen. So sagt Suzuki nämlich auch: "Will ein Musiker ein guter Künstler werden, muss er zunächst ein besserer
Mensch werden."v\autocite[103]{suzuki:erziehung_ist_liebe}
Beim Erlernen eines Instrumentes muss der Schüler nach seinem Modell täglich die Tatkraft
aufbringen, das neue Stück zu üben und dies Tatkraft aufzubringen, ist die wahre
Schwierigkeit. (60)
"Wir müssen unsere Fähigkeiten einfach üben und weitererziehen, das bedeutet,
dieselbe Tätigkeit immer aufs Neue wiederholen, bis sie natürlich, einfach und
leicht vonstatten geht. Das ist das ganze Geheimnis." (Suzuki S. 57)
Nach unseren Erkenntnissen der heutigen Zeit kann man diese banale Wiederholen
allerdings sehr kritisch betrachten. Schließlich geht es auch um die Intention
hinter der Tätigkeit und ...
Es gehört aber bei dem Erlernen eines Instrumentes neben der Tatkraft,
die Suzuki nennt, auch dazu, die Fähigkeiten der Geduld mit einzuüben. "Leistung bedeutet, dass auch Tatkraft und Geduld genau wie die anderen
Fähigkeiten ausgebildet werden müssen."
\autocite[60]{suzuki:erziehung_ist_liebe}
"Und durch die Talenterziehung, die
sich auf Liebe gründet, wird Sinn für Wahrheit, Freude und Schönheit ein
natürlicher Teil des Charakters der Kinder werden." \autocite[75]{suzuki:erziehung_ist_liebe}

Ähnlich wie bei der musikalischen Früherziehung ist es Suzuki ein großes Anliegen,
die Kinder schon im frühen Kindesalter mit der Musik in Berührung zu bringen.
Nicht umsonst sind die jüngsten Schüler in der Talenterziehung drei Jahre alt. Auch schon vorher
spricht er davon, dass sich bei den Kindern viel in der Entwicklung vollzieht
und die Einflüsse auch schon im frühsten Entstehungsstadium auf das Kind
einwirken. Er nennt das Bild eines Samenkorns, was in der Erde wächst und
gedeiht und bei welchem schon sehr viel passiert, ehe man es heraussprießen sieht. "Einen Samen, der
in den Boden versenkt ist, können wir nicht sehen, doch Wasser, Temperatur,
Licht und Schatten wirken als täglicher Anreiz."\autocite[17]{suzuki:erziehung_ist_liebe}
Es spielt also das gesamte Umfeld eine Rolle, wozu alle Begleiter des Schülers
zählen. Außerdem trägt der Lehrer aus den frühstens Jahren eine sehr
große Verantwortung. Es ist "von größter Bedeutung im ersten Monat einen guten Lehrer zu
haben." \autocite[21]{suzuki:erziehung_ist_liebe}
 Nach diesem Bild muss man schon direkt nach der Geburt beginnen, das Kind mit
der Musik in Berührung bringen bzw. so früh wie
möglich und das Kind im Lernporzess begleiten, da gerade die frühste Kindheit
kritisch ist. \autocite[25]{suzuki:erziehung_ist_liebe}
So gehört zur Suzuki-Talenterziehung dazu, dass die Kinder die Musikstücke,
die sie spielen und üben sollen, auch außerh]alb des Unterrichts viel anhören, oder sie im Hintergrund zu Hause
abgespielt werden, sodass die Stücke in ihrem Alltag integriert sind.
Der Talenterziehung liegt das Prinzip der Muttersprache zugrunde und Suzuki versucht
die Musik so natürlich in das Leben des Kindes zu integrieren, dass das Kind die
Musik wie eine Muttersprache erlernen kann.  So erhalten nicht die Schüler,
sondern die Eltern die ersten
Geigenstunden und die Kinder sollen im Idealfall die Schüler zu Hause mit dem
Instrument spielerisch in den Kontakt treten. \autocite[75]{suzuki:erziehung_ist_liebe}
Es hängt also viel von der Umgebung des Schülers ab, also von seinen Eltern, ob
es Geschwister hat, die Musizieren, den Mitschülern und allen anderen Begleitern.
Suzuki formuliert ganz drastisch: "Was in der Umgebung fehlt, wird nicht entwickelt." \autocite[25]{suzuki:erziehung_ist_liebe}

Wenn Suzuki aber von der alles entscheidenden Umgebung spricht, wie muss diese
dann im besten Fall aussehen? Kann sich ein Schüler nicht trotzdem weiter
entwickeln, wenn er sich von seinen eigenen Fähigkeiten leiten und führen lässt?
Suzuki vergleicht es mit den berühmten Wolfskindern aus Indien, dass Kinder nur
ihrer Umgebung entsprechend die Fähigkeiten entwickeln können und unter keinen
Umständen darüber hinaus. \autocite[24]{suzuki:erziehung_ist_liebe}
Er vertritt den Standpunkt, dass Talent nicht vererbar ist, sondern alles eine Frage
der Entwicklung der Fähigkeiten ist und geht zunächst einmal davon aus, dass
in jedem Kind das gleiche Potential steckt. "Der springende Punkt ist nicht die
Erblichkeit, sondern die Umgebung." \autocite[24]{suzuki:erziehung_ist_liebe}
"Jedes Menschen Persönlichkeit, das bedeutet: seine Fähigkeiten, seine Art zu
denken und zu fühlen, wird durch Umstände und Umgebung geschnitzt und
gemeißelt." (S. 20 shinichi Suzuki)

Suzuki wollte (nach dem Krieg) vorallem kleinere Kinder und Anfänger mit seiner
damals neuen Methode unterrichten "(...)nicht um Genies auszubilden, sondern um
durch Geigespielen die Fähigkeiten der Kinder zu erweitern." (S. 43 Suzuki) So
ist ihm die ganzheitliche menschliche Ausbildung nämlich ein wichtiges Anliegen
und er der Meinung, dass alle seine Schüler vorallem auch zu "edlen Menschen"
herangezogen wurden, wenn sie durch seine Schule gingen. 

Auswendig spielen: "Meine Schüler müssen das Musikstück auswendig beherrschen
und dürfen sich nicht auf die Noten verlassen." (S. 46 suzuki)

"Fähigkeit durch Üben entwickeln" "Jedermann kann sich selbst ausbilden; er muß
lediglich die wirkungsvollsten Methoden kennen" (Suzuki S. 52)

In Suzukis Persönlichkeit wird das unererlässliche Streben nach Verbesserung
seiner selbst als Person und in seinen musiaklischen Fähigekeiten ganz deutlich
sichtbar.


Geigenentdeckung von seinem Vater bzw. in der Fabrik dort S. 76

"Lebenskraft als Erziehung./ Mehr Erziehung als Unterricht. (S. 105/106)

"Das Wort "Erziehung enthält in seiner Bedeutung zwei Begriffe. Neben
"unterrichten" bedeutet es auch "Aufziehen", "bilden", das heißt formend
gestalten" (suzuki S. 106)



\subsection{Gemeinsames Musizieren}
Das gemeinschaftliche Musizieren steht in Suzukis Talenterziehung im direkten
Zusammenhang mit dem Erlenen des Instruments. Da sogar auch die Eltern mit
eingebunden werden, kommt im Vergleich zu anderen Musiksystemen auch noch dieser
vorgesehene Faktor hinzu. Wie Shinichi Suzuki in seiner Biografie immer wieder
deutlich zum Ausdruck bringt, spielt die Umgebung im Entwicklungsprozess des
Schülers eine sehr wichtige Rolle. Außerdem lernen die Kinder mindestens einmal in der
Woche im Gruppenunterricht von ihrem Lehrer und zusammen mit ihren Mitschülern
neue Stücke. Betrachtet man diesen Gruppenprozess allerdings einmal etwas
genauer, so merkt man, dass die Rolle des Lehrers eine ganz klar übergeordnete
ist und kein Raum für jegliche andere Gruppendynamiken gelassen wird. Nach
Suzukis Beschreibung finden "Auch in größeren Gruppen (finden) ein- und zweipolige
Interaktionen statt." \autocite[30]{suzuki:erziehung_ist_liebe}
Allerdings bietet seine Art wie er den Gruppenunterricht gestaltet sehr wenig
Raum für Individualität oder Austausch der Schüler untereinander. Der Lehrer ist
allen Beschreibungen zu Folge immer in seiner dominanten Rolle und gibt bei
Suzuki im wahrsten Sinne des Wortes den Ton an.
So sind
die Schüler der Talenterziehung zwar in einer Gruppe, aber durch die fehlende
Interaktion untereinander haben sie sehr wenig Gelegenheit bewusst voneinander
zu lernen und in den Austausch zu kommen. Das differenzielle Lernen findet
trotzdem statt, da sie mitbekommen, werde die Stücke schnell zu beherrschen
lernt und wer manchmal etwas länger benötigt oder auch wie die Mitspieler
jeweils ihr Instrument halten und ob es organisch aussieht oder nicht. Was
wiederum für die Schüler untereinander interessant ist, ist, dass sie
unterschiedlich fortgeschritten sind mit dem Instrument und somit die weniger
Fortgeschrittenen Schüler von den Anderen etwas abschauen und lernen können.
Aber funktioniert dieser Weg auch umgekehrt? Ein fortgeschrittener Schüler kann
vermutlich mehr an seinem Charakter arbeiten und sich in Tugenden wie Geduld und
stetige Selbstverbesserung üben; außerdem kann er sehen und analysieren, womit die Mitspieler zu
kämpfen haben und woran das liegen mag, aber darüber hinaus kann der
Fortgeschrittene rein instrumental
weniger von dem Anfänger lernen. Orientiert sich dieser wiederum am Lehrer, kann
er sich doch noch klanglich oder technisch etwas abschauen.
Leider wird der Schüler als Individuum aber komplett in den Hintergrund gestellt
und wenig dazu angeregt sich frei zu entfalten oder nach seinem eigenen Weg mit
dem Instrument zu forschen. Kulturell betrachtet und auch wenn man die Gruündung
der Talenterziehung in den historischen Kontext einordnet, kann man sich
durchaus erklären, wieso kurz nach dem zweiten Weltkrieg das Bedürfnis an einem
geordneten und von Oben bestimmten Gruppenprozess ein sehr erwünschtes Modell
war. Nicht umsonst hat Suzuki mit seiner Talenterziehung so große Wellen
geschlagen, die auch bald nach Amerika überschwappten und heutzutage auch in
Europa gängig ist. Trotz allem ist die Mentalität und auch die Erziehung in
Europa eine Andere. Gerade bei Künstlern und Musikern ist die Erziehung sehr auf die Ausgestaltung der
Individualität ausgerichtet, mit dem Ziel Musikerpersönlichkeiten
auszubilden. Trotz allem sagt Shinichi Suzuki aber: "Die Kunst ist nicht irgendwo im luftleeren Raum zu finden, ein
Kunstwerk ist der
Ausdruck der umfassenden Persönlichkeit, des Gefühls und der Fähigkeiten eines
Menschen." \autocite[103]{suzuki:erziehung_ist_liebe} Die Gefühle und auch
Fähigkeiten der Menschen sind aber so individuell, dass er der Förderung dieser
individuellen Entwicklung meiner Meinung nach nicht gut genug nachkommt, obwohl
er gerade eben diese als die hohe Kunst ansieht.
Suzuki thematisiert die Mentalität nichts desto trotz auch und betont immer
wieder, wie wichtig es ist, die Menschen zu guten und sich immer weiter
verbessernden Menschen zu erziehen. Außerdem appelliert er an die Verantwortung
der Gesellschaft gegenüber, den Kindern, dass jedem geborenen Kind Versorgung
und Erziehung gesichert ist. \autocite[130]{suzuki:erziehung_ist_liebe}
Und für ihn ist das entschiedende Alter wo die Kinder am meisten in ihrer
Persönlichkeit und ihrem Charakter geformt werden
in den ersten zehn Lebensjahren. 


Im Bezug auf das gemeinsame Musizieren ist dann aber die äußere Umgebung das
Ensemble. Wie viel Einfluss hat dann der Ensemble-Baustein? Und wie groß ist der
Teil den es einnimmt als äußerer Einfluss auf einen Schüler? Fest steht, dass
auch ein Ensemble alleine nur ein Baustein von der Gesamtumgebung eines Schülers
ist. Er ist außerdem im besten Falle eine Ergänzung zum Instrumentalunterricht
und öffnet den musikalischen Horizont des Schülers, aber er ersetzt nicht die
Rolle der Familie, der Schule und des sonstigen Kontextes. 






(Evtl diesen Abschnitt ins Fazit)
Wir können daraus schließen, dass die musikalische Ausbildung weit über
Einzelunterrichtseinheit hinausgeht und es auf den gesamten Umgang mit der Musik
und dem Instrument im Umfeld eines Schülers ankommt. Deswegen müssen wir
unbedingt Angebote in unserem System schaffen, die über den Einzelunterricht
hinausgehen! 




"Die Entfaltung eines Talents verläuft gradlinig, darauf kann man sich
verlassen." S. 120 \autocite[120]{suzuki:erziehung_ist_liebe} steht aber im Gegensatz zu der Lerntheorie, wo man
"chaotisch" lernt.

Die Suzuki-Talenterziehung hat weltweit sehr hohe Wellen geschlagen, sodass 1969
bereits mehr als 200.000 Kinder an den Kursen teilgenommen haben. S. 122
