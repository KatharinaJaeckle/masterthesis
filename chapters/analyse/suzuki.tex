\section{Suzuki}

- Shinichi Suzuki als Person?

"Auch in größeren Gruppen finden ein- und zweipolige Interaktionen statt."
(S.30, die kunst zu unterrichten)

Suzuki Methode geht erst einmal davon aus das man Talent nicht hat, sondern
lernen kann. Beispiel Muttersprache (S. 13 shinishi Suzuki)

Dementsprechend muss man schon direkt nach der Geburt anfangen bzw. so früh wie
möglich und das Kind im Lernporzess begleiten. Wie wichtig dsa Umfeld dabei ist
kommt im folgenden Zitat von Shinichi Suzuki gut zum Ausdruck: "Einen Samen, der
in den Boden versenkt ist, können wir nicht sehen, doch Wasser, Temperatur,
Licht und Schatten wirken als täglicher Anreiz." (Shinichi Suzuki S. 17)
Wer ist nun dieses Umfeld? Sind es die Mitspieler, die eigenen Eltern oder
Geschwister, der Lehrer? Es spielt also das gesamte Umfeld zusammen, wozu alle
diese Begleiter eines Instrumentalschülers/Menschen zählen. 

- meine Annahme: je aktiver diese den Lernprozess mitbegleiten, desto
erfolgreicher lernt der Schüler. Wenn jemand aus einer Musikerfamilie kommt,
oder Geschwister hat, die Musizieren, dann bekommt das Kind dort schon sehr viel
mit. (Kapitel mit Baby voller Freude S.20)

"Jedes Menschen Persönlichkeit, das bedeutet: seine Fähigkeiten, seine Art zu
denken und zu fühlen, wird durch Umstände und Umgebung geschnitzt und
gemeißelt." (S. 20 shinichi Suzuki)

Rolle des Lehrers: "von größter Bedeutung im ersten Monat einen guten Lehrer zu
haben.2 (S. 21 Shinichi Suzuki)

Es ist fragwürdig und sicher auch ein kultureller Unterschied/Einfluss, dass Suzuki viele seiner Beobachtungen und Theorien von
Vögeln auf den Menschen überträgt. 

Wolfskinder
"Der springende Punkt ist eben nicht die Erblichkeit, sondern die Umgebung.
Kinder leben, sehen und fühlen, und ihrer Umgebung entsprechend entwickeln sie
ihre Fähigkeiten." (S.24 shinichi suzuki)

"Was in der Umgebung fehlt, wird nicht entwickelt." 
"Gerade die frühste Kidnheit ist kritisch."(s. 25)
"Erst von diesem Zeitpunkt an machen sich die psychologischen Einflüsse geltend,
und zwar von der Umwelt des Kindes ausgehend. Die Bedingungen dieser Umwelt
schaffen den Kern seiner Fähigkeiten." (S. 26, Suzuki )

Umgebung
Wenn Suzuki aber von der alles entscheidenden Umgebung spricht, wie muss diese
dann im besten Fall aussehen? 

Piaget sagt im Gegenteil, dass das Lernen von dem Schüler ausgeht und aus seinem
Innern kommt. In diesem Zusammenhang ist die Umgebung also nicht alles entscheidend.
--> Lessings Text nochmal dazu durchforsten

Im Bezug auf das gemeinsame Musizieren ist dann aber die äußere Umgebung das
Ensemble. Wie viel Einfluss hat dann der Ensemble-Baustein? Und wie groß ist der
Teil den es einnimmt als äußerer Einfluss auf einen Schüler? Fest steht, dass
auch ein Ensemble alleine nur ein Baustein von der Gesamtumgebung eines Schülers
ist. Er ist außerdem im besten Falle eine Ergänzung zum Instrumentalunterricht
und öffnet den musikalischen Horizont des Schülers, aber er ersetzt nicht die
Rolle der Familie, der Schule und des sonstigen Kontextes. 

Suzuki wollte (nach dem Krieg) vorallem kleinere Kinder und Anfänger mit seiner
damals neuen Methode unterrichten "(...)nicht um Genies auszubilden, sondern um
durch Geigespielen die Fähigkeiten der Kinder zu erweitern." (S. 43 Suzuki) So
ist im die ganzheitliche menschliche Ausbildung nämlich ein wichtiges Anliegen
und er der Meinung, dass alle seine Schüler vorallem auch zu "edlen Menschen"
herangezogen wurden, wenn sie durch seine Schule gingen. 

Auswendig spielen: "Meine Schüler müssen das Musikstück auswendig beherrschen
und dürfen sich nicht auf die Noten verlassen." (S. 46 suzuki)

"Fähigkeit durch Üben entwickeln" "Jedermann kann sich selbst ausbilden; er muß
lediglich die wirkungsvollsten Methoden kennen" (Suzuki S. 52)

(Evtl diesen Abschnitt ins Fazit)
Wir können daraus schließen, dass die musikalische Ausbildung weit über
Einzelunterrichtseinheit hinausgeht und es auf den gesamten Umgang mit der Musik
und dem Instrument im Umfeld eines Schülers ankommt. Deswegen müssen wir
unbedingt Angebote in unserem System schaffen, die über den Einzelunterricht
hinausgehen! 
