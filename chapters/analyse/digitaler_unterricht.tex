\section{Gemeinsames Musizieren im Online-Instrumentalunterricht}

Der digitale Instrumentalunterricht ist ein noch sehr neues Format, welches
durch die Corona-Pandemie alle Instrumentallehrer*innen in Deutschland vor neue
Herausforderungen gestellt hat. Durch die Lockdowns waren die Lehrkräfte
gezwungen ihren Schüler*innen das Instrument online weiter beizubringen, während
sämtliche Ensemble-, Band- , Orchester oder Musikvereinsproben erst einmal auf
Eis gelegt waren, da keine größere Personenanzahl zusammentreffen durfte.
Allerdings haben mir alle bekannten Lehrkräfte es geschafft, diese Zeit zu
überbrücken -manche besser und manche schlechter. Der
online-Instrumentalunterricht ist glücklicherweise durch die technischen
Möglichkeiten heutzutage gut möglich, aber die technische Ausrüstung spielt
dabei doch eine Rolle. Es gibt viele verschiedene Programme und Medien, die den
Instrumentalunterricht bereichern können. Dazu zählen zum Beispiel die App
\emph{Garage Band}, das Programm \emph{DAW} (Digital Audion Workstation),
verschiedene Hardware-Produkte wie zum Beispiel Mikrofone, Laptops, Kopfhöhrer
usw. Nachdem die Lockdowns wieder vorbei sind und der Instrumentalunterricht
wieder in Präsenz stattfindet, meint man zunächste es hätte sich nichts
geändert. Auch das gemeinsame Musizieren konnte wieder fortgesetzt und
aufgenommen werden. Aber wie mir von vielen Seiten rückgemeldet wurde: die
gemeinsame Musizierpraxis hat den SChüler*innen extrem gefehlt. Dadurch, dass
auch der Schulunterricht phasenweise online ablief, fehlte den Lernenden die
gesamte soziale Interaktion und Kommunikation im realen Leben. Die Musikschulen
sind wieder zu ihrem Präsenzunterricht zurückgekehrt und auch die Ensembles
können wieder ihre Proben aufnehmen. Im Internet sprießen inzwischen aber immer
mehr Online-Musikschulen aus dem Boden, die ausschließlich den
Instrumentalunterricht über die Plattformen online anbieten, dazu zählen zum
Beispiel \emph{kinder-musikunterricht.de}, \emph{musikmachen-online.de} oder
\emph{MyMusicSchool.com}. Bei \emph{MyMusicSchool.com} wird ausschließlich
Online-Einzelunterricht angeboten, während sich die Online-Musikschule
\emph{kinder-musikunterricht.de} vor allem an sehr jungen Kindern und damit der
musikalischen Früherzierhung orientiert, weshalb sich wiederum alle Angebote auf
Gruppenunterricht beziehen. Die Altersspanne der Zielgruppe liegt bei dieser
Musikschule zwischen drei Monate bis hin zum Melodikaunterricht, der ab einem
Alter von sechs Jahren besucht werden darf. \autocite{online_musikschule_emp}
Die Seite \emph{musikmachen-online.de} bietet als eine der wenigen
Online-Musikschulen verschiedene Möglichkeiten zum gemeinsamen Musizieren an. \autocite{online_musikschule_mmo}
Bei allen Gruppenangeboten, die von Hausmusik über Saxophon-Quartett, Bandproben
bis zu Impro-Workshops reichen, wird nie gleichzeitig online musiziert, da dies
rein technisch nicht machbar ist. Stattdessen arbeiten sie mit Playalongs, zu denen die
Schüler*innen für sich in \emph{Breakout-Rooms} spielen können und wo sie Einzeln Feedback von der
Lehrperson erhalten. Die "Gruppen-Calls" laden dann zum Austausch und zur
gegenseitigen Inspiration ein. \autocite{online_musikschule_mmo}


\subsection{Digitale Kettenkomposition}
Eine weitere Möglichkeit das gemeinsame Musizieren in den
Online-Instrumentalunterricht zu integrieren ist die \emph{Digitale
Kettenkomposition}, die ich, wie im Vorwort bereits
angedeutet, zusammen mit Timo Langpap entworfen habe. Wir haben unsere
Instrumentalschüler*innen mit ihren verschiedenen Instrumenten eigene Musik entwerfen
lassen und wie einen Kettenbrief für die anderen Schüler*innen zugänglich
gemacht, sodass sie sich wiederum ein Stück aussuchen konnten, zu dem sie dann
wieder etwas improvisiert und das dann aufgenommen haben. So entstanden Songs
aus musikalischen Ketten, die bis zu sieben Mitwirkende hatten, während andere
Stücke nur von zwei Schüler*innen gestaltet wurden. Trotz allem war das ganze
leider sehr anonym und die Schüler konnten sich bis zuletzt nicht gegenseitig kennenlernen.
Positiv hingegen war, dass die Schüler*innen neugierig auf die Musik der anderen
Schüler*innen waren und selbst kreativ werden konnten, was bei allen Beteiligten einen
großen Motivationsschub veranlasst hat, da es doch ein größeres übergeordnetes
Projekt und Ziel gab, auf das mehrere Instrumentalisten aus verschiedenen Orten
gemeinsam hingearbeitet haben. 

\subsection{Wertung des Online-Instrumentalunterrichts}
Gemeinsames Musizieren im Online-Instrumentalunterricht ist aufgrund der Latenz
nicht möglich, soviel steht fest. Dabei ist es zunächst einmal egal, ob es
Online-Einzel- oder Gruppenunterricht ist. Das wird auch auf der Seite von
\autocite{online_musikschule_mmo} bei allen Gruppenangeboten deutlich.
Allerdings ist der soziale Austausch von Erfahrungen auch schon sehr wertvoll
und ein Online-Gruppenangebot deshalb trotz allem sinnvoll! Leider bieten aber
nicht viele Online-Musikschulen den gemeinsamen Gruppenunterricht an. Die
Online-Musikschule \emph{kinder-musikunterricht.de} macht zwar den Anschein und
bietet ausschließlich Gruppenunterrichte an, zeigt jedoch auf ihrer Website
nicht auf, wie sie den Online-Gruppenunterricht zu gestalten denken und vermittelt
dem Interessenten damit einen ziemlich falschen Eindruck. Außerdem sehe ich den
Online-Instrumentalunterricht gerade in den ersten Lebensjahren eines Kindes
sehr kritisch, da die Kinder nicht vor dem Laptop von einer Lehrkraft etwas
vorgespielt bekommen sollten, sondern es viel wertvoller ist beispielsweise das
soziale Verhalten zu schulen, womit auf der Inernet-Seite geworben wird. Dieses
soziale Verhalten können Kinder meiner Meinung nach aber nicht online lernen,
sondern nur im realen Austausch und in der Interaktion mit anderen Kindern. 
Außerdem sollten die Kleinkinder nach meiner persönlichen Ansicht nicht zu früh
vor dem Computer oder Laptop sitzen. Der Großteil der Angebote des
Online-Instrumentalunterrichts bietet ausschließlich Einzelunterricht an. 
Dieser Online-Markt zestört das gemeinsame Musizieren, da es wie eingangs
erwähnt nicht einmal im Einzelunterricht zwischen der Lehrkraft und dem/der
Schüler*in zustande kommen kann. So kann ein Anfänger nur sehr schwer in einen
\emph{Spielflow} gelangen. Das Angebot ist ein weiteres Beispiel
dafür, dass viele Instrumentalpädagog*innen noch nicht begriffen haben, wie
essentiell das gemeinsame Musizieren ist. Andererseits ist die Hemmschwelle ein
Instrument neu zu Lernen etwas herabgesetzt, da man zu Hause in seiner gewohnten
Umgebung und ohne große Fahrtwege ein Instrument lernen kann. Wenn dadurch
mehr Menschen das Musizieren für sich entdecken und es wieder mehr in den Alltag
von der ganzen Gesellschaft
integriert profitieren mit etwas Glück all die verschiedenen
Ensembles und bekommen mehr Zuwachs. Dies beglückt im besten Fall dann wieder
die einzelnen Spieler*innen, sodass sich ihre Motivation wieder stärkt.
Online-Einzel-Unterricht in Kombination mit einem Ensemble, in dem man
zusammenkommt und gemeinsam im echten Leben und gleichzeitig (ohne Latenz)
musiziert, schätze ich wiederum als sehr sinnvoll ein. 

