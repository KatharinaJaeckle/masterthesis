\section{El Sistema}

Der Musiker, Ökonom und Aktivist José Antonio Abreu gründete 1975 in Venezuela
das bis heute bestehende und sogenannte "El Sistema". 
Ursprünglich sollte die Ausbildung in der klassischen Musik den besonders verarmten Kindern aus den Vororten
eine Perspektive verschaffen. Das System in Venezuela sieht so aus, dass die
Kinder vier Stunden Proben und Unterrichte direkt nach der Schule haben und am
Wochenende zusätzliche Probentermine, die variieren. Abreus Ansicht und Vision ist, dass die
Musik als Vermittler von sozialer Entwicklung in der höchsten Form betrachtet
werden sollte, da sie die höchsten Werte wie Solidarität, Harmonie und
wechselseitiges Verständnis überträgt und die Fähigkeit dazu besitzt, eine
Gemeinschaft miteiander zu verbinden und große Gefühle auszudrücken. Das Motto
von der Organisation war "Play and fight", was die Entschlossenheit hinter "El
Sistema" zu stehen zum Ausdruck bringen soll, sowie die Vitalität und
gleichzeitig auch den politisch kritischen Geist zum Ausdruck bringt. Der
heutzutage sehr bekannte Dirigent Gustavo Dudamel wurde mit "El Sistema" groß
und sagt "music saved my life and has saved the lives of thousands of atrisk
children in Venezuela ... like food, like health care, like education, music has
to be a right for every citizen."%\autocite{wikipedia}% Das Besondere an diesem
Das Besondere an dem Projekt ist wie schon angedeuted wird, dass es eben auch
gerade in den Vororten von Caracas die Kinder mit ins Boot holte, die sonst
aufgrund ihrer schwierigen Lebensumstände und Armut keine Gelegenheit gehabt
hätten, mit der (klassischen) Musik in Berührung zu kommen und gleichzeitig
einen Weg aus der Armut, dem Drogenmissbrauch und der Kriminalität heraus zu
finden. Nachdem das Simon Bolivar Orchester nach und nach Erfolg erntete und international
immer bekannter wurde, bis es zuletzt in großen Konzertsälen Europas
konzertieren durfte, bekam es auch immer mehr finanzielle Unterstützung von der
venezuelanischen Regierung. Das pädagogische Programm weitete sich aber nicht
nur inlands aus, sondern fand bald in Amerika, Kanada, England, Kolumbien,
Portugal, Phillipinen und Peru seine Nachfolger und führte somit weltweit
hunderttausende von Kindern an die Musik heran. Etwas umstritten ist das Projekt
jedoch auch, worüber Geoffrey Baker neben einem Zeitungsartikel ein ganzes
Buch verfasste. Darin beklagt er den kulturellen Autoritarismus, die
Hyper-Disziplin, die Ausbeutung, den Wettbewerb und die
Geschlechter-Diskriminierung. 


\subsection{Übertragung nach Deutschland}

