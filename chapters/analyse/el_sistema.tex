\section{El Sistema}

\subsection{Gründer José Antonio Abreu}
\enquote{Ursprünglich wurde Musik von einer Minderheit für eine Minderheit
gemacht. Dann wurde es zur Kunst einer Minderheit für die Mehrheit, und jetzt
stehen wir am Anfang eines neuen Zeitalters, in dem Musik das Vorhaben einer
MEhrheit für die Mehrheit ist.} \autocite[5]{kaufmann:el_sistema} Das sind die
Worte des Visionärs, der El Sistema in die Welt gerufen hat. Aus dem Zitat geht
deutlich hervor, dass er die Musik vor allem für die Mehrheit zugänglich machen
will und das erreicht El Sistema ohne Weiteres, da es weltweit Anklang findet
und internaionale Wellen schlägt. Abreu stammt aus einer sehr musikaffinen
Familie, angefangen bei seinen Großeltern, die ihre Vorliebe zu der Opernwelt
aus Italien mitbrachten. Auch Abreus Mutter und Vater selbst musizierten, wo
sowohl die klassische Musik der Großeltern aus Europa eine Rolle spielte, als
auch die lateinamerikaniche Volksmusik, die sie umgab. Mit neun Jahren begann
Abreu das Klavierspiel zu lernen. Seine erste Musiklehrerin prägte ihn sehr. Vor
allem ihr Leitsatz "Niemand ist unmusikalisch, jeder kann ein Instrument
erlernen - es gibt keine Disqualifikation durch Unbegabtheit, sehr wohl aber ein
Fördern und Fordern in unterschiedlicher Ausprägung."
\autocite[20]{kaufmann:el_sistema} Außerdem schaffte es die Klavierlehrerin
ihren Schülern mit auf den Weg zu geben, dass man als Pianist nicht zwangsläufig
ein Einzelgänger sein muss und sich durch die Arbeit mit Chören integrieren, bei
Theateraufführungen und eigentlich bei jeder Art künstlerischer Betätigung
mitwirken kann. Neben dem Klavier lernte Abreu aber auch bald noch Geige.
Aufgewachsen in Barquisimeto, "das sich selbst als die musikalische Haupftstadt
Venezuelas bezeichnet(...)." \autocite[22]{kaufmann:el_sistema} hatte Abreu
durch eine sehr gute Musikschule die Möglichkeit Geige bei einem aus Riga
stammenden Geigenlehrer zu lernen. Außerdem gab es an dieser Musikschule auch
ein Orchester, in dem er bald mitspielen durfte, obwohl er
gerade erst dabei war das Geigenspiel zu lernen. Er wurde aber von seinen Mitspieler und
besonders vone einer anderen Geigerin so gut an die Hand genommen, dass er keine
frustrierenden Erfahrungen sammeln musste. Dieses Prinzip wurde auch bei der
Gründung seiner Kinder- und Jugendorchester später zu einem wichtigen Teil der
Methode. Später studierte Abreu Klavier, Orgel, Cembalo, Dirigieren und
Komposition, sowie Ökonomie in Caracas. In der Hauptstadt ist er auch zum ersten
Mal mit dem Elend der Vorstadt in den sogenannten Barrios konfrontiert. "Doch
auch die Zustände innerhalb des professionellen Musikbetriebs in Caracas, die
sich ihm durch seine Arbeit mit dem Orquesta Sinfónica Venezuela und in seinen
Kammermusikprogrammen erschließen, berühren ihn zutief: Der Musikbetrieb
verweigert dem eigenen Nachwuchs die berufliche Ausübung und bevorzugt
eingewanderte Musiker für die Positionen in
Orchestern."\autocite[28]{kaufmann:el_sistema} Nach seinen umfassenden Studien
und teilweise auch schon parallel beginnt Abreu zu arbeiten und wird später
zum Vorsitzenden der Wirtschaftskommission in der Finanzabteilung des
Abgeordnetenhauses. Dennoch bleiben ihm die schwierigen Umstände des Landes und
vor allem der Kinder immer noch belastend im Kopf und die Lösung des Problems
sieht er in einer guten Ausbildung, in der die Menschen dazu erzogen werden ihr
Leben selbstbestimmt und selbstbewusst zu
führen.\autocite[31]{kaufmann:el_sistema} 



\subsection{El Sistema}
El Sistema ist eine Orchesterbewegung, die weltweil viele Wellen schlug und für
viele Länder zum kultur- und sozialpolitischen Vorbild wurde, da es Kindern und
Jugendlichen in einem von schweren sozialen Problemen betroffenem Land durch die
Ermöglichung einer musikalischen Ausbildung eine gesellschaftsrelevante
Initiative schuf. 

Gründung: Die Gründungsidee entfaltete sich in Abreu über einen längeren
Zeitraum und wurde durch das Leiten eines Kammermusikensembles, in dem auch
seine Schwestern mitwirkten immer präsenter, zumal sie sich viel darüber
austauschten und philosophierten. Das Ensemble bestand aus acht Musikern, die
Abreu also schon als potentielle Lehrkräfte zu Hand hatte.
\autocite[34]{kaufmann:el_sistema} Was sein Konzept betrifft, so ist das
gemeinsame Musizieren der zentrale Gedanke. \enquote{Seine These, dass das Erlernen von
Instrumenten am besten im Zusammenspiel mit anderen und folglich im Orchester
erfolge, scheint ihm immer wieder aufs Neue
richtig.}\autocite[34]{kaufmann:el_sistema} Er wollte, dass die Kinder
\enquote{selbst die Kraft erfahren dürfen, die sich beim gemeinsamen Musikmachen
entwickelt.}\autocite[34]{kaufmann:el_sistema} Mit seinem Plan wollte er gleich
zwei sozialipolitische Themen lösen: er wollte den Kindern durch die Musik zu
einer besseren beruflichen Zukunft verhelfen und betrachtete gleichzeitig das
gemeinsame Musizieren als Werkzeug für eine bessere Bildung und grundsätzliche
Sozialisierung von Kindern und Jugendlichen vor allem aus den Armenvierteln. Seine Gründung
sollte also ein \enquote{Ausgangspunkt für eine strukturelle Veränderung in der
Allgemeinbildung, in der musikalischen Bildung und im professionellen
Musikangebot Venezuelas" sein.}\autocite[38]{kaufmann:el_sistema} Was gerade die
Benachteiligten angeht so war es Abreu ein Anliegen, dass aus den vielen
Niemanden in den Elendsvierteln viele Jemande werden, \enquote{die durch das gemeinsame
Musizieren an ihre eigene Zukunftsfähigkeit zu glauben
lernen.}\autocite[39]{kaufmann:el_sistema}

Die Ausbildung in der klassischen Musik soll den besonders verarmten Kindern aus
den Vororten eine Perspektive verschaffen. Das System in Venezuela sieht so aus,
dass die Kinder vier Stunden Proben und Unterrichte direkt nach der Schule haben
und am Wochenende zusätzliche Probentermine, die variieren. Abreus Ansicht und
Vision ist, dass die Musik als Vermittler von sozialer Entwicklung in der
höchsten Form betrachtet werden sollte, da sie die höchsten Werte wie
Solidarität, Harmonie und wechselseitiges Verständnis überträgt und die
Fähigkeit dazu besitzt, eine Gemeinschaft miteinander zu verbinden und große
Gefühle auszudrücken. Das Motto von der Organisation war \emph{Play and fight},
was die Entschlossenheit hinter El Sistema zu stehen zum Ausdruck bringen
soll, sowie die Vitalität und gleichzeitig auch den politisch kritischen Geist
zum Ausdruck bringt. Der heutzutage sehr bekannte Dirigent Gustavo Dudamel wurde
mit El Sistema groß und sagt \enquote{music saved my life and has saved the lives of
thousands of atrisk children in Venezuela ... like food, like health care, like
education, music has to be a right for every
citizen.}%\autocite{wikipedia}% 
Das Besondere an dem Projekt ist wie schon angedeuted wird, dass es eben auch
gerade in den Vororten von Caracas die Kinder mit ins Boot holte, die sonst
aufgrund ihrer schwierigen Lebensumstände und Armut keine Gelegenheit gehabt
hätten, mit der (klassischen) Musik in Berührung zu kommen und gleichzeitig
einen Weg aus der Armut, dem Drogenmissbrauch und der Kriminalität heraus zu
finden. Aus El Sistema geht auch das bekannte Simon Bolivar Orchester hervor,
das nach und nach Erfolg erntete und international immer bekannter wurde, bis es
zuletzt in großen Konzertsälen Europas konzertieren durfte. Es bekam immer
mehr finanzielle und politische Unterstützung von der venezuelanischen Regierung. Das
pädagogische Programm weitete sich aber nicht nur inlands aus, sondern fand bald
in Amerika, Kanada, England, Kolumbien, Portugal, Phillipinen und Peru seine
Nachfolger und führte somit weltweit hunderttausende von Kindern an die Musik
heran. Etwas umstritten ist das Projekt jedoch auch, worüber Geoffrey Baker
neben einem Zeitungsartikel ein ganzes Buch verfasste. Darin beklagt er den
kulturellen Autoritarismus, die Hyper-Disziplin, die Ausbeutung, den Wettbewerb
und die Geschlechter-Diskriminierung. 

\subsection{Das Lernen in einer musikalischen Praxisgemeinschaft}
Die Ausbildung in El Sistema funktioniert so, dass die Schüler keine Musikschule
besuchen, wie wir es hier in Deutschland kennen, sondern den \emph{Nucleo}, also den Kern oder das Zentrum der sogenannten "Orchesterbewegung" besuchen. Hier
spielen die Schüler von Beginn an in einem richtigen Orchester mit, wo
ernsthaft musiziert wird. Neben den Orchesterproben im Tutti gibt es sowohl
Stimmproben als auch begleitenden Einzelunterricht. Geleitet werden die Proben
von professionellen Musikern aber auch von fortgeschrittenen Instrumentalisten.
\autocite[45]{kaufmann:el_sistema} Der Vorteil dabei ist, dass die
fortgeschritteneren Musiker sich nicht weiter langweilen müssen, sondern im
nächsten Schritt zum Nachdenken über eine sinnvolle Vermittlung ihrer
Fähigkeiten angeregt werden. Außerdem erfahren sie durch die Rolle des Lehrens
auch nochmal neue Kompetenzen und haben die Möglichkeit ihr Instrument weiter zu
entdecken, indem sie durch die Anfänger womöglich mit Fragestellungen und
Problemen konfrontiert sind, die ihnen selbst bis dahin gar nicht begegnet sind.
Die Anfänger hingegen werden zwar ins kalte Wasser geworden, aber gleichzeitig
auch von der Gruppe gehalten und unterstützt.
%Vom wilden Lernen/Musizieren Lernen-auch außerhalb von Schule und Unterricht,
%Herausgegeben von Natalia Ardila-Mantilla und Peter Röbke.%
Das was die Orchesterbewegung ausmacht ist, dass es eine wirkliche
Praxisgemeinschaft ist, in der die Musik allgegenwärtig und fester
Bestandteil des Alltages ist. Diese musikalische Praxisgemeinschaft zeichnet
sich neben ihrer Komplexität durch ihre Heterogenität aus. Die Vielfalt durch
die die unterschiedlichen Persönlichkeiten mit ihren jeweiligen Neigungen und
Bedürfnissen ergibt. %vom wilden Lernen s. 161
Außerdem liegt bei dem orchestralen Musizieren der Fokus weniger auf dem
systematischen Erwerb von technischen Fähigkeiten. Viel mehr dient das
Instrument als Mittel zum Zweck und kann mehr wie ein Werkzeug betrachtet
werden. \enquote{Lernen ist der Prozess, in dem man die vollständige Teilhabe an dieser
Praxisgemeinschaft erwirbt.} Im besten Fall werden die Neuen von den Alten gut
aufgenommen, sodass es zu einem fruchtbaren Wechselspiel zwischen Erfahrenen und
Unerfahrenen kommen kann. Aber vor allem für den Anfänger kann ein großer Sog
entstehen und das Bedürfnis des Dazugehörens ein ausgeprägter Wunsch werden.
Durch die heterogene Gruppierung ergibt sich auch, dass die Praxisgemeinschaft
keine Teilnahmebedingungen bestimmt. Vielmehr versucht jedes Orchestermitglied
so viel und so gut es kann sein Bestes zu geben und beizutragen. Durch die
vielen Proben, die absichtlich viel Zeit der Kinder beansprucht, soll das
Musizieren wie eine zweite Muttersprache werden, die immer präsent ist.%S.162
Außerdem will Abreu durch sein System die \enquote{die theorielastige und auf den
bekannten Instrumentalunterricht fixtierte Methodik verworfen und durch das
Prinzip des gemeinsamen Musizierens ersetzt (werden.)}
\autocite[45]{kaufmann:el_sistema}


%Fortschritt des einzelnen Schüler?

%\subsection{Übertragung nach Deutschland}

