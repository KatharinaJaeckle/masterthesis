\section{El Sistema}

Der Musiker, Ökonom und Aktivist José Antonio Abreu gründete 1975 in Venezuela
das bis heute bestehende und sogenannte "El Sistema". Eine Orchesterbewegung,
die weltweil viele Wellen schlug und für viele Länder zum kultur- und
sozialpolitischen Vorbild wurde, da es Kindern und Jugendlichen in einem von
schweren sozialen Problemen betroffenem Land durch die Ermöglichung einer
musikalischen Ausbildung eine gesellschaftsrelevante
Initiative schuf.
Die Ausbildung in der klassischen Musik soll den besonders verarmten Kindern aus den Vororten
eine Perspektive verschaffen. Das System in Venezuela sieht so aus, dass die
Kinder vier Stunden Proben und Unterrichte direkt nach der Schule haben und am
Wochenende zusätzliche Probentermine, die variieren. Abreus Ansicht und Vision ist, dass die
Musik als Vermittler von sozialer Entwicklung in der höchsten Form betrachtet
werden sollte, da sie die höchsten Werte wie Solidarität, Harmonie und
wechselseitiges Verständnis überträgt und die Fähigkeit dazu besitzt, eine
Gemeinschaft miteiander zu verbinden und große Gefühle auszudrücken. Das Motto
von der Organisation war "Play and fight", was die Entschlossenheit hinter "El
Sistema" zu stehen zum Ausdruck bringen soll, sowie die Vitalität und
gleichzeitig auch den politisch kritischen Geist zum Ausdruck bringt. Der
heutzutage sehr bekannte Dirigent Gustavo Dudamel wurde mit "El Sistema" groß
und sagt "music saved my life and has saved the lives of thousands of atrisk
children in Venezuela ... like food, like health care, like education, music has
to be a right for every citizen."%\autocite{wikipedia}% Das Besondere an diesem
Das Besondere an dem Projekt ist wie schon angedeuted wird, dass es eben auch
gerade in den Vororten von Caracas die Kinder mit ins Boot holte, die sonst
aufgrund ihrer schwierigen Lebensumstände und Armut keine Gelegenheit gehabt
hätten, mit der (klassischen) Musik in Berührung zu kommen und gleichzeitig
einen Weg aus der Armut, dem Drogenmissbrauch und der Kriminalität heraus zu
finden. Aus El Sistema geht auch das bekannte Simon Bolivar Orchester hervor,
das nach und nach Erfolg erntete und international
immer bekannter wurde, bis es zuletzt in großen Konzertsälen Europas
konzertieren durfte, bekam es auch immer mehr finanzielle Unterstützung von der
venezuelanischen Regierung. Das pädagogische Programm weitete sich aber nicht
nur inlands aus, sondern fand bald in Amerika, Kanada, England, Kolumbien,
Portugal, Phillipinen und Peru seine Nachfolger und führte somit weltweit
hunderttausende von Kindern an die Musik heran. Etwas umstritten ist das Projekt
jedoch auch, worüber Geoffrey Baker neben einem Zeitungsartikel ein ganzes
Buch verfasste. Darin beklagt er den kulturellen Autoritarismus, die
Hyper-Disziplin, die Ausbeutung, den Wettbewerb und die
Geschlechter-Diskriminierung. 

\subsection{Das Lernen in einer musikalischen Praxisgemeinschaft}
Die Ausbildung in El Sistema funktioniert so, dass die Schüler keine Musikschule
besuchen, wie wir es hier in Deutschland kennen, sondern den "Nucleo", also den
den Kern oder das Zentrum der sogenannten "Orchesterbewegung" besuchen. Hier
spielen die Schüler von Beginn an in einem "richtigen" Orchester mit, wo
ernsthaft musiziert wird. Hier werden die Anfänger zwar ins kalte Wasser
geworden, aber gleichzeitig auch von der Gruppe gehalten.
%Vom wilden Lernen/Musizieren Lernen-auch außerhalb von Schule und Unterricht,
%Herausgegeben von Natalia Ardila-Mantilla und Peter Röbke.%
Das was die Orchesterbewegung ausmacht ist, dass es eine wirkliche
Praxisgemeinschaft ist, in der die Musik allgegenwärtig ist und fester
Bestandteil des Alltages ist. Diese musikalische Praxisgemeinschaft zeichnet
sich neben ihrer Komplexität durch ihre Heterogenität aus. Die Vielfalt durch
die unterschiedlichen Persönlichkeiten mit ihren jeweiligen Neigungen und
Bedürfnissen ergibt. %vom wilden Lernen s. 161
Außerdem liegt bei dem orchestralen Musizieren der Fokus weniger auf dem
systematischen Erwerb von technischen Fähigkeiten. Viel mehr dient das
Instrument als Mittel zum Zweck und kann mehr wie ein Werkzeug betrachtet
werden. "Lernen ist der Prozess, in dem man die vollständige Teilhabe an dieser
Praxisgemeinschaft erwirbt." Im besten Fall werden die Neuen von den Alten gut
aufgenommen, sodass es zu einem fruchtbaren Wechselspiel zwischen Erfahrenen und
Unerfahrenen kommen kann. Aber vor allem für den Anfänger kann ein großer Sog
entstehen und das Bedürfnis des Dazugehörens ein ausgeprägter Wunsch werden.
Durch die heterogene Gruppierung ergibt sich auch, dass die Praxisgemeinschaft
keine Teilnahmebedingungen bestimmt. Vielmehr versucht jedes Orchestermitglied
so viel und so gut es kann sein Bestes zu geben und beizutragen. Durch die
vielen Proben, die absichtlich viel Zeit der Kinder beansprucht, soll das
Musizieren wie eine zweite Muttersprache werden, die immer präsent ist.%S.162


\subsection{Übertragung nach Deutschland}

