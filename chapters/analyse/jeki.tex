\section{Jedem Kind sein Instrument}

Das Projekt \emph{Jedem Kind sein Instrument} wurde 2001 an der Waldorfschule in
Bochum von Mirjam Schieren und Christian Kröner ins Leben gerufen. Daraufhin
fand es zunächst an verschiedenen Grundschulen in Nordrhein-Westfalen
Verbreitung. Im Zusammenhang mit Bochum als Kulturhauptstadt Ruhr 2010, fand das
Projekt auch in der Politik Anklang und wurde ab 2011 allein durch die
Kulturstiftung des Bundeslandes Nordrhein-Westfalen gefördert. Es wurde
allerdings zunächst neben NRW nur in Hessen, Sachsen und Hamburg durchgeführt,
war also längst nicht flächendeckend in ganz Deutschland präsent. In vielen
Bundesländern ist es inzwischen jedoch in Planung. Neben \emph{Jedem Kind ein
Instrument} wurde das Projekt in \emph{JeKits} umgewandelt -\emph{Jedem Kind
Instrumente, Tanzen, Singen}, was das Spektrum etwas weitet. \enquote{Der
JeKi-Instrumentalunterricht basiert auf der gesellschaftlichen Übereinkunft,
dass das Instrumentalspiel eine wichtige Form der Teilhabe an Musikkultur
darstellt, die allen Kindern prinzipiell offen stehen
sollte.}\autocite[94]{krupp_schleussner:jeki} So wie das Bildungssystem in
Deutschland Bundesländersache ist, wird auch JeKits in den verschiedenen
Bundesländern unterschiedlich in den Schulalltag integriert. In
Nordrhein-Westfalen, wo JeKi seinen Ursprung hat, können die Schulkinder zu
jedem Jahreswechsel entscheiden, ob sie das Programm verlassen wollen.
\autocite[95]{krupp_schleussner:jeki} Allerdings muss man dazu wissen, dass JeKi
ab der zweiten Klasse kostenpflichtig ist, was wiederum zu einem unfreiwilligen
Verzicht führen kann, wenn es sich Eltern nicht leisten können, ihren Kindern
JeKi weiterhin zu finanzieren, oder eben doch den Sport als Hobby bevorzugen.
Die Schüler*innen bekommen die Instrumente kostenlos als Leihgabe zum Üben für
zu Hause und auch den Unterricht zur Verfügung gestellt. Außerdem gibt es in NRW
die Möglichkeit einen Antrag auf Gebührenbefreiung zu stellen, wodurch
gewährleistet sein soll, dass auch ärmere Kinder an dem Unterricht teilnehmen
können. In Hamburg hingegen ist das JeKi-Programm zwar kostenlos, aber dafür
besteht keine Entscheidungsfreiheit, ob man weiterhin an dem Programm teilnehmen
will.
\autocite[95]{krupp_schleussner:jeki}
JeKi und JeKits wird immer in Kooperation mit verschiedenen außerschulischen
Institutionen wie Musik- und Tanzschulen durchgeführt.

Zu JeKits selbst und wie es funktioniert: im ersten Schuljahr werden die
Schüler*innen an verschiedene Instrumente herangeführt und erst im
darauffolgenden zweiten Schuljahr beginnen die Kinder in kleinen Gruppen von
ihnen innerhalb des Angebotes ausgewählte Instrumente zu erlenen. Wieder ein
Jahr später gehört zu dem wöchentlichen Kleingruppenunterricht dann der Besuch
des jahrgangsübergreifenden Schulorchesters mit dazu. Im Vordergrund steht das
gemeinsame Musizieren. \enquote{JeKits will Kindern die Erfahrung des
Instrumentalspiels, des Tanzens oder des Singens als ästhetisches Handeln in der
Gruppe ermöglichen.}
\autocite{https://de.wikipedia.org/wiki/JeKits}
Sie lernen also ausschließlich in Kleingruppen oder größeren Gruppen gemeinsam
Musizieren und ihr Instrument zu beherrschen. Einzelunterricht ist in dem Modell
nicht vorgesehen.

%Instrumentaler Klassenunterricht Anselm Ernst S. 91 ff

\subsection{Pro und Contra}
Es gibt sehr viele Kritiker gegenüber JeKits. Ein Problem ist beispielweise,
dass durch die schnelle Expansion des Projektes die geeigneten Lehrkräfte fehlen
und die Instrumentalpädagog*innen zudem nicht ausreichend auf diese neuen
fordernden Unterrichtsformen vorbereitet sind. Außerdem sind die Kooperationen
mit den Musikschulen teilweise etwas kompliziert, da die Musikschulen mit den
Räumlichkeiten nicht gut auf große Gruppen eingestellt sind, sowie die Lagerung
und Versorgung der Instrumente eine gewisse Komplexität mit sich bringt. Was
wiederum schön ist, ist dass die Zahl der Kinder die ein Instrument spielen,
über Formen des Klassenmusizierens an Grundschulen wächst.
\autocite[119]{ernst:die_zukunftsfaehige_musikschule} Aber auch hier besteht das
Problem, dass die Kapazitäten der Musikschulen nicht ausreichen, alle Kinder
aufzunehmen, die darauffolgend weiter ihr Instrument lernen wollen. Hinzu kommt
auch, dass die Musikschulen keine finanziellen Subventionen mehr anbieten
können, sodass es zwangsläufig im späteren Verlauf zur finanziellen Beteiligung
der Eltern kommt und somit die Chancengeleichheit gefährdet ist. Allerdings
sprechen wir im späteren Verlauf von Einzelunterricht, der den Lernenden erteilt
wird. Selbstverständlich ist Einzelunterricht auch etwas teurer, als
Gruppenunterricht. Um die Kinder an die Musik heranzuführen, finde ich JeKits
eine gute Basis. Gleichzeitig finde ich es aber auch wichtig, dass sie ergänzend
so schnell wie möglich Einzelunterricht erhalten. Ein weiteres kritisches Thema
ist die Verteilung der Bildungsträger der verschiedenen Bundesländer, sodass es
für JeKits nicht möglich ist, sich bundesweit einheitlich zu entwickeln. So ist
das Projekt, welches durchaus gute Ansätze hat, leider sehr unterschiedlich in
Deutschland vertreten. 
