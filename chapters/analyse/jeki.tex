\section{Jedem Kind ein Instrument}

Das Projekt "Jedem Kind sein Instrument" wurde 2001 an der Waldorfschule in
Bochum von Mirjam Schieren und Christian Kröner ins Leben gerufen. Daraufhin
fand es an verschiedenen Grundschulen in Nordrhein-Westfalen Verbreitung. Im
Zusammenhang mit Bochum als Kulturhauptstadt Ruhr 2010, fand das Projekt auch in
der Politik Anklang und wurde ab 2011 allein durch die Kulturstiftung des
Bundeslandes Nordrhein-Westfalen gefördert. Es wurde allerdings zunächst neben
NRW nur in Hessen, Sachsen und Hamburg durchgeführt, war also längst nicht
flächendeckend in ganz Deutschland präsent. In vielen Bundesländern ist es
jedoch in Planung. Neben "Jedem Kind ein Instrument" wurde das Projekt in
"JeKits" umgewandelt -Jedem Kind Instrumente, Tanzen, Singen, was das Spektrum
etwas weitet. "Der JeKi-Instrumentalunterricht basiert auf der
gesellschaftlichen Übereinkunft, dass das Instrumentalspiel eine wichtige Form
der Teilhabe an Musikkultur darstellt, die allen Kindern prinzipiell offen
stehen sollte."\autocite[94]{krupp_schleussner:jeki} So wie das Bildungssystem
in Deutschland Bundesländersache ist, wird auch JeKi in den verschiedenen
Bundesländern unterschiedlich in den Schulalltag integriert. In
Nordrhein-Westfalen, wo JeKi seinen Ursprung hat, können die Schülkinder zu
jedem Jahreswechsel entscheiden, ob sie das Programm verlassen wollen.
\autocite[95]{krupp_schleussner:jeki} Allerdings muss man dazu wissen, dass JeKi
ab der zweiten Klasse kostenpflichtig ist, was wiederum zu einem unfreiwilligen
Verzicht führen kann, wenn es sich Eltern nicht leisten können, ihren Kindern
JeKi weiterhin zu finanzieren, oder eben doch den Sport als Hobby bevorzugen.
Nichts desto trotz bekommen die Schüler die Instrumente kostenlos als Leihgabe
zum Üben für zu Hause und auch den Unterricht zur Verfügung gestellt. Außerdem
gibt es in NRW die Möglichkeit einen Antrag auf Gebührenbefreiung zu stellen,
wodruch gewährleistet sein soll, dass auch weniger liquide Eltern ihre Kinder an
dem Unterricht teilhaben lassen können. In Hamburg hingegen ist das
JeKi-Programm zwar kostenlos, aber dafür besteht keine Entscheidungsfreiheit, ob
man weiterhin an dem Programm teilnehmen will.
\autocite[95]{krupp_schleussner:jeki}
JeKi und JeKits wird immer in Kooperation mit verschiedenen außerschulischen
Institutionen wie Musik- und Tanzschulen durchgeführt.

Zu dem JeKits selbst und wie es funktioniert: im ersten Schuljahr werden die
Schüler an verschiedene Instrumente herangeführt und erst im darauffolgenden
zweiten Schuljahr beginnen die Kinder in kleinen Gruppen ihre Instrumente zu
erlenen. Wieder ein Jahr später gehört zu dem wöchentlichen
Kleingruppenunterricht dann der Besuch des jahrgangsübergreifenden
Schulorchesters mit dazu. Im Vordergrund steht das gemeinsame Musizieren.
\enquote{JeKits will Kindern die Erfahrung des Instrumentalspiels, des Tanzens
oder des Singens als ästhetisches Handeln in der Gruppe ermöglichen.}
% (https://de.wikipedia.org/wiki/JeKits_%E2%80%93_Jedem_Kind_Instrumente,_Tanzen,_Singen)4.11.21.



Kritik: meine Annahme, dass es Quatsch ist, einem Schüler im Klassenmusizieren
ein Isntrument beizubringen.

Es gibt sehr viele Kritiker gegenüber JeKits. Ein Problem ist beispielweise,
dass durch die schnelle Expansion die geeigneten Lehrkräfte fehlte und die
Instrumentalpädagogen zudem nicht ausreichend auf diese neuen fordernden
Unterrichtsformen vorbereitet waren. Außerdem sind die Kooperationen mit den
Musikschulen teilweise etwas kompliziert, da die Musikschulen mit den
Räumlichkeiten nicht gut auf große Gruppen eingestellt ist, sowie die Lagerung
und Versorgung der Instrumente eine gewisse Komplexität mit sich bringt. Was
jedoch schön ist, dass "die Zahl der Kinder, die über Formen des
Klassenmusizierens an Grundschulen (wächst)." "Es wächst die Zahl der Kinder,
die über Formen des Klassenmusizierens an Grundschulen (...)) gewonnen werden."
S. 118 Aber auch hier besteht das Problem, dass die Kapazitäten der Musikschulen
nicht ausreichen, alle Kinder aufzunehmen, was jedoch sehr bedenklich ist. Hinzu
kommt auch, dass die Musikschulen keine finanziellen Subventionen mehr anbieten
können, sodass es zwangsläufig teuer für die Eltern wird und somit die
Chancengeleichheit vom Beginn des Projekts leider nicht weitergeführt werden
kann.

Instrumentaler Klassenunterricht Anselm Ernst S. 91 ff

Fortschritt der Einzelnen Schüler?



\subsection{Das Lernen in einer musikalischen Praxisgemeinschaft}
