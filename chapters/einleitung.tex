\addchap{Einleitung}


Thema vorstellen
"Ziel präsentieren"
"wichtige Quellen benennen"
Gliederung/Inhalt/Ausblick
Aufbau und Vorgehensweise: Gendern und Co. 


"Jedes Menschen Persönlichkeit, das bedeutet: seine Fähigkeiten, seine Art zu
denken und zu fühlen, wird durch Umstände und Umgebung geschnitzt und
gemeißelt." (S. 20 shinichi Suzuki)

Das gemeinsame Musizieren zieht sich bei meinem Lebenslauf wie ein roter Faden
durch die ganze Laufbahn. Als Instrumentallehrerin habe ich allerdings auch
beobachtet, wie es für Schüler ist, die lediglich den
Einzel-Instruemntalunterricht kennen. Dabei ist mir aufgefallen, dass ein
wichtiger Bestandteil fehlt, nämlich der für mich die Musik ausmacht. Das
gemeinsame Musizieren. Wenn man alleine den Aspekt des
Differenziellen Lernens betrachtet wird schnell klar, das Einzelunterricht
alleine nicht ausreicht um einen Schüler motiviert am Instrument zu halten und
ihn weiter zu bringen. Da sich auf der sozialen Ebene im Miteinander so viel
ereignet, kann eine einzelne Lehrkraft nicht alle Ebenen abdecken und bedienen.
Oft ist in unserem Musikschulsystem das gemeinsame Musizieren aber gar nicht
vorgesehen, sondern vielmehr ein optionales Zusatzangebot, im Orchester, im Chor
oder in der Band zu spielen. Dieses Zusatzangebot ist auch nicht immer gegeben,
je nach Größe der Musikschule und schon sind die Schüler schnell auf sich und
ihren Instrumentallehrer allein gestellt. In anderen Ländern gibt es andere
Sitten und somit auch verschiedene Systeme, in denen Instrumentalschüler im
Miteinander das Instrument lernen. Einige will ich deshalb näher beleuchten und
untersuchen, inwieweit sie für die Lernenden wertvoll sind. Im Vordergrund
meiner Arbeit soll stehen, warum das gemeinsame Musizieren so
wichtig ist.
Andreas Doerne schreibt sehr zutreffend "(...) dass Musik in ihrem Wesen Mitteilung, Kommunikation, Gestaltung, das
Vermitteln einer Botschaft ist, die in Sternstunden über das Erleben der
Einzelnen hinausreicht und sie miteinander verbinden kann." S. 11
%(andreas doerne?) 
Das "gemeinsame Musizieren" will allerdings gelernt sein, genau so wie es
aber auch "gemeinsam Musizieren Lernen" möglich ist. Wie das in Deutschland,
Venezuela oder auch Japan funktioniert und was in welchem System der Fall ist, werde ich in dieser Arbeit
genauer beleuchten. Des Weiteren werde ich in meiner Arbeit die verschiedenen
Systeme in Hinblick auf das sozialen Miteinander und ihren Zusammenhang mit
dem Instrumentalunterricht analyiseren. Außerdem will ich den Entwicklungs- und
Fortschrittprozess der Schüler heausarbeiten, die in den verschiedenen Systemen doch sehr unterschiedlich sein
können.

Da die Vielfalt an Terminologien zur Benennung der verschiedenen
Geschlechter heutzutage sehr groß ist, werde ich fortfolgend in meiner Masterarbeit
absatzweise wechselnd männlich und weiblich gendern. Ich weise an dieser Stelle
daraufhin, dass sowohl alle männlichen, weiblichen und auch diversen
Geschlechter mit einbezogen werden und gemeint sind. 




"Musik als Kommunikationsmittel" S. 17

Aristoteles: "der Schlüssel zu einem glücklichen Leben besteht im Miteinander,
das Leben in der Gemeinschaft unter Freunden." (S. 9 "Die Kunst zu unterrichten)

"Sie muss gelernt und ein Leben lang geübt werden." (auch S. 9)


"Musik ist Teil der Symbolwelt des Menschen: Mitteilung, Kommunikation,
Interaktion." Wolfgang Suppan, (S. 91 Andreas Doerne)

