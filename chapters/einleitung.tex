\addchap{Einleitung}

\enquote{Musik ist Teil der Symbolwelt des Menschen: Mitteilung, Kommunikation,
Interaktion.} \autocite[91]{doerne:umfassend_musizieren}

Das gemeinsame Musizieren zieht sich bei meinem Lebenslauf wie ein roter Faden
durch die Laufbahn. Als Instrumentallehrerin habe ich beobachtet, wie es für
Schüler*innen ist, die lediglich den Einzel-Instrumentalunterricht kennen. Dabei
ist mir aufgefallen, dass ein wesentlicher Bestandteil fehlt: das gemeinsame
Musizieren. Unser System der Musikschulen und die musikalischen Ausbildung ist
hauptsächlich darauf ausgelegt, erst einmal Einzelunterricht zu besuchen um ein
Instrument zu lernen. Das Spielen in einem Ensemble ist nur eine eventuelle
Ergänzung für Diejenigen, die besonders Freude an ihrem Instrument haben. Wenn
man alleine den Aspekt des
\emph{Informellen Lernens}
betrachtet wird schnell klar, dass Einzelunterricht alleine nicht ausreicht um
Lernende motiviert am Instrument zu halten. Da sich auf der sozialen Ebene im
Miteinander so viel ereignet, kann eine einzelne Lehrkraft nicht alle Ebenen
abdecken und bedienen. Oft ist in unserem Musikschulsystem das gemeinsame
Musizieren aber gar nicht vorgesehen, sondern vielmehr ein optionales
Zusatzangebot, in Form von einem Mitwirken im Orchester, im Chor oder in der
Band. Dieses Zusatzangebot ist aber auch nicht immer gegeben; besonders bei
kleineren Institutionen oder auch bei Privatschüler*innen sind die
Instrumentallehrer*innen mit ihren Schüler*innen auf sich alleine gestellt. Das
\enquote{gemeinsame Musizieren} will gelernt sein, genau so wie aber auch
\enquote{gemeinsam Musizieren Lernen} möglich ist. Was ist die bessere Variante und
schließt das Eine das Andere aus? Zunächst werfe ich einen Blick in die Theorie
und werde verschiedene Aspekte von Gruppenprozessen herausarbeiten. Wie das
gemeinsame Musizieren in anderen Ländern, z.B. Venezuela oder auch Japan,
funktioniert, werde ich in dieser Arbeit genauer beleuchten. Anschließend wird
betrachtet, wie der gängige Weg eines Instrumentalschülers in Deutschland
aussieht. Des Weiteren werde ich die verschiedenen Systeme in Hinblick auf das
sozialen Miteinander und ihren Zusammenhang mit dem Instrumentalunterricht
analysieren. Dabei arbeite ich die Quintessenzen der jeweiligen Systeme heraus,
und warum das gemeinsame Musizieren so wichtig ist. Ziel ist es, daraus ein
mögliches Szenario für ein besser in das System integriertes
\enquote{gemeinsames Musizieren Lernen} in Deutschland abzuleiten.


Da die Vielfalt an Terminologien bezüglich der verschiedenen Geschlechter
heutzutage sehr groß ist, werde ich in meiner Masterarbeit die jeweiligen
Wörtert gendern. Ich weise an dieser Stelle daraufhin, dass sowohl alle
männlichen, weiblichen und auch diversen Geschlechter mit einbezogen und gemeint
sind. 
