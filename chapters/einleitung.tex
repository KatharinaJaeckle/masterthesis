\addchap{Einleitung}

"Musik ist Teil der Symbolwelt des Menschen: Mitteilung, Kommunikation,
Interaktion." \autocite[91]{doerne:umfassend_musizieren}

Das gemeinsame Musizieren zieht sich bei meinem Lebenslauf wie ein roter Faden
durch die Laufbahn. Als Instrumentallehrerin habe ich beobachtet, wie es für
Schüler ist, die lediglich den Einzel-Instruemntalunterricht kennen. Dabei ist
mir aufgefallen, dass ein wesentlicher Bestandteil fehlt: das gemeinsame
Musizieren. Wenn man alleine den Aspekt des Differenziellen Lernens betrachtet
wird schnell klar, dass Einzelunterricht alleine nicht ausreicht um einen
Schüler motiviert am Instrument zu halten. Da sich auf der sozialen Ebene im
Miteinander so viel ereignet, kann eine einzelne Lehrkraft nicht alle Ebenen
abdecken und bedienen. Oft ist in unserem Musikschulsystem das gemeinsame
Musizieren aber gar nicht vorgesehen, sondern vielmehr ein optionales
Zusatzangebot, in Form von einem Mitwirken im Orchester, im Chor oder in der
Band. Dieses Zusatzangebot ist aber auch nicht immer gegeben; besonders bei
kleineren Institutionen oder auch bei Privatschülern, sind die
Instrumentallehrer mit ihren Schülern auf sich alleine gestellt. Das "gemeinsame
Musizieren" will gelernt sein, genau so wie aber auch "gemeinsam Musizieren
Lernen" möglich ist. Was ist die bessere Variante und schließt das Eine das
Andere aus? Zunächst werfe ich einen Blick in die Theorie und werde die
verschiedenen Aspekte die Gruppenprozessen mit sich bringen, herauszuarbeiten.
Wie das gemeinsame Musizieren in anderen Ländern wie Venezuela oder auch Japan
funktioniert, werde ich in dieser Arbeit genauer beleuchten. Anschließend wird
betrachtet, wie der gängige Weg eines Instrumentalschülers in Deutschland
aussieht. Des Weiteren werde ich die verschiedenen Systeme in Hinblick auf das
sozialen Miteinander und ihren Zusammenhang mit dem Instrumentalunterricht
analyiseren. Dabei arbeite ich die Quintessenzen der jeweiligen Systeme heraus,
und warum das gemeinsame Musizieren so wichtig ist. Ziel ist es, daraus ein
mögliches Szenario für ein weiter verbreitetes und eingebettetes "gemeinsam
Musizieren Lernen" in Deutschland abzuleiten.


Da die Vielfalt an Terminologien bezüglich der verschiedenen Geschlechter
heutzutage sehr groß ist, werde ich in meiner Masterarbeit
absatzweise wechselnd männlich und weiblich gendern. Ich weise an dieser Stelle
daraufhin, dass sowohl alle männlichen, weiblichen und auch diversen
Geschlechter mit einbezogen und gemeint sind. 
