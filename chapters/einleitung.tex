\addchap{Einleitung}
Das gemeinsame Musizieren zieht sich bei meinem Lebenslauf wie ein roter Faden
durch die ganze Laufbahn. Als Instrumentallehrerin habe ich allerdings auch
beobachtet, wie es für Schüler ist, die lediglich den
Einzel-Instruemntalunterricht kennen. Dabei ist mir aufgefallen, dass ein
wichtiger Bestandteil der für mich die Musik ausmacht fehlt. Nämlich das
gemeinsame Musizieren. Wenn man alleine den Aspekt des
Differenziellen Lernens betrachtet wird schnell klar, das Einzelunterricht
alleine nicht ausreicht um einen Schüler motiviert am Instrument zu halten und
ihn weiter zu bringen. Da sich auf der sozialen Ebene im Miteinander so viel
ereignet, kann eine einzelne Lehrkraft nicht alle Ebenen abdecken und bedienen.
Oft ist in unserem Musikschulsystem das gemeinsame Musizieren aber gar nicht
vorgesehen, sondern vielmehr ein optionales Zusatzangebot, im Orchester, im Chor
oder in der Band zu spielen. Dieses Zusatzangebot ist auch nicht immer gegeben,
je nach Größe der Musikschule und schon sind die Schüler schnell auf sich und
ihren Instrumentallehrer allein gestellt. In anderen Ländern gibt es andere
Sitten und somit auch verschiedene Systeme, in denen Instrumentalschüler mehr im
Miteinander das Instrument lernen. Einige will ich deshalb näher beleuchten und
unteruchsen, inwieweit sie für die Lernenden wertvoll sind. Im Vordergrund
meiner Arbeit soll aber stehen, warum das gemeinsame Musizieren so
wichtig ist und das werde ich ich im Folgenden beleuchten. 


"(...) dass Muisk in ihrem Wesen Mitteilung, Kommunikation, Gestaltung, das
Vermitteln einer Botschaft ist, die in Sternstunden über das Erleben der
Einzelnen hinausreicht und sie miteinander verbinden kann." S. 11

"Musik als Kommunikationsmittel" S. 17

Aristoteles: "der Schlüssel zu einem glücklichen Leben besteht im Miteinander,
das Leben in der Gemeinschaft unter Freunden." (S. 9 "Die Kunst zu unterrichten)

"Sie muss gelernt und ein Leben lang geübt werden." (auch S. 9)
 