\section{Gruppenprozesse}

\emph{Gemeinsam musizieren Lernen} setzt voraus, dass wir eine Gruppe von
Instrumentalist*innen beisammen haben. Die Gruppe kann ganz unterschiedlich
zusammengesetzt sein. Sie kann unterschiedliche Alters- und Lernniveaus haben,
sie kann sich aus verschiedenen Instrumentalist*innen zusammensetzen oder auch
ausschließlich aus Spieler*innen mit ein und demselben Instrument. Eine Gruppe
kann homogen nur aus Anfänger*innen, oder Profi-Musiker*innen bestehen, sie kann
genreübergreifend arbeiten, improvisatorisch, oder nur mit Noten. Die
Teilnehmer*innen können freiweillig dabei sein, sie können verpflichtet worden
sein oder mit dem Ensemble ihren Lebensunterhalt verdienen. Soweit kann sich das
Freiwillige mit dem Notwendigen verbinden.
Bei erwachsenen Laienmusiker*innen
ist ein wesentliches Motiv die musikalische Betätigung und das Wiedererlangen
der sozialen Integration. Außerdem kann die Größe der Gruppe sehr stark
variieren. Wenn man von dem gemeinsamen Musizieren an sich spricht, fängt dieses
beim Musizieren zwischen Lehrer*in und Schüler*in an. Die Definition einer
Gruppe wiederum gilt \enquote{ab 3 Personen, deren Mitglieder sich über einen
längeren Zeitraum in regelmäßigem Kontakt miteinander befinden, gemeinsame Ziele
verfolgen und sich als zusammengehörig empfinden.}
\autocite{wikipedia:gruppe}
Des Weiteren kommt man bei der Gruppenzusammensetzung sehr schnell auf das Thema
der Hierarchie und wie eine Gruppe arbeitet. Wird sie von
Instrumentallehrer*innen angeleitet oder von einer Dirigent*in? Oder ist die
Gruppe selbstbestimmt, wie beispielweise das junge Orchester \emph{Stegreif}?
Außerdem ist es interessant zu betrachten, wie die Teilnehmer*innen innerhalb
der Gruppe interagieren. So vielfältig wie sich eine Gruppe zusammensetzen und
sich eine Gruppendynamik entwickeln kann, sind auch die Anforderungen an die
Instrumentalpädagog*innen.



\subsection{Aktionsformen in der Gruppe}

Im Unterschied zu \enquote{einer zweipoligen Interaktionsform im
Einzelunterricht ergeben sich mehrpolige Interaktionsformen im Gruppenuntericht,
was Unterrichtssituationen für gewöhnlich komplexer erscheinen lässt.}
\autocite[30]{losert:die_kunst_zu_unterrichten} Wie sich im späteren Verlauf heraus stellen wird,
sind in manchen Unterrichts-Modellen für die Gruppe auch mehrere Lehrer*innen
vorhergesehen. Nichtsdestotrotz bedarf es aber bei den Lehrkräften untereinander
einer präzisen Absprache der Rollenverteilung und einen Gesamtüberblick, den die
Lehrkraft immer haben sollte. Je nachdem wie die räumliche Situation es erlaubt,
können sogar verschiedene Spielräume für verschiedene Aktionsformen genutzt und
die Gruppe räumlich aufgeteilt werden. So kann es beispielsweise Phasen geben,
in denen jeder für sich tätig ist, Partnerarbeit, kleinere Gruppenunterteilungen
oder die Arbeit in der Gruppe als Ganze. Außerdem kann man auch im selben
Raum unterschiedlich agieren, dazu unter dem Kapitel \emph{Bedeutung für den
Lehrer} später mehr. In der Elementaren Musikpädagogik basieren alle Formen auf
dem beziehungsorientierten Arbeiten, in dem \enquote{über die Beziehungen, die
zur Musik aufgebaut werden sollen, […] auch die Kontakte mit anderen
Gruppenmitgliedern im Fokus} \autocite[10]{dartsch:kern_des_musizierens} stehen.
Mit dem ganzheitlichen Menschen im Blick, soll neben der angestrebten
Persönlichkeitsbildung die soziale Kommunikation ausgebaut werden.
\subsection{Psychosoziales Lernen in der Gruppe und Persönlichkeitsbildung}
Sich als Individuum und gleichzeitig auch als Teil der Gruppe zu erleben schult
die Persönlichkeitsentwicklung, da das Leben in der Kollektivität und
das Leben mit uns selbst, also der Individualität ein existentielles
Wechselspiel darstellt. Innerhalb von Gruppenprozessen sind die Musizierenden
immer im Geben und Nehmen, außerdem müssen sie sich klanglich in das Ensemble
einfügen, wofür zunächst einmal die Wahrnehmung des eigenen Klanges erforderlich
ist und im nächsten Schritt, die Wahrnehmung der Klänge der Mitspieler. Dazu
zählt beispielweise auch die Lautstärke. Haben die Musizierenden diese beiden
Faktoren herausgefunden, geht es im dritten Schritt darum, sich in das
Gesamtbild einzuordnen. Dieses Wechselspiel erleben wir auch im
gesellschaftlichen Miteinander, wenn Meinungen ausgetauscht werden. Die
Individualität in der Gruppe \enquote{[…] zeigt sich gerade auch im Umgang mit
anderen, Individualität entwickelt sich im Vergleich zu und in Abgrenzung von
anderen oder auch dadurch, dass Positionen in einer Gruppe besetzt
werden.}\autocite[95]{ribke:emp} Das kennt jeder aus seinen eigenen
Rollenbildern innerhalb einer Familie oder Verwandschaft. Die psychischen
Aspekte, welche alle die eigene Persönlichkeit betreffen sind solche wie die
emotionale, intuitive aber auch rationale, intelektuelle Ebenen die im
Wechselspiel zu den sozialen Aspekten und damit allen Faktoren und Einflüssen
der Umwelt stehen. Die Persönlichkeitsentwicklung geschieht dadurch, dass der
Mensch seine gesammelten Ich-Informationen auf verschiedensten Ebenen mit der
sozialen Umwelt in Bezug setzen muss und diese dann wieder für sich selbst
auswertet. Gleichzeitig ist die zwischenmenschliche soziale Interaktion der
Grundbaustein für musikalische Prozesse! Dies gilt besonders für die Elementare
Musikpädagogik, wo die musikalischen Prozesse bis zu einem gewissen Grad noch
sehr offen gehalten sind. Die Gruppe ist automatisch immer ein Korrektiv, da
sich keine einzelnen Meinungen gegenüberstehen, sondern die Gruppe als Kollektiv
ihre Richtlinien vorgibt. Die Entwicklung der Persönlichkeit sollte außerdem
auch ein wichtiges Erziehungsziel von Instrumentalpädagog*innen sein, da die
Schüler*innen in Instrumentalgruppen auch die Chance haben, die Ordnung der
Gemeinschaft noch einmal anders als in der Schule kennen zu lernen.
\enquote{Dazu müssten sie zu Eigenaktivität, Selbsttätigkeit und konsequenter
eigenständiger Arbeit hingeführt
werden.}\autocite[64]{losert:die_kunst_zu_unterrichten} Betrachtet man einmal
den Effekt des informellen Lernens, nachdem die Schüler*innen durch ihre
eigene Wahrnehmung und Interessen das aus dem Unterricht herausfiltern was für
sie interessant ist, so darf man nicht außer Acht lassen, dass das Lernen der
Schüler*innen untereinander eine sehr zentrale Rolle spielt. \enquote{Ein
Großteil musikalischen Lernens findet über Imitation statt
(…).}\autocite[98]{doerne:umfassend_musizieren}


\subsection{Hierarchie}

Eng mit den Rollenbildern verknüpft ist auch die Hierarchie, die beim Lernen der
Musizierenden untereinander eine wichtigere Rolle spielt. So gibt es oft
diejenigen, die lauter sind und diejenigen, auf die man weniger hört. In manchen
Ensembles haben bestimmte Personen einen höheren Rang als Andere. Nicht zuletzt
ist das abhängig davon, ob die Gruppe musikalisch angeleitet wird, oder
selbstbestimmt demokratisch alle Entscheidungen trifft. Wenn die Gruppe von
einem Dirigenten oder auch Lehrer*in angeleitet wird, gibt es diejenigen
Personen, die in ihrem Auftreten authentisch sind und auch manchmal diejenigen,
die es nicht sind. Für die Instrumentalpädagog*innen, die nach dem heutigen
System ausgebildet werden, spielt diese Qualifikation gar keine Rolle in der
Ausbildung und wird leider auch zu wenig geschult. So sollte sie mehr in den
Fokus genommen werden, sodass die Lehrkraft, die vor einer Gruppe steht auch mit dieser
umzugehen weiß. Die \enquote{natürliche Hierarchie} innerhalb des Ensembles kann
man sich auch positiv zu Nutze machen, indem man beispielsweise
fortgeschrittenere Schüler*innen zu Mentoren erklärt und sie dadurch
mitverantwortlich sind für die musikalische Entwicklung und den Fortschritt des
Ensembles.\autocite[95]{doerne:umfassend_musizieren} Hierarchien können das
Ensemblespiel aber auch negativ beeinflussen, wenn einzelne Mitspieler*innen
ausgegrenzt oder unterdrückt werden oder auch wenn der/die Lehrer*in oder
Dirigent*in ihrer/seiner Rolle nicht gerecht werden kann. \enquote{Im für den
Lehrer schlimmsten Fall hat er nicht nur einen Konflikt mit einem einzelnen
Schüler, sondern unter Umständen das ganze Ensemble gegen
sich.}\autocite[94]{mitzscherlich:musikpsychologie}



\subsection{Kommunikation}

Der Instrumentalunterricht stellt ein komplexes Geschehen dar, der durch
kommunikative Prozesse gekennzeichnet ist. Unter Kommunikation ist hier als Art
und Weise zu verstehen wie sich Lehrkräfte und Schüler*innen begegnen,
miteinander umgehen und sich mitteilen. Das kann sowohl im musikalischen
Miteinander sein als auch außerhalb des eigentlichen Musiziererprozesses sein und
somit ist die Kommunikation allgegenwärtig. \enquote{Sobald der Mensch
musiziert, betritt er einen kommunikativen Raum. Der Musizierende kommuniziert
mittels Musik, mit der Musik, mit seinem Instrument, mit der Komponistin, mit
sich selbst, mit seinen Mitmusikerinnen, mit dem Publikum und über das Publikum
mit der Gesellschaft, in der er lebt.}
\autocite[56]{doerne:umfassend_musizieren} Diese ganzen Ebenen werden durch das
Musizieren erreicht und sind gleichzeitig solch eine Wucht an Informationen und
Aufmerksamkeitsquellen, die alle gleichzeitig von dem/der Einzelnen einzuordnen
sind, dass das Musizieren selbst nur noch einen kleinen Teil ausmacht. Genau
dieses Jonglieren und Einordnen von Informationen, die jede/r Einzelne wahrnimmt
und gleichzeitig das Abrufen von seinen instrumentalen Fertigkeiten bedarf einer
hohen Konzentration.

Eine weitere Ebene, die durch die Musik ins Spiel gebracht wird und eng mit dem
Thema der Identität verknüpft ist, ist die kulturelle Ebene. Abgesehen von den
offensichtlich unterschiedlichen Kulturen, die durch verschiedene Länder
zusammen kommen, gibt es auch innerhalb eines Landes durch die Bandbreite
verschiedener Genres sehr verschiedene Beheimatungen. \enquote{Musik ist
Bestandteil des soziokulturellen Raumes, der entsteht, wenn mehrere Menschen
miteinander in Kontakt treten. Als solche kann sie vielerlei kommunikative
Funktionen erfüllen.} \autocite[56]{doerne:umfassend_musizieren} Der
soziokulturelle Raum entsteht durch die unterschiedlichen Identitäten und die
Musik ist das kommunikative Moment, das die Musizierenden universell verbindet,
egal wie unterschieldlich die Prägungen sind. Die Musik ist in jedem Alltag
eines Menschen verankert, nur in unterschiedlicher Ausprägung, Genre und
Gestaltung.

% Anselm Ernst schreibt in seinem Buch \emph{Die zukunftsfähige Musikschule}:
% \enquote{Musikalische Kommunikation durchzog (früher) den ganzen Alltag.}
% \autocite[37]{ernst:die_zukunftsfaehige_musikschule} Heutzutage konsumieren
% zwar sehr viele Menschen die Musik, aber sie gestalten sie aktiv weniger. Wenn
% sie jedoch selbst aktiv musizieren, dann kann das heutzutage auch bedeuten,
% dass jemand etwas vor dem Computer sitzend produziert, wodurch die
% Kommunikation mit Mitmenschen ersteinmal komplett wegfällt. Dabei spielt die
% Kommunikation in Unterrichtssituationen, sowie auch in Gruppen, die gemeinsam
% Musizieren, die zentrale Rolle.

Bei näherer Betrachtung des Vorgangs
innerhalb eines Musizierenden wird deutlich, dass er/sie durch seinen/ihren
soziokulturellen Hintergrund die Musik auf seine Weise formt, während aber
die Musik
wiederum auch auf den Musiker einwirkt. Andreas Doerne beschreibt dieses Vorgang
sehr zutreffend: \enquote{Der Musizierende prägt die Musik genauso wie die Musik
den Musizierenden prägt. Angestoßen durch Kommunikation, beginnen beide Welten -
die des Musikers und die der Musik - sich zu verbinden, indem sie zu einer
gemeinsamen Bewegung, einem gemeinsamen Atmen, einem gemeinsamen Fühlen und
Denken verschmelzen.} \autocite[60]{doerne:umfassend_musizieren}

Bleiben wir bei dem/der einzelnen Musizierenden, so kommt noch die
Kommunikationsebene zwischen dem/der Musizierenden und seinem Instrument hinzu.
Hierbei spielt es sicherlich eine Rolle, ob der/die Musizierende sein eigenes
Instrument besitzt, wie beispielsweise ein Bläser, Streicher o.Ä., während die
Pianisten weniger die Möglichkeit haben mit ihrem konkreten eigenen Instrument
zusammen zu wachsen. Dennoch tritt der/die Musiker*in mit dem bespielten
Instrumenten während des Spielens in eine enge Verbindung und es kommt zu einem
gleichberechtigten Austausch, sodass bei der/dem Musizierenden der Eindruck
entstehen kann, \enquote{dass er selbst wie auch das Instrument zugleich Sender
und Empfänger von Botschaften ist.} \autocite[59]{doerne:umfassend_musizieren}
Bei Sänger*innen, deren Instrument die eigene Stimme ist, verhält es sich
vermutlich ähnlich, wobei hier die engste persönliche Beziehungsebene zwischen
dem Instrument, also der Stimme, und dem/der Musiker*in selbst besteht. Eine
weitere Kommunikationsebene ist zwischen dem/der Musizierenden und dem Werk, das
er spielt, was auch wiederum mit dem Genre und damit wieder mit der kulturellen
Ebene eng zusammenhängt. \enquote{Setzt sich ein Musiker mit einem Musikwerk
auseinander, beschäftigt er sich gleichzeitig immer auch mit der Körper-,
Gefühls-, und Gedankenwelt des Komponisten.}
\autocite[59]{doerne:umfassend_musizieren} Indem er/sie ein Stück interpretiert
und reflektiert, welche Stellen was aussagen und welche Emotionen verkörpern
oder musikalisch ausdrücken sollen, kann eine sehr enge  Verbindung mit dem/der
Komponist*in aufgebaut werden. Am plakativsten sind Opernrollen, in denen die
Sänger*innen durch den Text gestützt eine ganz klar vorgegebene Rolle erfüllen.
Bei der Improvisation hingegen lässt der/die Musizierende ihren eigenen Gefühlen
und ihrer Intuition freien Lauf. Durch die Kommunikation mit Mitspielern,
bringen die Mitspieler*innen sich gegenseitig auf verschiedene musikalische
Ideen. Die Kommunikation kann hierbei dialogich oder aber auch monologisch
ablaufen. \autocite[61]{doerne:umfassend_musizieren} Man kann aber auch den/die
Musiker*in selbst als ein Ensemble betrachten, das aus \enquote{divergenten
Teilpersönlichkeiten und unterschiedlichen psychischen Dynamiken} besteht.
\autocite[64]{doerne:umfassend_musizieren} Da jede/r Musiker*in an jedem Tag
etwas anders agiert, die Stimmungen unterschiedlich sind und die Situationen
immer ein wenig abgeändert ablaufen, kommt immer wieder eine leicht veränderte
Interpretation eines Stückes zu stande.  Im Gegenteil dazu ist es viel
schwieriger, ein Stück immer exakt gleich abzurufen. Dies ist in vielen
Musizierprozessen und Stilistiken auch nicht die Intention, insofern sind die
verschiedenen psychischen Dynamiken ein ideales Gestaltungsmittel, welches ganz
von alleine in das Musizieren einfließt. Ziel von Musizierprozessen sollte das
transportieren und vermitteln der musikalischen Inhalte sein und nicht das
einfache Abrufen von technischen Mustern.Ein weiterer Aspekt der Kommunikation
betrifft die Konzert- und Aufführungssituation, in welcher ein Publikum mit
dabei ist. Auch wenn die Musiker*innen während dem Spielen mit dem
vollbringen des musikalischen Werkes beschäftigt sind, so sind manche
Aufmerksamkeitsfelder des Musizierenden trotzdem bei der Bühnensituation, dem
Publikum, welches nicht nur zuhört sondern auch zuschaut und jede Bewegung
beobachtet. Insofern kommuniziert der/die Musiker*in mit dem Publikum, selbst
wenn er/sie das Publikum im Vornehinein mit keiner Ansprache direkt adressiert.
Der/die Musiker*in ist der Sender von Signalen, während das Publikum wiederum
die Signale empfängt und darauf reagiert indem das Publikum auch wieder Signale
zurücksendet. 

%Nochmal nachlesen: monologische und dialogische Kommunikation
%- Gelungene Kommunikation: "dialogisch Kommunizieren" Andreas Doerne S. 67
%  "Geben und Nehmen", sich austauschen, abwechseln so wie in einem guten
%  Gespräch. "Beide Modi in einem ausgewogenen Verhältnis zueinander stehen." S.
%  67

%- Aber selbst wenn Kommunikation monologisch stattfindet, ist in einer
%  Konzertsituation das Publikum auch noch anwesend, was wiederum auf den
%  Musiker abfärbt, weshalb \enquote{eine Art unterschwelliger Dialog} entsteht.
%  \autocite[66]{doerne umfassend musizieren}
%- noochmal nachlesen: Kommunikation mit Publikum (Andreas Doerne S. 66)



\subsection{Motivation}

\enquote{Aus meiner Sicht liegt das wichtigste Ergebnis in der Motivation ein
Instrument zu erlernen und damit auf dem Weg des eigenständigen Musizierens
weiterzuschreiten. Ein fast ebenso wesentlicher Erfolg ist gegeben, wenn die
Kinder eine ausgeprägt positive Einstellung zum gemeinsamen Musizieren lernen
und gewinnen.}\autocite[40]{ernst:die_zukunftsfaehige_musikschule} Das
gemeinsame Musizieren ist nämlich ein sehr großer Erfolgsfaktor, die
Instrumentalschüler*innen für die Musik zu begeistern und damit auch zum
dranbleiben zu motivieren. Viele Quellen bestätigen die These, dass von dem
Ensemblespiel die größte Motivation für ein dauerhaftes Üben und Musizieren
ausgeht. Losert schreibt zum Beispiel, dass \enquote{(…) das Zusammenspiel für
Instrumentalschüler eine der stärksten Quellen der Motivation.} sei.
\autocite[92]{losert:die_kunst_zu_unterrichten} Das liegt womöglich daran, dass
sich der/die Instrumentalschüler*in im Vergleich zu Anderen und teilweise auch
gleichaltrigen hört und einordnen lernen kann. Nicht nur die auditive Ebene ist
dort bedient, sondern auch die Haltungen der Instrumente und Spielpositionen von
den Instrumenten können abgeglichen werden, sowie der generelle Umgang der
Mitspieler*innen mit ihren Instrumenten. Außerdem kann sich der/die
Instrumentalschüler*in von seinen Mitspielern inspirieren und motivieren lassen,
indem er/sie eine Stelle genauso gut beherrschen will, wie sein/e Mitspieler*in.
Oder es beflügelt ihn/sie, wie ein/e Andere/r die Musik gestaltet, was ihn/sie
dann wieder auf neue eigene Ideen bringen kann, wie er/sie eine Stelle spielt.

Teilweise können Instrumentalschüler*innen im Ensemble Stücke spielen, die sie
alleine vielleicht noch nicht ganz meistern könnten. \enquote{Wenn jeder an
anderen Stellen Schwierigkeiten hat, ist ein Anderer wiederum an anderen Stellen
sicher im Notentext und kann an den Stellen für eine Unterstützung sorgen.}
\autocite[32]{losert:die_kunst_zu_unterrichten}
Da das Ensemble also auch dazu helfen kann den Einzelnen über schwierigere
Passagen hinweg zu tragen, erreicht der/die Schüler*in ggf. einen ganz neuen
Horizont und gelangt dadurch zu mehr Selbstbewusstsein. Auch fortgeschrittene
Schüler*innen können profitieren, indem sie entweder solistisch heraustreten
oder lernen können ihre Klang- und Spielvorstellung weiter zu geben. Dazu müssen
sie die eigene Vorstellung oftmals selbst noch einmal mehr schärfen. \enquote{In
guten Fällen reißt das Ensemble den Einzelnen mit, motiviert auch den Faulen,
schwingt sich gemeinsam zu größeren Leistungen auf, als es der Einzelne
erbringen könnte, stärkt und hält die Einzelnen.}
\autocite[94]{mitzscherlich:musikpsychologie} \enquote{Zudem können
gruppendynamische Prozesse zu einer individuellen Effektivitätssteigerung
führen.} \autocite{losert:die_kunst_zu_unterrichten} Vorbilder, das
Zusammentreffen und gemeinsame Hinarbeiten auf ein Ziel, können den/die
Einzelne/n durchaus motivieren, mit an einem Strang zu ziehen. Diese Faktoren
entfallen im Einzelunterricht und bei fehlender Motivation handeln die
Schüler*innen eventuell um der Lehrkraft einen Gefallen zu tun.
In einer Gruppe hingegen, kann man sich gegenseitig für etwas
begeistern und mitreißen.\enquote{Ein Begeisterungszusammenhang macht die
Agierenden zu gleichberechtigten Partnern einer Produktionsgemeinschaft.}
\autocite[198]{busch:grundwissen_instrumentalpaedagogik} Dieses motivierende
Erlebnis kann der Einzelunterricht schlichtweg nicht ersetzten. \enquote{Die
Erfahrung von Zugehörigkeit über das gemeinsame Musizieren wird für immer mehr
Menschen Motivation und wichtiger Teil der musikalischen Identität.}
\autocite[123]{mitzscherlich:musikpsychologie}

Eine gewisse Bringschuld – oder auch negativ betitelt der Gruppenzwang – kann eine/n
Einzelne/n aber auch dazu bringen, sich gut oder besser auf die Proben
vorzubereiten, da in jedem Einzelnen auch die Angst vor Zurückweisung stecken
kann. \enquote{Das Anschlussmotiv äußert sich im Bredürfnis, Beziehungen
einzugehen, im Wunsch nach Nähe zu eienr sozialen Gruppe, dem Hingezogensein zu
anderen Menschen und der Angst vor Zurückweisung.}
\autocite[120ff]{losert:die_kunst_zu_unterrichten} Aber gleichzeitig ist dieser
Gruppendruck auch ein Motor, der einen Spieler antreiben kann, sich für etwas
einzusetzen und Energbie in etwas zu investieren, die man in einer anderen Form
wieder zurück bekommt.






%(Aus der Praxis: Gemeinsames Lernen fördert nicht nur soziale Kompetenzen wie
%Kommunikations- und Kooperationsfähigkeiten, auch musikalische Fähigkeiten, wie
%beispielsweise rhythmische und intonatorische Genauigkeit, musiktheoretische
%Kenntnisse oder Gehörbildung, lassen sich gut im Gruppenunterricht anbahnen."
%(S. 32 Die Kunst des Unterrichtens))


\section{Bedeutung für die Lernenden}

Welche Unterrichtsform für welche/n Schüler*in besonders geeignet ist, ist ein
sehr individuelles Unterfangen. Während der/die eine Schüler*in die ganze
Aufmerksamkeit einer Lehrkraft genießt, kann sich der/die andere im
Einzelunterricht unwohl fühlen. Oftmals ändern sich die
Unterrichtsformen für die Schüler*innen allerdings auch, da es gängig ist in
Gruppenunterrichten zu starten. Im Endeffekt ist es aber im fortgeschrittenen
Stadium am besten, wenn die Instrumentalschüler*innen sowohl Einzelunterricht
haben, als auch in eine Musiziergruppe eingebunden sind. Der/die Schüler*in
erfährt im Ensemble und \enquote{[…] beim Musizieren soziale Resonanz […]}
\autocite[28]{mitzscherlich:musikpsychologie} Durch das Zusammenspiel mit der
Lehrkraft, im Gruppen- oder Ensembleunterricht entsteht eine Steigerung
musikalischer Kommunikation. \autocite[99]{mitzscherlich:musikpsychologie} Denn
genauso, wie sich die Individualität im Leben durch den Austausch und die
Bezugnahme mit anderen Menschen entwickelt, entwickelt sich auch die die
Individualität des Musizierens in Beziehung zu anderen. Der Unterschied zwischen
einem Gruppen- und Einzelunterricht ist außerdem, dass im Ensemblespiel
\enquote{der Fokus weniger auf das Lehren des Lehrers als auf das Verstehen und
Einüben des Schülers gerichtet} ist.
\autocite[31]{losert:die_kunst_zu_unterrichten} Dadurch kann sich die
Instrumentallehrkraft viel mehr zurücknehmen, als im Einzelunterricht, wo
dem/der Schüler*in oft nur erklärt wird, wie er/sie etwas zu tun hat um ihn/sie
dann damit nach Hause zum Üben, Umsetzen und Einstudieren zu schicken. Viele
Instrumentallehrer*innen meinen, dass sie dem/der Schüler*in in einer Stunde
möglichst viel Input solcher Art geben müssen. Meiner Meinung nach besteht der
bessere Unterricht für den/die Schüler*in aber darin, den Schüler*innen auch
innerhalb der Stunde die ersten Gehversuche mit einem Stück oder einer neuen
Technik zu gestatten. Rein verbal etwas zu erläutern ist nur der Anfang, während
es wirklich wertvoll für den/die Lernende/n ist, wenn er/sie ausprobiert
seinen/ihren eigenen Weg zu finden. Diesen Prozess kann die Lehrkraft wiederum
im Instrumentalunterricht wunderbar betreuen und dabei ggf. Hilfestellungen
leisten. Wenn man den/die Schüler*in selbst im Unterricht forschen lässt und die
Lehrkraft ihn/sie durch die richtigen Fragen und Anweisungen in eine gute
Richtung lenkt, kann der/die Schüler*in ungemein profitieren. Im Ensemble
passiert dieses Forschen des Schülers ganz von alleine, da der/die Lehrer*in
nicht auf jede/n Schüler*in eingehen kann. Gewisse Richtungen kann die Lehrkraft
zwar vorgeben, aber der/die Schüler*in muss selbstverantwortlich und forschend
mit seinem/ihrem Instrument umgehen. Außerdem ist im Ensemble der Vorteil, dass
er durch Imitation viel von seinen Mitspielern lernen kann. Durch diese
Eigeninitiative kommt der/die Schüler*in in eine Selbsttätigkeit, die für die
Motivation am Instrument von großer Bedeutung ist. Wie in 1.5 erwähnt, erhält er das Gefühl
von Zugehörigkeit und Zusammenklang beim Spielen im Ensemble, was ihn gut durch
unmotivierte Phasen tragen kann. Die Gruppe übernimmt also Aufgaben, die eine
Lehrkraft im Einzelunterricht nicht bieten kann.



\section{Bedeutung für die Lehrenden}

\enquote{Die Besonderheit in der instrumentalpädagogischen Arbeit mit Gruppen
resultiert aus der Tatsache, dass mehrere Lernende zeitgleich so unterrichtet
werden müssen, dass ein gemeinsamer Lernprozess stattfindet.}
\autocite[221]{busch:grundwissen_instrumentalpaedagogik} Die Herausforderung
besteht also darin, allen Lernenden weiterzuhelfen, auch wenn jede/r Einzelne
mit individuellen Schwierigkeiten zu kämpfen hat. Der Vorteil in der Gruppe ist
auch, dass sich die Lernenden untereinander Lösungsmöglichkeiten abschauen und
auch gegenseitig helfen können. Je nach Gruppengröße muss die Lehrkraft
verschiedene Aufgaben übernehmen, zu der auch eine andere
Vorbereitungsnotwendigkeit als für den Einzelunterricht hinzukommt. Während die
Lehrkraft im Einzelunterricht unmittelbar auf die Impulse des Schülers eingehen
kann, muss die Lehrkraft bei Kleingruppen neben klaren Vorgaben die
Schüler*innen zu einem von- und miteinander Lernen anregen und sich selbst
zurücknehmen. Im Großgruppenunterricht ist es essentiell, dass die Lehrkraft im
Vorfeld eine Struktur und Stücke festlegt, um die Inhalte gut anleiten zu
können.\autocite[220]{busch:grundwissen_instrumentalpaedagogik} Meistens hat man
in dieser Konstellation die Situation eines Frontalunterrichts, was beim
Klassenmusizieren von den Instrumentalpädagog*innen verlangt wird, während die
Instrumentalpädagog*innen mit diesem Szenario oftmals nicht vertraut sind. Bei
sehr großen Gruppen, wie beispielweise im Klassenmusizieren ist es allerdings
auch keine Seltenheit, dass zwei oder mehrere Lehrende in unterschiedlichen
Konstellationen zusammenarbeiten. Dies bringt auch wieder neue Herausforderungen
mit sich, weil sich die Lehrenden gut untereinander absprechen und auch
gegenseitig aufeinander eingehen und flexibel reagieren müssen. Dafür sind klare
Absprachen im Voraus sehr dienlich. So kann zum einen vereinbart werden, dass
die eine Lehrkraft den Unterricht leitet, während die andere beobachtet
und assistiert. Es gibt aber auch die Möglichkeit, dass die Lehrenden sich
unterinander abwechseln und je nach inhaltlichem Schwerpunkt ablösen, sodass
jede/r die Aufgabe übernehmen kann, zu der er/sie einen besonderen Zugang hat.
Es gibt eine dritte Möglichkeit die gesamte Gruppe simultan zu unterrichten und
die vierte Möglichkeit die Gruppe der Lernenden aufzuteilen und im selben oder
in getrennten Räumen parallel zu unterrichten. Für die Instrumentallehrkraft
stellt Gruppen- oder Ensembleunterricht eine höhere Herausforderung dar, da
er/sie seine/ihre Aufmerksamkeit (gerecht) und flexibel verteilen,
differenziertere Spielanweisungen geben und eine musikalisch und menschlich
komplexere Situtation steuern muss. Andereseits ergeben sich im Umgang mit der
Gruppendynamik auch andere Möglichkeiten der Motivation und Förderung
musikalischen Lernens. Wie eine Lehrperson mit dieser Herausforderung besser
zurechtkommen kann, beschreiben Barbaraba Busch und Barbara Metzger sehr
zutreffend und umfassend in dem Kapitel \emph{Leitgedanken für die Arbeit
mit Gruppen} in dem Buch \emph{Grundwissen Instrumentalpädagogik}. Hierbei
spielen drei übergeordnete Ebenen eine zentrale Rolle: die Kommunikation, die
Raumnutzung und die Aufgabenstellung. Damit die Gruppendynamik zu einer guten
Lernathmosphäre beiträgt, ist wichtig, dass das soziale Miteinander
\emph{weitegehend störungsfrei} verläuft. Deswegen ist der erste zentrale Aspekt
die Kommunikation. Für die förderliche Lernathmosphäre hilft es, klare Regeln
einzuführen, die wiederum für die Lernenden zu Routinen werden und dadurch eine
gute Struktur in den Unterricht hineinbringen. Der Vorteil ist auch hier, dass
sich damit einige Schritte verselbstständigen, sodass die Lehrkraft nicht alles
ansagen muss. So will die Lehrkraft ja auch zwischen den Aktionsformen
variieren, damit die Schüler*innen bei der Sache bleiben. Die weiteren
beschriebenen Kommunikationsfelder betreffen die Lehrperson als solche, die ihre
Körpersprache bewusst einsetzen sollte, das Spektrum der Sprechstimme in
Lautstärke und Klangfarbe variieren und ausschöpfen kann und sich durchaus auch
die Einzelansprache eines Schüler*in zu nutze machen darf, trotz des
Gruppengefüges. Ob es für ein Lob oder eine Zurechtweisung ist, spielt dabei
keine Rolle. Wenn wir uns einmal die Situation vor Augen führen, dass eine
Lehrperson eine/n Schüler*in besonders loben will, so kann die Lehrkraft dieses
verbalisierte Lob durch ihre Körpersprache untermauern, indem er sich dem/der
Schüler*in zuwendet und sie/ihn direkt anblickt. Außerdem kann die Lehrkraft
ihn/sie auch bei ihrem/seinem direkten Namen nennen, sodass sich der/die
Schüler*in direkt angesprochen und wahrgenommen fühlt. Wenn die Lehrkraft jetzt
noch die Lautstärke der Stimme variiert und beispielsweise leiser spricht, führt
dies in der Gruppe wiederum möglicherweise zu einer verringerten
Gruppenlautstärke, da die anderen auch mitbekommen wollen, was die Lehrkraft
gerade gesagt hat. 

Als Lehrperson kann man den Unterrichtsraum unterschiedlich nutzten. So kann zum
Beispiel mit Blick zur Wand geschrieben, mit Bilck zum Fenster musiziert werden
und mit Blick zur Tür Bewegung stattfinden. Durch diese Variationen wird Schwung
in den Unterricht gebracht. Ein wichtiger Aspekt der Blickwinkel ist das
gegenseitige Wahrnehmen. Wichtig ist, dass sich alle Beteiligten gut sehen und
hören können, weshalb sich beim Gruppenmusizieren Kreise und Halbkreise
anbieten. Durch initiierte Bewegungsanlässe kann die Unterrichtseinheit außerdem
dynamisch gehalten werden. Durch einen Ortswechsel innerhalb eines Raumes kann
die gelegentlich sinkenden Aufmerksamkeit wieder angekurbelt werden. 

Der dritte Leitgedanke für die Arbeit mit Gruppen bezieht sich auf die
Aufgabenstellung. Um Missverständnisse zu vermeiden ist es zunächst einmal
wichtig, die Aufgaben so präzise wie möglich zu formulieren. Dies bedeutet
nicht, dass die Lehrperson implizit schon alles vorweg nimmt. Es darf und soll
im Unterrichtsgeschehen auch Explorationsphasen geben, in denen die
Schüler*innen zunächst einmal unkoordiniert etwas selbst entdecken und für sich
herausfinden sollen. Diese Phasen sind im Gegenteil auch sehr wichtig für einen
abwechslungsreichen Unterricht, da die Schüler*innen ihre persönlichen
Erfahrungen sammeln können und dadurch die intrinsische Motivation stärker
angekurbelt wird. Dies führt auch zu dem nächsten Punkt, dass die Schüler*innen
permanent beschäftigt und zu einer konkreten Aufgabe angeleitet sein sollten.
Wie man die Aufgaben innerhalb einer Gruppe verteilt kann aber wiederum
variieren. Es können alle gleichzeitig das Selbe tun, genauso gut können die
Schüler*innen nacheinander aber im gleichen Metrum etwas spielen, oder jede/r
Schüler*in bekommt seine/ihre eigene Aufgabe in einem kammermusikalischen
Unterfangen. Zu dieser Beschäftigung kann auch zählen, einen bestimmten Aspekt
bei einem anderen Schüler oder in einem Musikstück zu beobachten. Auch wenn dies
kein aktives Ausführen von einer musikalischen Aktion ist. Außerdem gibt es die
Möglichkeit der Binnendifferenzierung innheralb einer Gruppe, die besagt, dass
man die Gruppe unterteilt und ihnen unterschiedlichen Aufgaben aufträgt.
Allerdings sollte vermieden werden, dass kein paralleler Einzelunterricht
innerhalb des Gruppenunterrichts entsteht. In der Unterteilung haben wir in
größeren Gruppen einen großen Spielraum, den man sich wunderbar zu Nutze machen
kann. So kann man mit allen zusammen arbeiten, man kann die Gruppe in zwei
Hälften teilen, oder Partnerarbeit bis hin zur Einzelarbeit durchführen. Diese
vielen Aktionsformen zu variieren ist ein guter Wegweiser. Wie auch im
Einzelunterricht gilt im Gruppenunterricht die Eigeninitiative der Lernenden zu
stärken, was auch durch eine aktive Mitbestimmung der Lernenden geschehen kann.
In welchen Momenten die Lehrkraft diese Mitbestimmung zulässt, liegt in ihrem
eigenen Ermessen. Ähnlich ist es mit dem Umgang von Nebenschauplätzen: wenn es
sinnvoll scheint, kann man durchaus auf sie eingehen und sie ggf. sogar mit in
den Unterricht einbeziehen. In Bezug auf die musikspezifischen Parameter, die
einer Lehrperson zur Verfügung stehen, beschreiben Barbara Busch und Barbara
Metzger außerdem drei weitere Handlungsfelder. Der erste Aspekt gilt der
unterschiedlichen Umgangsweise mit Musik wie der Rezeption, Reproduktion,
Produktion und Transformation.\autocite{busch:grundwissen_instrumentalpaedagogik} Bedient sich eine Lehrkraft dieser
unterschiedlichen Felder, erleben die Lernenden ganz unterschiedliche
Erfahrungsfleder in der Musik. Der zweite Aspekt geht fließend in diese
Erfahrungsfelder über, nämlich der Ausdrucksformen die wir am Instrument haben.
Zu ihnen gehört neben dem Spiel am Instrument auch das Bewegen zur und mit der
Musik und das Abbilden in jeglicher kreativen Form. Der dritte Aspekt gilt dem
bewussten Hören und der Schulung der inneren Klangvorstellung, die gezielt
angeregt werden sollen, da sie für jeden Instrumentalisten von großer Bedeutung
sind.

%- (differenzielles Lernen) "Von einem bestimmen Alter an ist den
%  Heranwachsenden die Rückmeldung der Gleichaltrigen sowieso wichtiger als die
%  des Lehrers, dann bewirtk die Kritik eines Pultnachbarn manchmal eher eine
%  Veränderung oder eine intensivere Auseinandersetzung mit einem Stück als die
%  des Dirigenten "Da vorn." S. 95


%Evtl. ins Fazit? ((Die Musikhochschule müssten in diesem Punkt noch etwas
%besser ausbilden, da jeder Instrumentalpädagoge darauf vorbereitet sein sollte
%und es so wichtig ist. S. 94 oberer Abschnitt ))

%"(…) oft ziemlich eng ausgebildeten Instrumentalpädagogen(…) S. 119


%"Generell gilt, dass Unterrichtsformen, in denen der Lehrer mehr zurücktritt
%und die Schüler eine Aufgabe alleine bearbeiten lässt, höchst fruchtbar für das
%Unterrichtsgeschehen sein können. ( In solchen Phasen profitieren die Schüler
%stark voneinander, sie erklären und demonstrieren sich gegenseitig bzw.
%erarbeiten eigenständig einen Unterrichtsgegenstand. ) 
