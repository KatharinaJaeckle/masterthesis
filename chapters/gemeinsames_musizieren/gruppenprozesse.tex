\section{Gruppenprozesse}


- Individualität in der Gruppe 
"(...)Individualität zeigt sich gerade auch im Umgang mit anderen,
Individualität entwickelt sich im Vergleich zu und in ABgrenzung von anderen
oder auch dadurch, dass Positionen in einer Gruppe besetzt werden."

- unterschiedliche Alters- und Lernniveaus (manchmal) evtl. S. 95

- Lernen / Differenzielles Lernen? 
"Ein Großteil musikalischen Lernens findet über Imitation statt (...) S. 98

- Hierarchie
fortgeschrittenere Schüler zu Mentoren erklären und sie damit mitverantwortlich
für deren musikalische Entwicklung und den Fortschritt des Ensembles zu machen.
S. 95

- Altersgruppen/Erwachsene
"Wiedererlangen sozialer Integration scheint ein wesentliches Motiv
musikalischer Betätigung bei Erwachsenen (...). " s. 118

- "Selbsttätigkeit" (die Kunst zu unterrichten)
"Erst wenn es eine Notwendigkeit für einen Schüler gebe, eine Arbeit wirklich
selbst zu vollbringen, würde er ganzheitlich mit seinen Gefühlen, Trieben und
Willenskraft in die Tätigkeit involviert. (S. 63 bzw Eigl von Kerschensteiner)

-Gruppe als Korrektiv

\subsection*{Bedeutung für den Schüler*in}

Welche Unterrichtsform für welchen Schüler besonders geeignet ist, ist ein sehr
individuelles Unterfangen. Während der eine Schüler die ganze Aufmerksamkeit
eines Lehrers genießt, kann sich der andere im Einzelunterricht unwohl fühlen.
Wer ein ser kommunikativer Mensch ist, ist hingegen vermutlich in den
Gruppenunterrichten besser aufgehoben. Oftmals ändern sich die Unterrichtsformen
für einen Schüler allerdings auch, da es gängig ist in Gruppenunterrichten zu
starten. Die Form eines Kleingruppenunterrichts bei Anfängern bietet sich auch
an, siehe Zitat \autocite[220]{Busch:grundwissen_Instrumentalpädagogik}
Im Endeffekt ist es aber im fortgeschrittenen Stadium am besten, wenn die
Instrumentalschüler sowohl Einzelunterricht haben, als auch in eine
Musiziergruppe eingebunden sind. 

"Man erfährt beim Musizieren soziale Resonanz(...)" S.28

"Eine Steigerung musikalischer Kommunikation erfährt der Schüler im
Zusammenspiel mit dem Lehrer, im Gruppen- oder Ensembleunterricht. (...) Die Individualität des eigenen Musizierens entwickelt sich in
Beziehung zu
anderen."  S.99 Punkt 5

"Das Gefühl von Zugehörigkeit und Zusammenklang beim Spielen im Ensemble."


\subsection*{Bedeutung für den Lehrer}

"Die Besonderheiten in der instrumentalpädagogischen Arbeit mit Gruppen
resultiert aus der Tatsache, dass mehrere Lernende zeitlgleich so unterrichtet
werden müssen, dass ein gemeinsamer Lernprozess stattfindet."
\autocite[221]{Busch:grundwissen_Instrumentalpädagogik} Die Herausforderung
besteht also darin, allen Lernenden weiterzuhelfen, auch wenn jeder Einzelne mit
individuellen Schwierigkeiten zu kämpfen hat. (Der Vorteil in der Gruppe ist aber
auch, dass sich die Lernenden untereinander Lösungsmöglichkeiten abschauen nud
auch gegenseitig helfen können.) 
Je nach Gruppengröße muss der Lehrer verschiedene Aufgaben übernehmen,
zu der
auch eine unterschiedliche Vorbereitungsnotwendigkeit hinzukommt. Während der
der Lehrende im Einzelunterricht unmittelbar auf die Impulse des Schülers
eingehen kann, so muss der Lehrende bei Kleingruppen neben klaren Vorgaben die
Schüler zu einem von- und miteinander Lernen anregen und sich selbst auch einmal
zurücknehmen. Im Großgruppenunterricht ist es essentiell, dass der Lehrer im
Vorfeld eine Struktur und Stücke festlegt um die Inhalte gut anleiten zu können.\autocite[220]{Busch:grundwissen_Instrumentalpädagogik}
Meistens hat man in dieser Konstellation die Situation eines Frontalunterrichts,
was beim Klassenmusizieren von den Instrumentalpädagogen verlangt wird, während die
Instrumentalpädagogen mit diesem Szenario oftmals nicht vertraut sind. Bei sehr
großen Gruppen, wie beispielweise im
Klassenmusizieren ist es allerdings auch keine Seltenheit, dass zwei oder
mehrere Lehrende in unterschiedlichen Konstellationen zusammenarbeiten. Dies
bringt auch wieder neue Herausforderungen mit sich, so müssen sich die Lehrenden
gut untereinander absprechen und auch gegenseitig aufeinander eingehen und
flexibel reagieren können. Dafür sind klare Absprachen im Voraus sehr dienlich,
so kann zum einen vereinbart werden, dass der eine Lehrer den Unterricht leitet,
während der andere beobachtet und assistiert. Es gibt aber auch die Möglichkeit,
dass die Lehrenden sich unterinander abwechseln und je nach inhaltlichem
Schwerpunkt ablösen, sodass jeder das Gebiet übernehmen kann, in dem er einen
besonderen Zugang hat. Es gibt eine dritte Möglichkeit die gesamte Gruppe
simultan zu unterrichten und die vierte Möglichkeit die Gruppe der Lernenden
aufzuteilen und im selben oder in getrennten Räumen parallel zu unterrichten.
\autocite{cook s. 461} 

"Für den Instrumentallehrer stellt Gruppen- oder Ensembleunterricht eine höhere
Herausforderung dar, da er seine Aufmerksamkeit (gerecht) und flexibel
verteilen, differenziertere Spielanweisungen geben und eine musikalisch und
menschlich komplexer Situtation steuern muss. Andereseits ergeben sich im Umgang
mit der Gruppendynamik auch andere Möglichkeiten der Motivation und Förderung
musikalischen Lernens." S. 99 Punkt 6 (doerne umfassend musizieren?)
Wie eine Lehrperson mit dieser Herausforderung besser zurechtkommen kann,
beschreiben Barbaraba Busch und Barbara Metzger sehr zutreffend und auch
umfassend in dem Kapitel "Leitgedanken für die Arbeit mit Gruppen" in dem Buch
Grundwissen Instrumentalpädagogik. Hierbei stehen spielen drei übergeordnete
Ebenen eine zentrale Rolle, zu der die Kommunikation, die Raumnutzung und die
Aufgabenstellung gehören. 
Damit die Gruppendynamik zu einer guten Lernathmosphäre beiträgt, ist wichtig,
dass das soziale Miteinander "weitegehend störungsfrei" verläuft. Deswegen ist
der erste zentrale Aspekt die Kommunikation. Für die fördernliche
Lernathmosphäre hilft es, klare Regeln einzuführen, die wiederum für die
Lernenden zu Routinen werden und dadurch eine gute Struktur in den Unterricht
hineinbringen. Der Vorteil ist auch hier, dass sich damit einige Schritte
verselbstständigen, sodass der Lehrer nicht alles ansagen muss. So will der
Lehrende jaa auch zwischen den sogenannten Aktionsformen variieren, damit die
Schüler bei der Sache bleiben. Die weiteren beschriebenen Kommunikationsfelder
betreffen die Lehrperson als solche, die ihre Körpersprache bewusst einsetzen
sollte, das Spektrum der Sprechstimme in Lautstärke und Klangfarbe variieren und
ausschöpfen kann und sich durchaus auch die Einzelansprache eines Schüler zu
nutze machen darf, trotz des Gruppengefüges. Ob es für ein Lob oder eine
Zurechtweisung ist, spielt dabei keine Rolle. Wenn wir uns einmal die Situation
vor Augen führen, dass eine Lehrperson einen Schüler besonders loben will, so
kann der Lehrende dieses verbalisierte Lob durch seine Körpersprache
untermauern, indem er sich dem Schüler zuwendet und ihn direkt anblickt.
Außerdem kann er ihn auch direkt bei seinem direkten Namen nennen, sodass sich
der Schüler direkt angesprochen und wahrgenommen fühlt. Wenn der Lehrer jetzt
noch die Lautstärke der Stimme variiert und beispielsweise leiser spricht, führt
dies in der Gruppe wiederum möglicherweise zu einer verringerten
Gruppenlaustärke, da die anderen auch mitbekommen wollen, was die Lehrkraft
gerade gesagt hat. 


- (differenzielles Lernen)
"Von einem bestimmen Alter an ist den Heranwachsenden die Rückmeldung der
Gleichaltrigen sowieso wichtiger als die des Lehrers, dann bewirtk die Kritik
eines Pultnachbarn manchmal eher eine Veränderung oder eine intensivere
Auseinandersetzung mit einem Stück als die des Dirigenten "Da vorn." S. 95


Evtl. ins Fazit? ((Die Musikhochschule müssten in diesem Punkt noch etwas besser ausbilden, da
jeder Instrumentalpädagoge darauf vorbereitet sein sollte und es so wichtig ist.
S. 94 oberer Abschnitt ))

"(...) oft ziemlich eng ausgebildeten Instrumentalpädagogen(...) S. 119
