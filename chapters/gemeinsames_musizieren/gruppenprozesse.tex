\section{Gruppenprozesse}
"Gemeinsam musizieren Lernen" setzt voraus, dass wir eine Gruppe von
Instrumentalisten beisammen haben. Die Gruppe kann aber ganz unterschiedlich
zusammengesetzt sein. Sie kann unterschiedliche Alters- und Lernniveaus haben,
sie kann sich aus verschieden Instrumenten zusammensetzen oder auch
ausschließlich aus ein und demselben. Eine Gruppe kann auch homogen nur aus
Anfängern, oder Profi-Musikern bestehen, sie kann genreübergreifend arbeiten,
improvisatorisch, oder nur mit Noten. Die Teilnehmer können freiweillig dabei
sein, sie können verpflichtet worden sein oder mit dem Ensemble ihren
Lebensunterhalt verdienen, womit sich das Freiwillige mit dem Notwendigen
verbindet. 
Bei Erwachsenen Laienmusikern scheint ein
wesentliches Motiv der musikalischen Betätigung das Wiedererlangen der sozialen
Integration zu sein. (S.118 andreas doerne?) Außerdem kann die Größe der Gruppe sehr stark
variieren. Wenn man von dem gemeinsamen Musizieren spricht, fängt dieses beim
Musizieren zwischen Lehrer und dem Schüler an. Die Definition einer "Gruppe"
wiederum gilt "ab 3 Personen, deren Mitglieder sich über einen längeren Zeitraum
in regelmäßigem Kontakt miteinander befinden, gemeinsame Ziele verfolgen und
sich als zusammengehörig empfinden."
\autocite{wikipedia:gruppe}
Des Weiteren kommt man bei der Gruppenzusammensetzung sehr schnell auf das Thema der
Hierarchie und wie eine Gruppe arbeitet. Wird sie von Instrumentallehrern
angeleitet oder von einem Dirigenten? Oder ist die Gruppe selbstbestimmt, wie das
junge Orchester "Stegreif"? Außerdem ist es interessant zu betrachten, wie die
Teilnehmer innerhalb der Gruppe interagieren.
So vielfältig wie sich eine Gruppe zusammensetzen und sich eine Gruppendynamik
entwickeln kann, sind auch die Anforderungen an die Instrumentalpädagogen.


\subsection{Aktionsformen in der Gruppe}
"Aus einer zweipoligen Interaktionsform im Einzelunterricht ergeben sich
mehrpolige Interaktionsformen im Gruppenuntericht, was Unterrichtssituationen
für gewöhnlich komplexer erscheinen lässt." \autocite[30]{losert:die_kunst_zu_unterrichten}
Wie wir aber später sehen werden, sind in manchen Systemen für die Gruppe auch
mehrere Lehrer vorhergesehen. Nichts desto trotz bedarf es aber auch bei den
Lehrern untereinander eine präzise Absprache der Rollenverteilung und den
Gesamtüberblick, den die Lehrende immer haben sollte. 
Je nachdem wie die räumliche Situation es erlaubt können sogar
verschiedene
Spielräume für verschiedene Aktionsformen genutzt und die Gruppe räumlich
aufgeteilt werden. So
kann es beispielsweise Phasen geben, in denen jeder für sich tätig ist,
Partnerarbeit, kleinere Gruppenunterteilungen oder die
Arbeit in der Gruppe als Ganze. In der Elementaren Musikpädagogik basieren alle
Formen auf dem beziehungsorientierten
Arbeiten, in dem \enquote{über die Beziehungen, die zur Musik aufgebaut werden sollen,
[...] auch die Kontakte mit anderen Gruppenmitgliedern im Fokus} 
\autocite[10]{dartsch:kern_des_musizierens} stehen. Mit dem ganzheitlichen Menschen im
Blick, soll neben der angestrebten Persönlichkeitsbildung die soziale
Kommunikation ausgebaut werden.

\subsection{Psychosoziales Lernen in der Gruppe/Persönlichkeitsbildung}
Sich als Individuum und gleichzeitig auch als Teil der Gruppe zu erleben schult
die Persönlichkeitsentwicklung ungemein, da das Leben in der Kollektivität und
das Leben mit uns selbst, also der Individualität ein existentielles
Wechselspiel darstellt. Dieses Wechselspiel erleben wir nicht nur in der Gesellschaft,
sondern auch in der Musik. Die Individualität in der Gruppe 
"(...) zeigt sich gerade auch im Umgang mit anderen,
Individualität entwickelt sich im Vergleich zu und in Abgrenzung von anderen
oder auch dadurch, dass Positionen in einer Gruppe besetzt werden."
%andreas,s.95? 
Das kennt jeder aus seinen eigenen Rollenbildern innerhalb einer Familie oder
Verwandschaft. Die psychischen Aspekte, welche alle die eigene Persönlichkeit
betreffen sind solche wie die emotionale, intuitive aber auch rationale,
intelektuelle Ebenen die im Wechselspiel zu den sozialen Aspekten
und damit allen Faktoren und Einflüssen der Umwelt stehen. Die
Persönlichkeitsentwicklung geschieht dadurch, dass der Mensch
seine gesammelten Ich-Informationen auf verschiedensten Ebenen mit der sozialen
Umwelt in Bezug setzen muss und diese dann wieder für sich selbst auswertet.
Gleichzeitig ist die zwischenmenschliche soziale
Interaktion der Grundbaustein für musikalische Prozesse! %(Quelle noch aus Bib)
Dies gilt besonders für die Elementare Musikpädagogik, wo die musikalischen
Prozesse bis zu einem gewissen Grad noch sehr offen gehalten sind.
Die Gruppe ist automatisch immer ein Korrektiv, da sich keine einzelnen
Meinungen gegenüberstehen, sondern die Gruppe als Kollektiv ihre Richtlinien
vorgibt. 
Die Entwicklung der Persönlichkeit sollte außerdem auch ein wichtiges
Erziehungsziel von Instrumentalpädagogen sein, da die Schüler in
Instrumentalgruppen auch die Chance haben, die Ordnung der Gemeinschaft noch
einmal anders als in der Schule kennen
zu lernen. \enquote{Dazu müssten sie zu Eigenaktivität, Selbsttätigkeit und konsequenter
eigenständiger Arbeit hingeführt werden.}\autocite[64]{losert:die_kunst_zu_unterrichten}
Betrachtet man einmal den Effekt des differenziellen Lernens, nachdem die
Schüler durch ihre eigene Wahrnehmung und Interessen das aus dem Unterricht
herausfiltern, was für sie interessant ist, so darf man nicht außer Acht lassen,
dass das Lernen der Schüler untereinander eine sehr zentrale Rolle spielt. 
\enquote{Ein Großteil musikalischen Lernens findet über Imitation statt (...).}\autocite[98]{doerne:umfassend_musizieren}

\subsection{Hierarchie}
Eng mit den Rollenbildern verknüpft ist auch die Hierarchie, die beim Lernen der Musizierenden untereinander eine
wichtige Rolle spielt. So gibt es oft diejenigen, die mehr zu sagen haben und
diejenigen auf die man weniger hört. In manchen Ensembles haben bestimmte
Personen einen höheren Rang als Andere und nicht zuletzt ist die Frage ob die
Gruppe musikalisch angeleitet wird, oder selbstbestimmt demokratisch alle
Entscheidungen trifft. Wenn die Gruppe von einem Dirigenten oder auch Lehrer
angeleitet wird, gibt es diejenigen Personen, die in ihrem Auftreten authentisch
sind, aber leider auch manchmal diejenigen, die es nicht sind. Für die
Instrumentalpädagogen, die nach dem heutigen System ausgebildet werden, spielt
diese Qualifikation gar keine Rolle und wird leider auch zu wenig geschult. Denn
sie sollte eine wichtige Rolle spielen, sodass der Lehrer der vor einer Gruppe
steht auch mit dieser umzugehen weiß. Dazu später mehr. Die "natürliche Hierarchie"
innerhalb des Ensembles kann man sich auch positiv zu Nutze machen, indem man beispielsweise
fortgeschrittenere Schüler zu Mentoren erklärt und sie
damit
mitverantwortlich sind 
für die musikalische Entwicklung und den Fortschritt des Ensembles.\autocite[95]{doerne:umfassend_musizieren}


\subsection{Kommunikation}
Die Kommunikation spielt in Unterrichtssituationen, sowie auch in Gruppen, die
gemeinsam Musizieren die zentrale Rolle. Der Instrumentalunterricht stellt ein komplexes Geschehen dar, der
durch
kommunikative Prozesse gekennzeichnet ist. %(Grundwissen IP - Kapitel 4 S. 193)
Unter Kommunikation verstehe ich die Art und Weise wie sich Lehrer und Schüler
begegnen und miteinander umgehen, das kann im musikalischen Miteinander oder
auch außerhalb des Spielens sein, sie ist allgegenwärtig. 
"Sobald der Mensch musiziert, betritt er einen kommunikativen Raum. Der
Musizierende komminuziert mittels Musik, mit der Musik, mit sienem Instrument,
mit dem Komponisten, mit sich selbst, mit seinen Mitmusikern, mit dem Publikum
und über das Publikum mit der Gesellschaft, in der er lebt." \autocite[56]{doerne:umfassend_musizieren}
Diese ganzen Ebenen werden durch das Musizieren erreicht und sind gleichzeitig
solch eine Wucht an Informationen, die alle gleichzeitig von dem Einzelnen
einzuordnen sind, dass das Musizieren selbst nur noch einen kleinen Teil
ausmacht. Aber genau dieses Jonglieren und Einordnen von Informationen, die
jeder Einzelne wahrnimmt und gleichzeitig das Abrufen von seinen instrumentalen
Fertigkeiten bedarf einer hohen Konzentration. Es schult den Menschen im Umgang
mit seinen Mitmenschen ungemein und über die Musik hinaus. Eine weitere Ebene,
die durch die Musik ins Spiel gebracht wird und eng mit dem Thema der
Identität verknüpftf ist, ist die kulturelle Ebene. Abgesehen von den
offensichtlich unterschiedlichen Kulturen, die durch verschiedene Länder
zusammen kommen, gibt es auch innerhalb eines Landes durch die Bandbreite
verschiedener Genres sehr verschiedene Beheimatungen. 
"Musik ist Bestandteil des soziokulturellen Raumes, der entsteht, wenn mehrere
Menschen miteinander in Kontakt treten. Als solche kann sie vielerlei
kommunikative Funktionen erfüllen." \autocite[56]{doerne:umfassend_musizieren} Der soziokulturelle
Raum entsteht durch die unterschiedlichen Identitäten und die Musik ist das
kommunikative Moment, das die Musizierenden universell verbindet, egal wie
unterschieldlich die Prägungen sind. Betrachten wir nicht nur die Gruppe sondern
auch nochmal den Vorgang innerhalb eines Musizierenden. Durch seinen soziokulturellen Hintergrund
formt der Musizierende die Musik nämlich auf seine Weise, während aber die Musik
wiederum auch auf den Musiker einwirkt. Andreas Doerne beschreibt dieses Vorgang sehr
zutreffend:
"Der Musizierende prägt die Musik genauso wie die Musik den Musizierenden prägt.
Angestoßen durch Kommunikation, beginnen beide Welten - die des Musikers und die
der Musik - sich zu verbinden, indem sie zu einer gemeinsamen Bewegung, einem
gemeinsamen Atmen, einem gemeinsamen Fühlen und Denken verschmelzen." \autocite[60]{doerne:umfassend_musizieren}
Bleiben wir bei dem einzelnen Musizierenden, so kommt noch die
Kommunikationseben mit dem Musizierenden und seinem Instrument hinzu. Hierbei
spielt es sicherlich eine Rolle, ob der Musizierende sein eigenes Instrument
besitzt, wie beispielsweise ein Bläser, Streicher o.Ä., während die Pianisten
weniger die Möglichkeit haben mit ihrem konkreten eigenen Instrument zusammen zu
wachsen. Dennoch tritt der Musiker mit dem bespielten Instrumenten während des
Spielens in eine enge Verbindung und treten dabei in eine Art gleichberechtigten
Austausch, sodass beim Musizierenden der Eindruck entstehen kann, "dass er selbst wie auch das Instrument zugleich Sender und
Empfänger von Botschaften ist." \autocite[59]{doerne:umfassend_musizieren}
Bei Sängern, deren Instrument die eigene Stimme ist, verhält es sich vermutlich
ähnlich, wobei hier die engste persönliche Beziehungsebene zwischen dem
"Instrument" also der Stimme und dem Musiker selbst besteht. 
Eine weitere Kommunikationsebene ist die der zwischen dem Musizierenden und dem
Werk, das er spielt, was auch wiederum mit dem Genre und damit wieder mit der
kuklturellen Ebene eng zusammenhängt.
"Setzt sich ein Musiker mit einem Musikwerk auseinander, beschäftigt er sich
gleichzeitig immer auch mit der Körper-, Gefühls-, und Gedankenwelt des
Komponsiten." \autocite[59]{doerne:umfassend_musizieren} Indem er ein Stück
interpretiert und sich überlegt, welche Stellen was aussagen und welche
Emotionen verkörpern oder musikalisch ausdrücken sollen, bringt ihn damit sehr
eng in Verbindung mit dem Komponisten. Am plakativsten sind Opernrollen, in denen die
Sänger durch den Text gestützt eine ganz klar vorgegebene Rolle erfüllen. Bei
der Improvisation hingegen lässt der Musizierende seinen eigenen Gefühlen und
seiner Intuition freien Lauf. Durch die Kommunikation mit Mitspielern hingegen,
bringen die Mitspieler sich gegenseitig auch auf verschiedene musikalische
Ideen. Die Kommunikation kann hierbei dialogich oder aber auch monologisch
ablaufen. \autocite[61]{doerne:umfassend_musizieren}
Aber selbst wenn Kommunikation monologisch stattfindet, ist in einer Konzertsituation das
Publikum auch noch anwesend, was wiederum die monologische 


"Der Musizierende selbst als Ensemble" bestehend aus divergenten Teilpersönlichkeiten und
unterschiedlichen psychischen Dynamiken." (S. 64 Andreas Doerne)

"innere Team -schulz von thun" s. 65

Kommunikation mit Publikum Andreas Doerne S. 66

Gelungene Kommunikation: "dialogisch Kommunizieren" Andreas Doerne S. 67

"So sind Kommunikation und Persönlichkeitsbildung zwei Seiten der Medaille."
(Andreas 69/ Schul von Thun)



Aus der Praxis:
Gemeinsames Lernen fördert nicht nur soziale Kompetenzen wie Kommunikations- und
Kooperationsfähigkeiten, auch musikalische Fähigkeiten, wie beispielsweise
rhythmische und intonatorische Genauigkeit, musiktheoretische Kenntnisse oder
Gehörbildung, lassen sich gut im Gruppenunterricht anbahnen." (S. 32 Die Kunst
des Unterrichtens)


\subsection{Motivation}
"(...) ist das Zusammenspiel für Instrumentalschüler eine der stärksten Quellen
der Motivation." S. 92 Das liegt daran, dass es sich im Vergleich zu Anderen und
teilweise auch gleichaltrigen hört und einordnen lernen kann. Außerdem kann der Instrumentalschüler
sich von anderen inspirieren und motivieren lassen und teilweise Stücke spielen,
die er alleine vielleicht noch nicht ganz spielen könnte. Da das Ensemble aber
auch dazu helfen kann den Einzelnen über schwierigere Passagen hinweg zu tragen,
erreicht der Schüler ggf. einen ganz neuen Horizont und gelangt dadurch wieder
zu mehr Selbstbewusstsein. Aber auch fortgeschrittene Schüler können
profitieren, indem sie entweder solistisch heraustreten können aber Lernen
können ihre Klang- und Spielvorstellung weiter zu geben. Dazu müssen sie die
eigene oftmals auch selbst noch einmal mehr schärfen. "In guten Fällen reißt das
Ensemble den Einzelnen mit, motiviert auch den Faulen, schwingt sich gemeinsam
zu größeren Leistungen auf, als es der Einzelne erbringen könnte, stärkt undd
hält die Einzelnen." (S.94)

Thema Begeisterung: auch im Einzel- Instrumentalunterricht relevant aber vor allem in der Gruppe potenziert es
sich "Ein Begeisterungszusammenhang macht die Agierenden zu gleichberechtigten
Partnern einer Produktionsgemeinschaft." S. 198 Grundwissen
Instrumentalpädagogik


"(...) nach wie vor kann man Jugendliche durch sein eigenes Spiel, die erlebten
Wirkungen und des Zusammenspiels im Ensembles für Musik begeistern." S. 122


"Die Erfahrung von Zugehörigkeit über das gemeinsame Musizieren wird für immer
mehr Menschen Motivation und wichtiger Teil der musikalischen Identität.2 s. 123

"Das Anschlussmotiv äußert sich im BEdürfnis, BEziehungen einzugehen, im Wunsch
nach Nähe zu eienr soziialen Gruppe, dem HIngezogensein zu anderen Mneschen und
der Angst vor Zurückweisung." (S. 120-121 die Kunst zu Unterrichtern// Barbara Roth)

"Vom Ensemblespiel geht die größte Motivation für ein dauerhaftes Üben und
Musizieren aus. Wer in einem Ensemble 



- Gemeinsam Stücke spielen, die man sonst nicht spielen kann. Jeder hat an
anderen Stellen Schwierigkeiten und ist wiederum an anderen Stellen sicher im
Notentext und kann an den Stellen für eine Unterstützung sorgen. (S. 32 Die
Kunst zu unterrichten)

- "Selbsttätigkeit" (die Kunst zu unterrichten)
"Erst wenn es eine Notwendigkeit für einen Schüler gebe, eine Arbeit wirklich
selbst zu vollbringen, würde er ganzheitlich mit seinen Gefühlen, Trieben und
Willenskraft in die Tätigkeit involviert. (S. 63 bzw Eigl von Kerschensteiner)


"Zudem können gruppendynamische Prozesse zu einer individuellen
Effektivitätssteigerung führen." (S. 32 kunst zu unterrichten)

\section{Bedeutung für den Schüler}
Welche Unterrichtsform für welchen Schüler besonders geeignet ist, ist ein sehr
individuelles Unterfangen. Während der eine Schüler die ganze Aufmerksamkeit
eines Lehrers genießt, kann sich der andere im Einzelunterricht unwohl fühlen.
Wer ein ser kommunikativer Mensch ist, ist hingegen vermutlich in den
Gruppenunterrichten besser aufgehoben. Oftmals ändern sich die Unterrichtsformen
für einen Schüler allerdings auch, da es gängig ist in Gruppenunterrichten zu
starten. Die Form eines Kleingruppenunterrichts bei Anfängern bietet sich auch
an, siehe Zitat \autocite[220]{busch:grundwissen_instrumentalpaedagogik}
Im Endeffekt ist es aber im fortgeschrittenen Stadium am besten, wenn die
Instrumentalschüler sowohl Einzelunterricht haben, als auch in eine
Musiziergruppe eingebunden sind. 

"Man erfährt beim Musizieren soziale Resonanz(...)" S.28

"Eine Steigerung musikalischer Kommunikation erfährt der Schüler im
Zusammenspiel mit dem Lehrer, im Gruppen- oder Ensembleunterricht. (...) Die Individualität des eigenen Musizierens entwickelt sich in
Beziehung zu
anderen."  S.99 Punkt 5

"Das Gefühl von Zugehörigkeit und Zusammenklang beim Spielen im Ensemble."

Der Fokus ist weniger auf
das Lehren des Lehrers als auf das Verstehen und Einüben des Schülers
gerichtet."  (S.31 die kunst zu unterrichte)


\section{Bedeutung für den Lehrer}

"Die Besonderheiten in der instrumentalpädagogischen Arbeit mit Gruppen
resultiert aus der Tatsache, dass mehrere Lernende zeitlgleich so unterrichtet
werden müssen, dass ein gemeinsamer Lernprozess stattfindet."
\autocite[221]{busch:grundwissen_instrumentalpaedagogik} Die Herausforderung
besteht also darin, allen Lernenden weiterzuhelfen, auch wenn jeder Einzelne mit
individuellen Schwierigkeiten zu kämpfen hat. (Der Vorteil in der Gruppe ist aber
auch, dass sich die Lernenden untereinander Lösungsmöglichkeiten abschauen nud
auch gegenseitig helfen können.) 
Je nach Gruppengröße muss der Lehrer verschiedene Aufgaben übernehmen,
zu der
auch eine unterschiedliche Vorbereitungsnotwendigkeit hinzukommt. Während der
der Lehrende im Einzelunterricht unmittelbar auf die Impulse des Schülers
eingehen kann, so muss der Lehrende bei Kleingruppen neben klaren Vorgaben die
Schüler zu einem von- und miteinander Lernen anregen und sich selbst auch einmal
zurücknehmen. Im Großgruppenunterricht ist es essentiell, dass der Lehrer im
Vorfeld eine Struktur und Stücke festlegt um die Inhalte gut anleiten zu können.\autocite[220]{busch:grundwissen_instrumentalpaedagogik}
Meistens hat man in dieser Konstellation die Situation eines Frontalunterrichts,
was beim Klassenmusizieren von den Instrumentalpädagogen verlangt wird, während die
Instrumentalpädagogen mit diesem Szenario oftmals nicht vertraut sind. Bei sehr
großen Gruppen, wie beispielweise im
Klassenmusizieren ist es allerdings auch keine Seltenheit, dass zwei oder
mehrere Lehrende in unterschiedlichen Konstellationen zusammenarbeiten. Dies
bringt auch wieder neue Herausforderungen mit sich, so müssen sich die Lehrenden
gut untereinander absprechen und auch gegenseitig aufeinander eingehen und
flexibel reagieren können. Dafür sind klare Absprachen im Voraus sehr dienlich,
so kann zum einen vereinbart werden, dass der eine Lehrer den Unterricht leitet,
während der andere beobachtet und assistiert. Es gibt aber auch die Möglichkeit,
dass die Lehrenden sich unterinander abwechseln und je nach inhaltlichem
Schwerpunkt ablösen, sodass jeder das Gebiet übernehmen kann, in dem er einen
besonderen Zugang hat. Es gibt eine dritte Möglichkeit die gesamte Gruppe
simultan zu unterrichten und die vierte Möglichkeit die Gruppe der Lernenden
aufzuteilen und im selben oder in getrennten Räumen parallel zu unterrichten.
\autocite{cook s. 461} 

"Für den Instrumentallehrer stellt Gruppen- oder Ensembleunterricht eine höhere
Herausforderung dar, da er seine Aufmerksamkeit (gerecht) und flexibel
verteilen, differenziertere Spielanweisungen geben und eine musikalisch und
menschlich komplexer Situtation steuern muss. Andereseits ergeben sich im Umgang
mit der Gruppendynamik auch andere Möglichkeiten der Motivation und Förderung
musikalischen Lernens." S. 99 Punkt 6 (doerne umfassend musizieren?)
Wie eine Lehrperson mit dieser Herausforderung besser zurechtkommen kann,
beschreiben Barbaraba Busch und Barbara Metzger sehr zutreffend und auch
umfassend in dem Kapitel "Leitgedanken für die Arbeit mit Gruppen" in dem Buch
Grundwissen Instrumentalpädagogik. Hierbei stehen spielen drei übergeordnete
Ebenen eine zentrale Rolle, zu der die Kommunikation, die Raumnutzung und die
Aufgabenstellung gehören. 
Damit die Gruppendynamik zu einer guten Lernathmosphäre beiträgt, ist wichtig,
dass das soziale Miteinander "weitegehend störungsfrei" verläuft. Deswegen ist
der erste zentrale Aspekt die Kommunikation. Für die fördernliche
Lernathmosphäre hilft es, klare Regeln einzuführen, die wiederum für die
Lernenden zu Routinen werden und dadurch eine gute Struktur in den Unterricht
hineinbringen. Der Vorteil ist auch hier, dass sich damit einige Schritte
verselbstständigen, sodass der Lehrer nicht alles ansagen muss. So will der
Lehrende jaa auch zwischen den sogenannten Aktionsformen variieren, damit die
Schüler bei der Sache bleiben. Die weiteren beschriebenen Kommunikationsfelder
betreffen die Lehrperson als solche, die ihre Körpersprache bewusst einsetzen
sollte, das Spektrum der Sprechstimme in Lautstärke und Klangfarbe variieren und
ausschöpfen kann und sich durchaus auch die Einzelansprache eines Schüler zu
nutze machen darf, trotz des Gruppengefüges. Ob es für ein Lob oder eine
Zurechtweisung ist, spielt dabei keine Rolle. Wenn wir uns einmal die Situation
vor Augen führen, dass eine Lehrperson einen Schüler besonders loben will, so
kann der Lehrende dieses verbalisierte Lob durch seine Körpersprache
untermauern, indem er sich dem Schüler zuwendet und ihn direkt anblickt.
Außerdem kann er ihn auch direkt bei seinem direkten Namen nennen, sodass sich
der Schüler direkt angesprochen und wahrgenommen fühlt. Wenn der Lehrer jetzt
noch die Lautstärke der Stimme variiert und beispielsweise leiser spricht, führt
dies in der Gruppe wiederum möglicherweise zu einer verringerten
Gruppenlaustärke, da die anderen auch mitbekommen wollen, was die Lehrkraft
gerade gesagt hat. 
Als Lehrperson kann man sich den Unterrichtsraum zu Nutze machen, indem man ihn
unterschiedlich nutzt. So kann zum Beispiel mit Blick zur Wand geschrieben, mit Bilck zum
Fenster musiziert und mit Blick zur Tür Bewegung stattfinden. Durch diese
Variationen wird auch schon Schwung in den Unterricht gebracht. Ein wichtiger
Aspekt der Blickwinkel ist das gegenseitige Wahrnehmen. Wichtig ist, dass sich
alle Beteiligten gut sehen und hören können, weshalb sich beim Gruppenmusizieren
Kreise und Halbkreise anbieten. Durch initiierte Bewegungsanlässe kann die
Unterrichtseinheit außerdem dynamisch gehalten werden. Durch einen Ortswechsel
innerhalb eines Raumes kann die gelegentlich sinkenden Aufmerksamkeit wieder
angekurbelt werden. 
Der dritte Leitgedanke für die Arbeit mit Gruppen bezieht sich auf die
Aufgabenstellung. Um Missverständnisse zu vermeiden ist es zunächst einmal
wichtig, die Aufgaben so präzise wie möglich zu formulieren. Dies bedeutet
nicht, dass die Lehrperson implizit schon alles vorweg nimmt. Es darf und soll
im Unterrichtsgeschehen auch "Explorationsphasen" geben, in denen die Schüler
zunächst einmal unkoordiniert etwas selbst entdecken und für sich herausfinden
sollen. Diese Phasen sind im Gegenteil auch sehr wichtig für einen
abwechslungsreichen Unterricht, da die Schüler ihre persönlichen Erfahrungen
sammeln können und dadurch die intrinsische Motivation stärker angekurbelt wird.
Dies führt auch zu dem nächsten Punkt, dass die Schüler permanent beschäftigt
oder nennen wir es besser zu einer konkreten Aufgabe angeleitet
sein sollten. Wie man die Aufgaben innerhalb einer Gruppe verteilt kann aber
wiederum variieren, es können alle gleichzeitig das Selbe tun, genauso gut
können die Schüler nacheinander aber im gleichen Metrum etwas spielen, oder
jeder Schüler bekommt seine eigene Aufgabe in einem Kammermusikalischen
unterfangen. Zu dieser Beschäftigung kann aber auch zählen, einen bestimmten
Aspekt bei einem anderen Schüler oder in einem Musikstück zu beobachten. Auch
wenn dies kein aktives Ausführen von einer musikalischen Aktion ist. Außerdem
gibt es die Möglichkeit der Binnendifferenzierung innheralb einer Gruppe, die
besagt, dass man die Gruppe unterteilt und ihnen unterschiedlichen Aufgaben
aufträgt. Allerdings sollte vermieden werden, dass kein paralleler
Einzelunterricht innerhalb des Gruppenunterrichts entsteht. In der Unterteilung
haben wir in größeren Gruppen einen großen Spielraum, den man sich wunderbar zu
Nutze machen kann. So kann man mit allen zusammen arbeiten, mann kann die Gruppe
in zwei Hälften teilen, oder Partnerarbeit bis hin zur Einzelarbeit durchführen.
Diese vielen Aktionsformen zu variieren ist ein guter Wegweiser. Wie auch im EU
gilt im GU die Eigeninitiative der Lernenden zu stärke, was auch durch eine aktive
Mitbestimmung der Lernenden geschehen kann. In welchen Momenten ein Lehrer diese
Mitbestimmung zulässt, liegt in seinem eigenen Ermessen. Ähnlich ist es mit dem
Umgang von Nebenschauplätzen: wenn es sinnvoll scheint, kann man durchaus auf
sie eingehen und sie ggf. sogar mit in den Unterricht einbeziehen. In Bezug auf
die musikspezifischen Parameter, die einer Lehrperson zur Verfügung stehen,
beschreiben Barbara Busch und Barbara Metzger außerdem drei weitere
Handlungsfelder. Der erste Aspekt gilt der unterschiedlichen Umgangsweise mit
Musik wie der Rezeption, Reproduktion, Produktion und Transformation. Bedient
sich ein Lehrer dieser unterschiedlichen Felder, gelangen die Lernenden ganz
unterschiedliche Erfahungsfleder in der Musik. Der zweite Aspekt geht fließend
in diese Erfahungsfelder über, nämlich der Ausdrucksformen die wir am Instrument
haben. Zu ihnen gehört neben dem Spiel am Instrument auch das Bewegen zur und
mit der Musik und das Abbilden in jeglicher kreativen Form. Der dritte Aspekt
gilt dem bewussten Hören und der Schulung der innere Klangvorstellung, die
gezielt angeregt werden sollen, da sie für
jeden Instrumentalisten von großer Bedeutung sind.

- (differenzielles Lernen)
"Von einem bestimmen Alter an ist den Heranwachsenden die Rückmeldung der
Gleichaltrigen sowieso wichtiger als die des Lehrers, dann bewirtk die Kritik
eines Pultnachbarn manchmal eher eine Veränderung oder eine intensivere
Auseinandersetzung mit einem Stück als die des Dirigenten "Da vorn." S. 95


Evtl. ins Fazit? ((Die Musikhochschule müssten in diesem Punkt noch etwas besser ausbilden, da
jeder Instrumentalpädagoge darauf vorbereitet sein sollte und es so wichtig ist.
S. 94 oberer Abschnitt ))

"(...) oft ziemlich eng ausgebildeten Instrumentalpädagogen(...) S. 119


"Generell gilt, dass Unterrichtsformen, in denen der Lehrer mehr zurücktritt und
die Schüler eine Aufgabe alleine bearbeiten lässt, höchst fruchtbar für das
Unterrichtsgeschehen sein können. ( In solchen Phasen profitieren die Schüler
stark voneinander, sie erklären und demonstrieren sich gegenseitig bzw.
erarbeiten eigenständig einen Unterrichtsgegenstand. ) 
