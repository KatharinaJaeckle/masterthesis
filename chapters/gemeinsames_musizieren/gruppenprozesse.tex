\section{Gruppenprozesse}


- Individualität in der Gruppe 
"(...)Individualität zeigt sich gerade auch im Umgang mit anderen,
Individualität entwickelt sich im Vergleich zu und in ABgrenzung von anderen
oder auch dadurch, dass Positionen in einer Gruppe besetzt werden."

- unterschiedliche Alters- und Lernniveaus (manchmal) evtl. S. 95

- Lernen / Differenzielles Lernen? 
"Ein Großteil musikalischen Lernens findet über Imitation statt (...) S. 98

- Hierarchie
fortgeschrittenere Schüler zu Mentoren erklären und sie damit mitverantwortlich
für deren musikalische Entwicklung und den Fortschritt des Ensembles zu machen.
S. 95

- Altersgruppen/Erwachsene
"Wiedererlangen sozialer Integration scheint ein wesentliches Motiv
musikalischer Betätigung bei Erwachsenen (...). " s. 118

- "Selbsttätigkeit" (die Kunst zu unterrichten)
"Erst wenn es eine Notwendigkeit für einen Schüler gebe, eine Arbeit wirklich
selbst zu vollbringen, würde er ganzheitlich mit seinen Gefühlen, Trieben und
Willenskraft in die Tätigkeit involviert. (S. 63 bzw Eigl von Kerschensteiner)



\subsection*{Bedeutung für den Schüler*in}

"Man erfährt beim Musizieren soziale Resonanz(...)" S.28

"Eine Steigerung musikalischer Kommunikation erfährt der Schüler im
Zusammenspiel mit dem Lehrer, im Gruppen- oder Ensembleunterricht. (...) Die Individualität des eigenen Musizierens entwickelt sich in
Beziehung zu
anderen."  S.99 Punkt 5

"Das Gefühl von Zugehörigkeit und Zusammenklang beim Spielen im Ensemble."


\subsection*{Bedeutung für den Lehrer}

"Für den Instrumentallehrer stellt Gruppen- oder Ensembleunterricht eine höhere
Herausforderung dar, da er seine Aufmerksamkeit (gerecht) und flexibel
verteilen, differenziertere Spielanweisungen geben und eine musikalisch und
menschlich komplexer Situtation steuern muss. Andereseits ergeben sich im Umgang
mit der Gruppendynamik auch andere Möglichkeiten der Motivation und Förderung
musikalischen Lernens." S. 99 Punkt 6

- (differenzielles Lernen)
"Von einem bestimmen Alter an ist den Heranwachsenden die Rückmeldung der
Gleichaltrigen sowieso wichtiger als die des Lehrers, dann bewirtk die Kritik
eines Pultnachbarn manchmal eher eine Veränderung oder eine intensivere
Auseinandersetzung mit einem Stück als die des Dirigenten "Da vorn." S. 95


Evtl. ins Fazit? ((Die Musikhochschule müssten in diesem Punkt noch etwas besser ausbilden, da
jeder Instrumentalpädagoge darauf vorbereitet sein sollte und es so wichtig ist.
S. 94 oberer Abschnitt ))

"(...) oft ziemlich eng ausgebildeten Instrumentalpädagogen(...) S. 119
