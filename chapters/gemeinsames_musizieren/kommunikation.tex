\section{Kommunikation}

Unter Kommunikation verstehe ich die Art und Weise wie sich Lehrer und Schüler
im Unterricht begegnen und miteinander umgehen. 

"Sobald der Mensch musiziert, betritt er einen kommunikativen Raum. Der
Musizierende komminuziert mittels Musik, mit der Musik, mit sienem Instrument,
mit dem Komponisten, mit sich selbst, mit sienen Mitmusikern, mit dem Publikum
und über das Publikum mit der Gesellschaft, in der er lebt." (S. 56 Andreas
Doerne)

Musik = Kommunikation?! 

"Musik ist Bestandteil des soziokulturellen Raumes, der entsteht, wenn mehrere
Menschen miteinander in Kontakt treten. Als solche kann sie vielerlei
kommunikative Funktionen erfüllen." (Andreas Doerne S. 56 )

Kommunikation Musiker-Instrument
"...das beim Musizieren der musizierende Mensch mit mit seinem Instrument in
einen Akt gleichberechtigten Austausches tritt und dabei für den Musizierenden
der Eindruck entshet, dass er selbst wie auch das Instrument zugleich Sender und
Empfänger von Botschaften ist." (Andreas Doerne S. 59)


Musiker-Komponist
"Setzt sich ein Musiker mit einem Musikwerk auseinander, beschäftigt er sich
gleichzeitig immer auch mit der Körper-, Gefühls-, und Gedankenwelt des
Komponsiten." (Andreas Doerne S. 59)

"Der Musizierende prägt die Musik genauso wie die Musik den Musizierenden prägt.
Angestoßen durch KOmmunikation, beginnen beide Welten - die des Musikers und die
der Musik - sich zu verbinden, indem sie zu einer gemeinsamen Bewegung, einem
gemeinsamen Atmen, einem gemeinsamen Fühlen und Denken verschmelzen." (S. 60)

Kommunikation beim Musizieren: dialogisch oder monologisch (S. 61 Andreas Doerne)


"Der Musizierende selbst als Ensemble" bestehend aus divergenten Teilpersönlichkeiten und
unterschiedlichen psychischen Dynamiken." (S. 64 Andreas Doerne)

"innere Team -schulz von thun" s. 65

Kommunikation mit Publikum Andreas Doerne S. 66

Gelungene KOmmunikation: "dialogisch KOmmunizieren" Andreas Doerne S. 67

"So sind Kommunikation und Persönlichkeitsbildung zwei Seiten der Medaille."
(Andreas 69/ Schul von Thun)
