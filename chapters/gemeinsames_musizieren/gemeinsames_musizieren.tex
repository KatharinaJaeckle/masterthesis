\addchap{Gemeinsames Musizieren}

Das gemeinsame Musizieren zu lernen ist ein wichtiger Prozess, der in der
Instrumentalausbildung ein fester Bestandteil sein sollte. Schließlich müssen
die Instrumentalschüler*innen erst einmal an die Gruppenprozesse herangeführt
werden um zu lernen wie man (musikalisch) interagieren kann. Das schließt
allerdings nicht aus, dass dabei das gemeinsam das Musizieren an sich gelernt wird. Im
besten Fall wird nämlich genau das erreicht. Bei dem gemeinsamen Musizieren
steht die Freude am Musizieren im
Vordergrund und nicht die technischen Hürden, die ein*e
Instrumentalschüler*in überwinden muss. Insofern sollten am
besten der Einzelunterricht und das gemeinsame Musizieren in der Praxis ineinander greifen.

Bei dem gemeinsamen Musizieren spielen viele verschiedene Faktoren eine Rolle.
Im Folgenden werde ich deswegen mehrere Aspekte beleuchten und erörtern.
Beispielsweise welche Aspekte die Schüler*innen weiterbringen und für sie
gewinnbringend sein können. Außerdem werfe ich einen Blick auf die Aufgabe der
Instrumentallehrer*innen um darauf einzugehen, wie man verschiedene Gruppen und
Situationen wirkungsvoll leiten kann. Alle Aspekte spielen sich auf der sozialen
interaktiven Ebene ab, im Wechselspiel zwischen dem Individuum und der Gruppe.
Dafür werde ich die Gruppenprozesse, sowie mögliche Aktionsformen für Gruppen,
die Hierarchie innerhalb einer Gruppe, die Kommunikation, Motivation und das
psychosoziale Lernen, welches in der Gruppe stattfindet betrachten, sowie die
Bedeutung von Gruppenarbeit für den Lernenden und die Bedeutung für
die Lehrkraft.

















%"Musik als Kommunikationsmittel" S. 17 doerne nicht

%Aristoteles: "der Schlüssel zu einem glücklichen Leben besteht im Miteinander,
%das Leben in der Gemeinschaft unter Freunden." (S. 9 "Die Kunst zu
%unterrichten)

%%"Sie muss gelernt und ein Leben lang geübt werden." Zitat? doerne nicht


%"Jedes Menschen Persönlichkeit, das bedeutet: seine Fähigkeiten, seine Art zu
%denken und zu fühlen, wird durch Umstände und Umgebung geschnitzt und
%gemeißelt." (S. 20 shinichi Suzuki)

%Andreas Doerne schreibt sehr zutreffend "(...) dass Musik in ihrem Wesen
%Mitteilung, Kommunikation, Gestaltung, das Vermitteln einer Botschaft ist, die
%in Sternstunden über das Erleben der Einzelnen hinausreicht und sie miteinander
%verbinden kann." (Nope niht doerne) zitat?



\section{Gruppenprozesse}


- Individualität in der Gruppe 
"(...)Individualität zeigt sich gerade auch im Umgang mit anderen,
Individualität entwickelt sich im Vergleich zu und in ABgrenzung von anderen
oder auch dadurch, dass Positionen in einer Gruppe besetzt werden."

- unterschiedliche Alters- und Lernniveaus (manchmal) evtl. S. 95

- Lernen / Differenzielles Lernen? 
"Ein Großteil musikalischen Lernens findet über Imitation statt (...) S. 98

- Hierarchie
fortgeschrittenere Schüler zu Mentoren erklären und sie damit mitverantwortlich
für deren musikalische Entwicklung und den Fortschritt des Ensembles zu machen.
S. 95

- Altersgruppen/Erwachsene
"Wiedererlangen sozialer Integration scheint ein wesentliches Motiv
musikalischer Betätigung bei Erwachsenen (...). " s. 118

- "Selbsttätigkeit" (die Kunst zu unterrichten)
"Erst wenn es eine Notwendigkeit für einen Schüler gebe, eine Arbeit wirklich
selbst zu vollbringen, würde er ganzheitlich mit seinen Gefühlen, Trieben und
Willenskraft in die Tätigkeit involviert. (S. 63 bzw Eigl von Kerschensteiner)

-Gruppe als Korrektiv

\subsection*{Bedeutung für den Schüler*in}

Welche Unterrichtsform für welchen Schüler besonders geeignet ist, ist ein sehr
individuelles Unterfangen. Während der eine Schüler die ganze Aufmerksamkeit
eines Lehrers genießt, kann sich der andere im Einzelunterricht unwohl fühlen.
Wer ein ser kommunikativer Mensch ist, ist hingegen vermutlich in den
Gruppenunterrichten besser aufgehoben. Oftmals ändern sich die Unterrichtsformen
für einen Schüler allerdings auch, da es gängig ist in Gruppenunterrichten zu
starten. Die Form eines Kleingruppenunterrichts bei Anfängern bietet sich auch
an, siehe Zitat \autocite[220]{Busch:grundwissen_Instrumentalpädagogik}

"Man erfährt beim Musizieren soziale Resonanz(...)" S.28

"Eine Steigerung musikalischer Kommunikation erfährt der Schüler im
Zusammenspiel mit dem Lehrer, im Gruppen- oder Ensembleunterricht. (...) Die Individualität des eigenen Musizierens entwickelt sich in
Beziehung zu
anderen."  S.99 Punkt 5

"Das Gefühl von Zugehörigkeit und Zusammenklang beim Spielen im Ensemble."


\subsection*{Bedeutung für den Lehrer}

Je nach Gruppengröße muss der Lehrer verschiedene Aufgaben übernehmen, zu der
auch eine unterschiedliche Vorbereitungsnotwendigkeit hinzukommt. Während der
der Lehrende im Einzelunterricht unmittelbar auf die Impulse des Schülers
eingehen kann, so muss der Lehrende bei Kleingruppen neben klaren Vorgaben die
Schüler zu einem von- und miteinander Lernen anregen und sich selbst auch einmal
zurücknehmen. Im Großgruppenunterricht ist es essentiell, dass der Lehrer im
Vorfeld eine Struktur und Stücke festlegt um die Inhalte gut anleiten zu können.\autocite[220]{Busch:grundwissen_Instrumentalpädagogik}
Meistens hat man in dieser Konstellation die Situation eines Frontalunterrichts,
was beim Klassenmusizieren von den Instrumentalpädagogen verlangt wird, während die
Instrumentalpädagogen mit diesem Szenario oftmals nicht vertraut sind.

"Für den Instrumentallehrer stellt Gruppen- oder Ensembleunterricht eine höhere
Herausforderung dar, da er seine Aufmerksamkeit (gerecht) und flexibel
verteilen, differenziertere Spielanweisungen geben und eine musikalisch und
menschlich komplexer Situtation steuern muss. Andereseits ergeben sich im Umgang
mit der Gruppendynamik auch andere Möglichkeiten der Motivation und Förderung
musikalischen Lernens." S. 99 Punkt 6

- (differenzielles Lernen)
"Von einem bestimmen Alter an ist den Heranwachsenden die Rückmeldung der
Gleichaltrigen sowieso wichtiger als die des Lehrers, dann bewirtk die Kritik
eines Pultnachbarn manchmal eher eine Veränderung oder eine intensivere
Auseinandersetzung mit einem Stück als die des Dirigenten "Da vorn." S. 95


Evtl. ins Fazit? ((Die Musikhochschule müssten in diesem Punkt noch etwas besser ausbilden, da
jeder Instrumentalpädagoge darauf vorbereitet sein sollte und es so wichtig ist.
S. 94 oberer Abschnitt ))

"(...) oft ziemlich eng ausgebildeten Instrumentalpädagogen(...) S. 119





