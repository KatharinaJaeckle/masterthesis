\addchap{Gemeinsames Musizieren}





Bei dem Gemeinsamen Musizieren spielen viele verschiedene Faktoren eine Rolle
für die Schüler. Im Folgenden werde ich deswegen verschiedene Aspekte
beleuchten und erörtern inwieweit sie die Schülerinnen weiterbringen und für sie
gewinnbringend sein können.
Alle Aspekte spielen sich auf der sozialen interaktiven Ebene ab, sei es im
Wechselspiel zwischen dem Individuum und der Gruppe. 

Betrachtet man eine Gruppe, die gemeinsam auf etwas hinarbeitet, spielen sich
viele Prozesse innerhalb ein jeder Gruppe ab. Diese können sehr unterschiedlich
aussehen, da die Hierarchie unterschieldich sein kann, wie wir später bei der
näheren Beleuchtung der verschiedenen Systeme, sehen werden. 

Gaudig: wichtigstes Erziehgunsziel die Entwicklung der Persönlichkeit. (S. 63
die Kunst zu uneterrichten) Gleichzeitig auch die Ordnung der Gemeinschaft
kennen lernen. Dazu müssten sie zu Eigenaktivität, Selbsttätigkeit und konsequenter
eigenständiger Arbeit hingeführt werden." (S. 64 die kunst zu unterrichten)

"Aus einer zweipoligen Interaktionsform im Einzelunterricht ergeben sich
mehrpolige Interaktionsformen im Gruppenuntericht, was Unterrichtssituationen
für gewöhnlich komplexer erscheinen lässt." (S.30, die Kuns zu Unterrichten)


"Generell gilt, dass Unterrichtsformen, in denen der LEhrer mehr zurücktritt und
die Schüler eine Aufgabe alleine bearbeiten lässt, höchst fruchtbar für das
Unterrichtsgeschehen sein können. ( In solchen Phasen profitieren die Schüler
stark voneinander, sie erklären und demonstrieren sich gegenseitig bzw.
erarbeiten eigenständig einen Unterrichtsgegenstand. ) 

Der Fokus ist weniger auf
das Lehren des Lehrers als auf das Verstehen und Einüben des Schülers
gerichtet."  (S.31 die kunst zu unterrichte)


Gemeinsames Lernen fördert nicht nur soziale Kompetenzen wie Kommunikations- und
Kooperationsfähigkeiten, auch musikalische Fähigkeiten, wie beispielsweise
rhythmische und intonatorische Genauigkeit, musiktheoretische Kenntnisse oder
Gehörtboldung, lassen sich gut im Gruppenunterricht anbahnen." (S. 32 Die Kunst
des Unterrichtens)

"Zudem können gruppendynamische Prozesse zu einer individuellen
Effektivitätssteigerung führen." (S. 32)

- Gemeinsam Stücke spielen, die man sonst nicht spielen kann. Jeder hat an
anderen Stellen Schwierigkeiten und ist wiederum an anderen Stellen sicher im
Notentext und kann an den Stellen für eine Unterstützung sorgen. (S. 32 Die
Kunst zu unterrichten)

Gruppenunterricht: technische Präzision geht verloren? 
Fazit: "Die  Erteilung von reinem Gruppenunterricht ist nicht ratsam. Das
bedeutet also, dass zum Gruppenunterricht immer der Einzelunterricht hinzutritt,
der speziell für eine derartige Einzelbetreuung der Kinder genutzt wird."
\autocite[57]{ernst:die_zukunftsfaehige_musikschule}



%"Musik als Kommunikationsmittel" S. 17 %doerne?

%Aristoteles: "der Schlüssel zu einem glücklichen Leben besteht im Miteinander,
%das Leben in der Gemeinschaft unter Freunden." (S. 9 "Die Kunst zu unterrichten)

%%"Sie muss gelernt und ein Leben lang geübt werden." (auch S. 9 %Doerne?)


%"Jedes Menschen Persönlichkeit, das bedeutet: seine Fähigkeiten, seine Art zu
%denken und zu fühlen, wird durch Umstände und Umgebung geschnitzt und
%gemeißelt." (S. 20 shinichi Suzuki)

%Andreas Doerne schreibt sehr zutreffend "(...) dass Musik in ihrem Wesen Mitteilung, Kommunikation, Gestaltung, das
%Vermitteln einer Botschaft ist, die in Sternstunden über das Erleben der
%Einzelnen hinausreicht und sie miteinander verbinden kann." S. 11 Doerne?
%(andreas doerne?) 



\section{Gruppenprozesse}


- Individualität in der Gruppe 
"(...)Individualität zeigt sich gerade auch im Umgang mit anderen,
Individualität entwickelt sich im Vergleich zu und in ABgrenzung von anderen
oder auch dadurch, dass Positionen in einer Gruppe besetzt werden."

- unterschiedliche Alters- und Lernniveaus (manchmal) evtl. S. 95

- Lernen / Differenzielles Lernen? 
"Ein Großteil musikalischen Lernens findet über Imitation statt (...) S. 98

- Hierarchie
fortgeschrittenere Schüler zu Mentoren erklären und sie damit mitverantwortlich
für deren musikalische Entwicklung und den Fortschritt des Ensembles zu machen.
S. 95

- Altersgruppen/Erwachsene
"Wiedererlangen sozialer Integration scheint ein wesentliches Motiv
musikalischer Betätigung bei Erwachsenen (...). " s. 118

- "Selbsttätigkeit" (die Kunst zu unterrichten)
"Erst wenn es eine Notwendigkeit für einen Schüler gebe, eine Arbeit wirklich
selbst zu vollbringen, würde er ganzheitlich mit seinen Gefühlen, Trieben und
Willenskraft in die Tätigkeit involviert. (S. 63 bzw Eigl von Kerschensteiner)

-Gruppe als Korrektiv

\subsection*{Bedeutung für den Schüler*in}

Welche Unterrichtsform für welchen Schüler besonders geeignet ist, ist ein sehr
individuelles Unterfangen. Während der eine Schüler die ganze Aufmerksamkeit
eines Lehrers genießt, kann sich der andere im Einzelunterricht unwohl fühlen.
Wer ein ser kommunikativer Mensch ist, ist hingegen vermutlich in den
Gruppenunterrichten besser aufgehoben. Oftmals ändern sich die Unterrichtsformen
für einen Schüler allerdings auch, da es gängig ist in Gruppenunterrichten zu
starten. Die Form eines Kleingruppenunterrichts bei Anfängern bietet sich auch
an, siehe Zitat \autocite[220]{Busch:grundwissen_Instrumentalpädagogik}

"Man erfährt beim Musizieren soziale Resonanz(...)" S.28

"Eine Steigerung musikalischer Kommunikation erfährt der Schüler im
Zusammenspiel mit dem Lehrer, im Gruppen- oder Ensembleunterricht. (...) Die Individualität des eigenen Musizierens entwickelt sich in
Beziehung zu
anderen."  S.99 Punkt 5

"Das Gefühl von Zugehörigkeit und Zusammenklang beim Spielen im Ensemble."


\subsection*{Bedeutung für den Lehrer}

Je nach Gruppengröße muss der Lehrer verschiedene Aufgaben übernehmen, zu der
auch eine unterschiedliche Vorbereitungsnotwendigkeit hinzukommt. Während der
der Lehrende im Einzelunterricht unmittelbar auf die Impulse des Schülers
eingehen kann, so muss der Lehrende bei Kleingruppen neben klaren Vorgaben die
Schüler zu einem von- und miteinander Lernen anregen und sich selbst auch einmal
zurücknehmen. Im Großgruppenunterricht ist es essentiell, dass der Lehrer im
Vorfeld eine Struktur und Stücke festlegt um die Inhalte gut anleiten zu können.\autocite[220]{Busch:grundwissen_Instrumentalpädagogik}
Meistens hat man in dieser Konstellation die Situation eines Frontalunterrichts,
was beim Klassenmusizieren von den Instrumentalpädagogen verlangt wird, während die
Instrumentalpädagogen mit diesem Szenario oftmals nicht vertraut sind.

"Für den Instrumentallehrer stellt Gruppen- oder Ensembleunterricht eine höhere
Herausforderung dar, da er seine Aufmerksamkeit (gerecht) und flexibel
verteilen, differenziertere Spielanweisungen geben und eine musikalisch und
menschlich komplexer Situtation steuern muss. Andereseits ergeben sich im Umgang
mit der Gruppendynamik auch andere Möglichkeiten der Motivation und Förderung
musikalischen Lernens." S. 99 Punkt 6

- (differenzielles Lernen)
"Von einem bestimmen Alter an ist den Heranwachsenden die Rückmeldung der
Gleichaltrigen sowieso wichtiger als die des Lehrers, dann bewirtk die Kritik
eines Pultnachbarn manchmal eher eine Veränderung oder eine intensivere
Auseinandersetzung mit einem Stück als die des Dirigenten "Da vorn." S. 95


Evtl. ins Fazit? ((Die Musikhochschule müssten in diesem Punkt noch etwas besser ausbilden, da
jeder Instrumentalpädagoge darauf vorbereitet sein sollte und es so wichtig ist.
S. 94 oberer Abschnitt ))

"(...) oft ziemlich eng ausgebildeten Instrumentalpädagogen(...) S. 119





