\section{Motivation}

"(...) ist das Zusammenspiel für Instrumentalschüler eine der stärksten Quellen
der Motivation." S. 92 Das liegt daran, dass es sich im Vergleich zu Anderen und
teilweise auch gleichaltrigen hört und einordnen lernen kann. Außerdem kann der Instrumentalschüler
sich von anderen inspirieren und motivieren lassen und teilweise Stücke spielen,
die er alleine vielleicht noch nicht ganz spielen könnte. Da das Ensemble aber
auch dazu helfen kann den Einzelnen über schwierigere Passagen hinweg zu tragen,
erreicht der Schüler ggf. einen ganz neuen Horizont und gelangt dadurch wieder
zu mehr Selbstbewusstsein. Aber auch fortgeschrittene Schüler können
profitieren, indem sie entweder solistisch heraustreten können aber Lernen
können ihre Klang- und Spielvorstellung weiter zu geben. Dazu müssen sie die
eigene oftmals auch selbst noch einmal mehr schärfen. "In guten Fällen reißt das
Ensemble den Einzelnen mit, motiviert auch den Faulen, schwingt sich gemeinsam
zu größeren Leistungen auf, als es der Einzelne erbringen könnte, stärkt undd
hält die Einzelnen." (S.94)


"(...) nach wie vor kann man Jugendliche durch sein eigenes Spiel, die erlebten
Wirkungen und des Zusammenspiels im Ensembles für Musik begeistern." S. 122


" Die ERfahrung von Zugehörigkeit über das gemeinsame Musizieren wird für immer
mehr Menschen Motivation und wichtiger Teil der musikalischen Identität.2 s. 123
