\addchap{Fazit}
Ein Plädoyer für das Gemeinsame Musizieren

%Seite 60-62 erklärt warum Ensemble-Spiel so unumgänglich wichitg ist (anselm
%ernst)
Warum das gemeinsame Musizieren so essentiell ist, für jeden
Instrumentalschüler, habe ich im Vorhinein genug beschrieben. Er ist
unumgänglich und sollte viel mehr fest in die musikalische Ausbildung integriert
sein. Eine Parallele der verschiedenen untersuchten Systeme ist, dass sowohl El
Sistema als auch das Konzept hinter JeKi vorallem für sozialschwächere Kinder
gedacht ist, um diesen durch die Musik eine bessere Perspektive zu ermöglichen.
Die Methode nach Suzuki hingegen, wird durch die wichtige Rolle der Eltern, die
als Begleitpersonen mit ihren Kindern den Unterricht besuchen und zu Hause mit
ihnen üben sollen eher zu einem elitären Modell, da sich nicht alle Eltern diese
große Unterstützung finanziell erlauben können bzw. auch nicht alle Eltern sich
diese Zeit nehmen wollen. Da ist die EMP als Grundform wiederum eine sehr
fortschrittliche Methode, welche wiederum in Ländern ausserhalb von Deutschland
noch nicht so verbreitet ist und die die Musik zum richtigen Zeitpunkt durch den
Kindergarten oder die Grundschule an die Kinder heranträgt. Es gibt zwar
musikalische Früherziehungsgruppen nach Gordon, in den USA beispielsweise und
sicherlich auch die ein oder andere musikalische Früherziehung in anderen
Ländern. Dadurch, dass die EMP in Deutschland zumindest teilweise in das
Bildungssystem eingegliedert ist, haben manche Kinder die Möglichkeit mit der
Musik und dem gemeinsamen Musizieren in Kotankt zu kommen. Fakt ist auch, dass
es noch keine Selbstverständlichkeit ist, dass jeder Kindergarten eine gute
musikalische Erziehung gewährleisten kann und die EMP leider doch noch nicht so
tief im System verankert ist, dass selbstverständlich jede Kindertagesstätte,
jeder Kindergarten oder jede Grundschule die EMP als Teil ihres Programmes
anbietet. Zum Teil wird dies durch das Angebot von den Musikschulen kompensiert,
aber auch in der späteren Ausbildung schafft es die Institution Musikschule
leider doch zu selten, das gemeinsame Musizieren durch Gruppenunterrichte,
genügend Ensembles und Orchesterangebote abzudecken. Dabei hat der Verband
deutscher Musikschulen richtig erkannt, dass \enquote{(...) ein ständiges
Zusammenwirken von Einzelspiel und Zusammenspiel einerseits, von praktischem
Musizieren und Verstehen des Gespielten andererseits.} wichtig ist.
\autocite[22]{losert:die_kunst_zu_unterrichten} Das System ist zu starr und zu
undruchlässig. Zuvor habe ich einige Formen des Gruppenunterrichts aufgelistet,
die viel durchlässiger eingesetzt werden sollten. Ich bin der Meinung, dass sich
Einzelunterricht in Kombination zu verschiedenen Aktionsformen in
Gruppenunterrichten am besten ergänzt und zu einem Bestandteil des festen
Angebotes werden sollte. Jeder Schüler sollte genau dieses Wechselspiel zwischen
Einzelunterricht und gemeinsamen Musizieren in seiner musikalischen Ausbildung
erleben können. Wie wir gelernt haben, fördert es nicht nur die Motivation,
sondern dient der Entwicklung des Einzelnen in seiner Persönlichkeit von großer
Bedeutung. Aus dem Kapitel der verschiedenen Unterrichtsformen geht deutlich
hervor, dass Gruppenunterricht ein besonders wichtiger Baustein unserer
Unterrichtspraxis ist und sein sollte. Einzelunterricht alleine bietet nicht das
ideale Lernfeld für Instrumentalschüler. Auch umgekehrt funktioniert das
ausschließliche Lernen in der Gruppe wie wir am Beispiel von JeKits sehen
genauso wenig und es bedarf den zusätzlichen EU. Da das Musikschulsystem in
Deutschland allerdings im Instrumentalunterricht zu sehr auf den instrumentalen
EU ausgelegt ist, möchte in an dieser Stelle noch einmal auf das Modell des
Multidimensionalen Unterricht nach Gerhard Wolters verweisen, was mir eine
spannende Alternative zu sein scheint.
%wie funktioniert das nochmal? und warum eine gute Alternative?
\enquote{Die wissenschaftliche Erkenntnis als solche reicht nicht aus. Neues
Wissen müssen sich die Lehrenden individuell aneignen und in praktische
pädagogische Kompetenz umwandeln.}
\autocite[10]{losert:die_kunst_zu_unterrichten} Ein möglicher Lösungsansatz
neben dem multidimensionalen Unterrichten wäre, die Einzelstunden eines Schülers
zu kürzen und mit der gekürzten Zeit von allen, eine gemeinsame Gruppenstunde zu
planen. So bleibt die Unterrichtszeit für die Lehrperson gleich, aber gerade bei
größeren Gruppen wird die Unterrichtszeit jedes Einzelnen Schülers deutlich
länger.\autocite[33]{losert:die_kunst_zu_unterrichten} Außerdem schreibt Anselm
Ernst, dass ein Schüler von Anfang an in ein Ensemble aufgenommen werden sollte
und sogar dazu verpflichtet sein sollte.
\autocite[61]{ernst:die_zukunftsfaehige_musikschule} Diese Verpflichtung sehe
ich genauso, auch wenn die Schüler heutzutage sehr wenig Zeit und volle
Wochenpläne haben. Nur wenn sich der Kreislauf schließt und durch die
Verpflichtung alle Schüler die Erfahrung des gemeinsamen Musizierens machen
können, kann unser Musikschulsystem und die Begeisterung zur Musik erhalten
werden. Man muss nicht vorgeben, in welcher Art von Ensemble ein Schüler wann zu
spielen hat, aber man muss ein breites Ensembleangebot anbieten können und die
Schüler können sich dann für eines ergänzend zu ihrem Einzelunterricht
entscheiden. Wenn die Instrumentalpädagogen Hilfsmittel bezüglich der
Gruppenarbeit benötigen, lohnt sich immer ein Blick auf die Musikalische
Früherziehung, von der wir einige Methoden ableiten können. 

%\enquote{Die  Erteilung von reinem Gruppenunterricht ist nicht ratsam. Das
%bedeutet also, dass zum Gruppenunterricht immer der Einzelunterricht
%hinzutritt, der speziell für eine derartige Einzelbetreuung der Kinder genutzt
%wird.} \autocite[57]{ernst:die_zukunftsfaehige_musikschule}






% - Ausbildung nicht gut genug S. 119/ S. 94
% - EMP hilft und bietet Potential/Techniken

% - Schwierigkeit der Organisation


% 4 Sozialformen im Instrumentalunterricht S. 34-35    (die Kunst zu
% unterrichten)



% Gruppenunterricht: technische Präzision geht verloren? Fazit: "
