\addchap{Fazit}
Ein Plädoyer für das Gemeinsame Musizieren

Eine Parallele der verschiedenen untersuchten Systeme ist, dass sowohl El Sistema als auch das Konzept hinter JeKi vorallem
für sozialschwächere Kinder gedacht ist, um diesen durch die Musik eine bessere
Perspektive zu ermöglichen. Die Methode nach Suzuki hingegen, wird durch die
wichtige Rolle der Eltern, die als Begleitpersonen mit ihren Kindern den
Unterricht besuchen und zu Hause mit ihnen üben sollen eher zu einem elitären
Modell, da sich nicht alle Eltern diese große Unterstützung finanziell erlauben
können bzw. auch nicht alle Eltern sich diese Zeit nehmen wollen. 
Da ist die EMP als Grundform wiederum eine sehr fortschrittliche Methode, welche
wiederum in anderen Ländern noch nicht so verbreitet ist und die die Musik zum
richtigen Zeitpunkt durch den Kindergarten oder die Grundschule an die Kinder
heranträgt. Dadurch, dass die EMP zumindest teilweise in das Bildungssystem eingegliedert ist, haben
alle Kinder die Möglichkeit mit der Musik und dem gemeinsamen Musizieren in
Kotankt zu kommen. Fakt ist aber auch, dass es noch keine Selbstverständlichkeit
ist, dass jeder Kindergarten eine gute musikalische Erziehung gewährleisten kann
und es leider doch noch nicht so tief im System verankert ist, dass
selbstverständlich jeder Kindergarten die EMP als Teil seines Programmes
anbietet.  

Wenn wir nun überlegen was wir aus den bestehenden Systemen mitnehmen können, so
komme ich zu dem Ergebnis, dass: 

Dies führt am Ende dazu, dass eine ideale Umgebung für das gemeinsame Musizieren
so geschaffen werden kann, indem man ...

Aufgabe der Musikschule hat der Vdm sehr richtig erkannt, indem er sagt: "(...)
wichtig ist ein ständiges Zusammenwirken von Einzelspiel und Zusammenspiel
einerseits, von praktischem Musizieren und Verstehen des Gespielten
andererseits." (S. 22 die kunst zu unterrichten)

Meiner Meinung nach sollte der Schüler genau dieses Wechselspiel zwischen
Einzelunterricht und gemeinsamen Musizieren in seiner musikalischen Ausbildung
erleben können. Wie wir gelernt haben, fördert es nicht nur die Motivation,
sondern ist auch zur Entwicklung des Einzelnen in seiner Persönlichkeit von
gr0ßer Bedeutung. 


- Ausbildung nicht gut genug S. 119/ S. 94
- EMP hilft und bietet Potential/Techniken

- Schwierigkeit der Organisation

Lösungsansatz:
- Einzelstunden der Schüler kürzen und dafür eine gemeinsame Gruppenstunde im
Anschluss planen (S. 33, die kunst zu unterrichten)
--> Unterrichtszeit für den Lehrer bleibt gleich, aber gerade bei größeren
Gruppen wird die UNterrichtszeit für einen Schüler deutlich länger!
    alternativ kann man auch wöchentlich routieren oder solchen


4 Sozialformen im Instrumentalunterricht S. 34-35    (die Kunst zu unterrichten)

Die wissenschaftliche Erkenntnis als solche reicht nicht aus. Neues Wissen
müssen sich die Lehrenden individuell aneignen und in praktische pädagogische
Kompetenz umwandeln." (S. 10 die Kunst zu unterrichten)

Anselm Ernst S. 61 "Unter dem Motto "von Anfang an" sollten alle Schülerinnen und
Schüler in ein Ensemble aufgenommen werden, ja sogar darauf verpflichtet werde."

- Seite 60-62 erklärt warum Ensemble-Spiel so unumgänglich wichitg ist

Aus dem Kapitel der verschiedenen Unterrichtsformen geht deutlich hervor, dass
Gruppenunterricht ein besonders wichtiger Baustein unserer Unterrichtspraxis
ist und sein sollte. Einzelunterricht alleine bietet nicht das ideale Lernfeld
für Instrumentalschüler. Auch umgekehrt funktioniert das ausschließliche Lernen
in der Gruppe wie wir am Beispiel von JeKi sehen genauso wenig und es bedarf den
zusätzlichen EU. Da das Musikschulsystem in Deutschland allerdings im
Instrumentalunterricht zu sehr auf den instrumentalen EU ausgelegt ist, möchste
in an dieser Stelle noch einmal auf das Modell des Multidimensionalen Unterricht
nach Gerhard Wolters verweisen, was mir eine spannende Alternative zu sein
scheint. 
