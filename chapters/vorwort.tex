\addchap{Vorwort}

Mein persönlicher Hintergrund aus dem ich heraus dieses Thema gewählt habe, war
zum einen die Situation meiner Privatschüler*innen, zum anderen die Situation der
Instrumentalschüler*innen am Birklehof, einem Privatinternat an dem ich unterrichte
und zu guter Letzt der Einfluss von Corona auf unsere soziale Umwelt. Meine
Privatschüler*innen erhalten einmal wöchentlich meinen Unterricht, darüber hinaus sind
sie aber nicht ansatzweise in eine musikalische Umgebung eingebettet. Drei
Ausnahmen sind eine Schülerin, deren Mutter selbst Geigenlehrerin ist, wo zu
Hause viel von der Familie aus musiziert wird und zwei Medizin-Studenten, die
mit großer Begeisterung im Uni-Orchester spielen. Aber die Zeiten von Corona
gehen nicht spurlos an dem Instrumentalunterricht vorüber. Die Lockdowns, die
beispielweise alle denkbaren Ensembleproben zu einem schwierigen Unterfangen
gemacht haben, die Abstandsregelen und auch das Verbot von Veranstaltungen in
größeren Gruppen, wie Konzerte, hat vor allem eines sehr deutlich gezeigt: der
Mensch braucht sein soziales Umfeld sowie die soziale Interaktion um sich weiter
zu entwickeln. Wie wichtig dieser Bestandteil auch für unsere musikalischen
Erfahrungen ist, werde ich in meiner Masterarbeit näher beleuchten.

Ein herzliches Dankeschön geht an die Einflüsse, Anregungen und den Input meiner
Professoren Wolfgang Lessing und Andreas Doerne, sowie Michael Stecher die
vorallem im musikpädagogischen Bereich prägend für mich waren und meinen
Horizont immer weiter wachsen lassen haben. Für mein künstlerisches Dasein, war
die Studienzeit genauso anregend, da viele, um nicht zu sagen alle Fragen die
aufgeworfen wurden, auch mit meiner Person als ausübende Künstlerin eng
verbunden waren. Außerdem bedanke ich mich bei meinem Kommilitonen Timo Langpap,
der zurecht in diesem Zusammenhang sagen würde, dass auch wir Instrumentallehrer*innen
unsere eigene künstlerische Persönlichkeit pflegen und in den
Instrumentalunterricht einbringen müssen. Wir haben viel Co-Working für Projekte
inner- und außerhalb des Studiums durchgeführt, uns immer wieder ausgetauscht und
diskutiert.
Dank Herr Prof. Lessing haben wir außerdem an dem Hochschulwettbewerb der
deutschen Musikhochschulen teilgenommen und mit unserem Projekt \emph{Digitale
Kettenkomposition} einen zweiten Preis erhalten. In dem Projekt haben wir
Schüler aus verschiedenen Orten, mit verschiedenen Instrumenten und Genres
musikalisch miteinander verknüpft und damit die ein oder andere Lockdown-Welle
für unsere Schüler erträglicher machen können. 

Ich bedanke mich bei den Korrekturlesern und für die
Unterstützung in der Nutzung des Programms LaTex bei Wolfgang Drescher.




\vspace{0.5cm}

\begin{flushright}
	{
		\small
		Freiburg im Breisgau, September 2021\\
		--- Katharina Jäckle
	}
\end{flushright}
