\section{Instrumentalunterricht}
Der Instrumentalunterricht bietet viele verschiedene Formen vom Einzelunterricht
(EU), über den Gruppenunterricht (GU), bis hin zu dem sogenannten
"Multidimensionalen Unterricht". Man könnte auch soweit gehen, dass Band- Chor-
oder Orchesterproben mit einem Lehrer zu dem normalen Instrumentalunterricht
zählen. Als Erstes verknüpfen wir im europäischen System jedoch mit dem
klassischen Instrumentalunterricht den Einzelunterricht, wo ein Lehrer einen
einzelnen Schüler im Normalfall zwischen dreißig und neunzig Minuten
unterrichtet. Dazu gehe ich im Folgenden noch mehr ein. Was den
Gruppenunterricht ausmacht, wird auch näher beleuchtet werden. Fest steht auf
jeden Fall, dass im Idealfall das Lehren und Lernen flexibel gestaltet werden
und sich im besten Fall der EU und GU ergänzen. In unserem Musikschulsystem ist
leider noch zu wenig Raum für die Kombination aus individuellem- und
gemeinschaftlichen Lernen. So schreibt Anselm Ernst in seinem Buch Die
zukunftsfähige Musikschule: "Wer genügend Raum und Zeit hat, kann sich mit Ruhe,
Geduld und Gelassenheit den Menschen und den Sachen widmen".
\autocite[84]{ernst:die_zukunftsfaehige_musikschule} Bei allen
Unterrichtsformen, so unterschiedlich sie sein mögen und mit ihren jeweiligen
Vor- und Nachteilen soll es darum gehen, dass der Instrumentalschüler das für
ihn am besten geeignete Lernumfeld geboten bekommt. Dazu sollte es auf alle
Fälle variabel sein und keine starre einseitige Unterrichtsform, sondern eine
Wechselwirkung aus allen Möglichkeiten, aus denen wir Musiker schöpfen können. 


\subsection{Einzelunterricht}
Im Einzelunterricht (EU) erhält der Instrumentalschüler die volle Aufmerksamkeit
des Lehrers. Dies hat seinen Vorteil darin, dass der Lehrer auf der
persönlicheren Ebene viel mehr auf den Schüler eingehen kann, sowie
spieltechnisch in einem auf den Schüler abgestimmten Tempo arbeiten kann.
Außerdem erlaubt der EU, tiefer und detaillierter auf das ein oder andere Thema
einzugehen, zum Beispiel in der Spieltechnik oder Körperhaltung. Die volle
Aufmerksamkeit des Lehrers kann jedoch auch genauso gut zum Nachteil werden, da
der Schüler ausschließlich von der Lehrperson seine Anweisungen erhält und der
Lernweg ausschließlich eindimensional stattfindet. Der Schüler kann sich zwar
auch etwas vom Lehrer abschauen und auf der differenziellen Ebene etwas
Unausgeprochenes aus dem Unterricht mitnehmen, aber der Lehrer alleine kann gar
nicht soviel anbieten wie eine Gruppe von Mitschülern. Gerade für Schüler, die
ein Problem mit der Hierarchie haben, ist möglicherweise der Einzelunterricht
nicht das optimale Lernumfeld. Andere Schüler hingegen genießen es, endlich
einmal die volle Aufmerksamkeit einer Erwachsenen Person und gegebenenfalls
sogar Vertrauensperson zu bekommen. Diese Erfahrung habe ich besonders bei
meinen Internatsschülern gemacht, sowie bei Instrumentalschülern, die zu Hause
viele Geschwister haben. 

\subsection{Gruppenunterricht} 
Beim Gruppenunterricht muss man zwischen Klein- und Großgruppen unterscheiden.
In einem Kleingruppenunterricht sprechen wir von zwei bis fünf Schülern, während
ein Großgruppenunterricht alles darüber hinaus beschreibt.
\autocite[219]{busch:grundwissen_instrumentalpaedagogik} Laut Anselm Ernst gilt
ein Gruppenunterricht ab drei oder mehr Schülern, da man erst ab drei Schülern
von einer Gruppe sprechen kann.
\autocite[79]{ernst:die_zukunftsfaehige_musikschule}
Allerdings gibt es zusätzlich noch das Szenario von zwei Schülern, die gemeinsam
Unterricht haben und somit weder dem Einzel-, Klein- noch dem Gruppenunterricht
zugeordnet werden können. Hier sprechen wir also dann vom
Partnerunterricht.\autocite[219]{busch:grundwissen_instrumentalpaedagogik} Bei
zwei Schülern dominiert der Lehrer trotz allem noch das Geschehen, aber nichts
desto trotz können die Schüler gemeinsam Duos spielen, ohne, dass die Lehrperson
direkt als Duo-Partner involviert ist, wie im (EU). "Während im
Großgruppenunterricht und somit auch im Klassenmusizieren bzw. Klassenunterricht
dem Erwerb spieltechnisch-musikalischer Fertigkeiten Priorität zukommt, ist das
Spielen im Ensemble hiervon anzugrenzen, da dies das gemeinschaftliche
Musizieren fokussiert und in der Regel einen vorausgehenden oder parallel
stattfindenden Instrumentalunterricht
voraussetzt."\autocite[219]{busch:grundwissen_instrumentalpaedagogik}

Die Vorteile des Gruppenunterrichts sind sehr vielseitig. Abgesehen davon, dass
gemeinsames Musizieren einfach mehr Spaß macht und da kann ich aus Erfahrung von
meinen Schülern sprechen, die förmlich danach lechzen, wieder gemeinsam zu
musizieren. Ein weiterer Aspekt ist, dadurch, dass das soziale Lernen in den
Vordergrund rückt, verschmelzen das soziale und musikalische Lernen. Außerdem
ensteht bei den Schüler durch das soziale Miteinander, woraus auch
Freundschaften entstehen können, eine stärkere Bindung zur Musik. Auch die
Übemotivation ist höher, da sich die Schüler vor den anderen zum Einen nicht
blamieren wollen und zum Anderen auch eine Verantwortung für das Mittragen des
Unterrichts übernehmen. Außerdem können die Schüler die Stücke gemeinsam Üben,
sodass sie auch beim Üben nicht alleingelassen sind und das Üben selbst auch
zusammen Üben lernen. In einem Alter, wo das Instrumentalspiel vermeintlich
uncool werden könnte, lernen die Schüler aber auch in der Gemeinschaft, dass es
auch andere Personen gibt, die ein Instrument spielen wodurch sich die Schüler
in ihrer musikalischen Bestätigung gegenseitig bestärken.
\autocite{ernst:die_zukunftsfaehige_musikschule}

Im Vergleich zum Musikunterrich in der Schule oder auch dem sogenannten
Klassenmusizieren, muss man außerdem den Vorteil sehen, dass die
Instrumentallehrer sehr flexibel sind in ihrer Gestaltung, da sie sich an keine
Lehrpläne halten müssen. Dies gilt besinders, wenn das Musizieren im Rahmen von
der Musikschule stattfindet. Für diese Freiheit sollten die Instrumentallehrer
dankbar sein.



\subsection{Klassenmusizieren}
Eine Form des Gruppenunterrichtes in Großgruppen stellt das Klassenmusizieren
dar. Es findet oft in sogenannten Bläser- oder Streicherklassen statt, sodass
die Schüler je nach Angebot der Schule auf die Familie der Streich- oder eben
Blasinstrumente festgelegt werden. Das Klassenmusizieren ist allderings nicht zu
verwechseln mit dem normalen Musikunterricht, welcher trotzdem noch abgehalten
wird. Im Idealfall läuft es so ab, dass die Schüler die verschiedenen
Instrumente die zur Auswahl stehen kennen und anschließend eines auswählen
dürfen. Ziel ist es auch hier wieder, dass die Kinder im gemeinsamen Musizieren,
zu dem auch das Singen, Solmisieren und Rhythmussspielchen gehören, einen
ganzheitlichen, praktischen Weg zur Musik finden.
\autocite[91]{ernst:die_zukunftsfaehige_musikschule} Dadurch, dass das
Klassenmusizieren im normalen Unterrichtsalltag integriert wird, lernen die
Schüler das Instrument, wie jedes andere Fach, welches sie unterrichtet bekommen
als selbstverständlich kennen. Dadurch, dass es im Stundenplan integriert ist,
ist es kein besonderes außerschulisches Angebot, welches nur von wohlhabenden
Eltern ermöglicht werden kann und für jeden zugänglich. So haben die Schüler
auch die Gelegenheit, die Musik wie eine weitere Sprache dazuzulernen. Im
Unterschied zum beispielsweise Englisch-Unterricht, sind sie aber nicht auf sich
alleine gestellt, vor allem dann, wenn sie bei Prüfungen ihre eigene Leistung
unter Beweis stellen müssen. "Beim instrumentalen Klassenunterricht jedoch
verwandelt sich die Klasse in der Schüler/innen in ein Ensemble. Sie lernen
gemeinsam und musizieren gemeinsam."
\autocite[92]{ernst:die_zukunftsfaehige_musikschule} Wie Anselm Ernst außerdem
sehr zutreffend beschreibt, verbinden sich die Einzelleistungen und addieren
sich nicht. Dass die Schüler sich als Gemeinschaft erleben ist der zentrale Wert
dieses Musizierunterrichts!

"Es wächst die Zahl der Kinder, die über Formen des Klassenmusizierens an
Grundschulen (...)) gewonnen werden." %/S. 118 welches buch? evtl. Doerne?/


JeKids: meine Annahme, dass es Quatsch ist, einem Schüler im Klassenmusizieren
ein Isntrument beizubringen.




\subsection{Multidimensionaler Unterricht} 
Das Konzeot des Multidimensionalen Unterrichts weitet die Unterrichtsformen in
alle denkbaren Strukturen aus. Das Konzept stammt von Gerhard Wolters und
beschreibt eine gute Möglichkeit, wie wir in unseren Musikschulen den
Multidimensionalen Unterricht als eine wunderbare Ergänzung zum Einzelunterricht
etablieren könnten. Er bezieht sieben Dimensionen ein, die sehr unterschiedliche
Richtungen einbeziehen. \autocite[86ff]{ernst:die_zukunftsfaehige_musikschule}
Damit ist der zuvor beschriebene Gruppenunterricht gemeint. Wolters spricht von
der Partnerschaft zwischen zwei und mehr Personen. Hier steht das Voneinander
Lernen im Vordergrung. Die zweite Dimension bezieht sich auf den Unterricht in
mehreren Räumen. Das bedeutet aber auch, dass die Lehrperson nicht immer
anwesend sein muss, während eines Unterrichtes. Egal ob es sich um eine Gruppe
oder einen Einzelunterricht handelt. Sehr revolutionär ist auch der Gedanke der
flexiblen Unterrichtszeiten. "Flexible (längere oder kürzere) Unterrichtszeiten
passen sich an die individuellen Lernbereitschaften, Lernrhythmen, Lerntempi
usw. an." \autocite[87]{ernst:die_zukunftsfaehige_musikschule} Außerdem hat es
den positiven Nebeneffekt, dass der Instrumentalschüler am Ende sogar längere
Unterrichtszeiten hat, als in dem gewohnten EU. Eine neue Dimension ist der
Unterricht mit mehreren Lerhkräften. Hier können die Interaktionen der
Lehrkräfte untereinander angeregt werden, was das Unterrichten für die
Instrumentallehrer durch die gegenseitigen Anregungen interessanter gestaltet.
Und auch die Schüler profitieren davon, da jede Lehrkraft seine eigenen Stärken
mitbringt und die Schüler direkt von verschiedenen Stärken profitieren können.
Was es in Kinderorchestern bereits gibt, wird in Gruppenunterrichten, wenn sie
denn angeboten werden, sehr selten praktiziert, ist das Lernen mit Partnern
unterschieldichen Alters. "Intelektuell, körperlich, emotional und sozial sind
gleichaltrige Kinder/Jugendliche sher unterschiedlich entwickelt, so dass auch
in dieser Dimension Flexibilität erforderlich ist.
\autocite[87]{ernst:die_zukunftsfaehige_musikschule} Die sechste Dimension
schließt sich dem unterschiedlichen Alter an, denn Wolters schreibt, dass auch
Unterricht von Schülern in unterschiedlichen Lernniveaus stattfinden soll. Zum
einen enspricht es der Lebensrealtität und zum anderen kann man die Schüler in
diesem Zusammenhang wunderbar dazu anleiten, sich gegenseitig zu helfen, ihnen
beibringen für einander ein Vorbild zu sein, sich gegenseitig zu motivieren und
auch Verantwortung zu übernehmen. "Von Gleichaltrigen oder Älteren lernen
Kinder/Jugendliche oft besser als von der Lehrperson."
\autocite[87]{ernst:die_zukunftsfaehige_musikschule} Die letzte Dimension
beschreibt den Unterricht mit verschiedenen Instrumenten. Dadurch "öffnet sich
der Gruppenunterricht zum Ensembleunterricht."
\autocite[87]{ernst:die_zukunftsfaehige_musikschule}
