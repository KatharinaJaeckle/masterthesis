\section{Instrumentalunterricht}

Jeder Mensch, der ein Instrument erlernt, erhält in seinem Leben früher oder
später Instrumentalunterricht, was in Deutschland erst einmal Einzelunterricht
bedeutet. Der Instrumentalunterricht bietet aber viele verschiedene Formen vom
Einzelunterricht, über den Gruppenunterricht, bis hin zu dem sogenannten
\emph{Multidimensionalen Unterricht}. Man könnte auch soweit gehen, dass Band-
Chor- oder Orchesterproben mit einem Lehrer zu dem normalen
Instrumentalunterricht zählen. Als Erstes verknüpfen wir im europäischen System
jedoch mit dem klassischen Instrumentalunterricht den Einzelunterricht, wo eine
Lehrkraft eine/einen einzelnen Schüler*in im Normalfall zwischen dreißig und
neunzig Minuten unterrichtet. Dazu gehe ich im Folgenden ein. Was den
Gruppenunterricht ausmacht, wird auch näher beleuchtet werden. Fest steht auf
jeden Fall, dass im Idealfall das Lehren und Lernen flexibel gestaltet werden
und sich im besten Fall der Einzelunterricht und Gruppenunterricht ergänzen. In
unserem Musikschulsystem ist leider noch zu wenig Raum für die Kombination aus
individuellem- und gemeinschaftlichen Lernen. So schreibt Anselm Ernst in seinem
Buch \emph{Die zukunftsfähige Musikschule}: \enquote{Wer genügend Raum und Zeit hat,
kann sich mit Ruhe, Geduld und Gelassenheit den Menschen und den Sachen widmen.}
\autocite[84]{ernst:die_zukunftsfaehige_musikschule} Bei allen
Unterrichtsformen, so unterschiedlich sie sein mögen und mit ihren jeweiligen
Vor- und Nachteilen soll es darum gehen, dass der/die Instrumentalschüler*in das
für ihn am besten geeignete Lernumfeld geboten bekommt. Dazu sollte es auf alle
Fälle variabel sein und keine starre einseitige Unterrichtsform, sondern eine
Wechselwirkung aus allen Möglichkeiten sein, aus denen wir Musiker schöpfen
können. 



\subsection{Einzelunterricht}

Im Einzelunterricht erhält der/die Instrumentalschüler*in die
volle Aufmerksamkeit der Lehrkraft. Dies hat seinen Vorteil darin, dass die
Lehrkraft auf der persönlicheren Ebene viel mehr auf den/die Schüler*in eingehen
kann, sowie spieltechnisch in einem auf den/die Schüler*in abgestimmten Tempo
arbeiten kann. Außerdem erlaubt der Einzelunterricht, tiefer und detaillierter
auf das ein oder andere Thema einzugehen, zum Beispiel in der Spieltechnik oder
Körperhaltung. Die volle Aufmerksamkeit der Lehrperson kann jedoch auch genauso
gut zum Nachteil werden, da der/die Schüler*in ausschließlich von der Lehrperson
seine Anweisungen erhält und der Lernweg aufgrunddessen eindimensional
stattfindet. Der/die Lernende kann sich zwar etwas von der Lehrperson abschauen
und etwas Unausgeprochenes aus dem Unterricht mitnehmen, aber die Lehrperson
alleine kann gar nicht soviel bieten, wie eine Gruppe von Mitschüler*innen.
Gerade für Schüler*innen, die ein Problem mit der Hierarchie haben, ist
möglicherweise der Einzelunterricht nicht das optimale Lernumfeld. Andere
Schüler hingegen genießen es, endlich einmal die volle Aufmerksamkeit einer
Erwachsenen Person und gegebenenfalls sogar Vertrauensperson zu bekommen. Diese
Erfahrung habe ich besonders bei meinen Internatsschüler*innen gemacht, die zu
Hause viele Geschwister haben. 



\subsection{Gruppenunterricht}

In den Musikschulen in Deutschland wird für Anfängergruppen oft ein
Gruppenunterricht angeboten, aber früher oder später haben
die Schüler*innen Einzelunterricht, oder aufgehört das Instrument zu spielen.
Beim Gruppenunterricht muss man zwischen Klein- und Großgruppen unterscheiden.
In einem Kleingruppenunterricht sprechen wir von zwei bis fünf Schülern, während
ein Großgruppenunterricht alles darüber hinaus beschreibt.
\autocite[219]{busch:grundwissen_instrumentalpaedagogik} Laut Anselm Ernst gilt
ein Gruppenunterricht ab drei oder mehr Schülern, da man erst ab drei Schülern
von einer Gruppe sprechen kann.
\autocite[79]{ernst:die_zukunftsfaehige_musikschule}
Allerdings gibt es zusätzlich noch das Szenario von zwei Schülern, die gemeinsam
Unterricht haben und somit weder dem Einzel-, Klein- noch dem Gruppenunterricht
zugeordnet werden können. Dies wird dann als Partnerunterricht bezeichnet.
\autocite[219]{busch:grundwissen_instrumentalpaedagogik} Bei zwei Schüler*innen
dominiert der Lehrer trotz allem noch das Geschehen, aber nichts desto trotz
können die Schüler gemeinsam Duos spielen, ohne, dass die Lehrperson direkt als
Duo-Partner involviert ist, wie im Einzelunterricht. \enquote{Während im
Großgruppenunterricht und somit auch im Klassenmusizieren bzw. Klassenunterricht
dem Erwerb spieltechnisch-musikalischer Fertigkeiten Priorität zukommt, ist das
Spielen im Ensemble hiervon anzugrenzen, da dies das gemeinschaftliche
Musizieren fokussiert und in der Regel einen vorausgehenden oder parallel
stattfindenden Instrumentalunterricht
voraussetzt.}\autocite[219]{busch:grundwissen_instrumentalpaedagogik}

Aus meiner Erfahrung als Instrumentallehrkraft konnte ich beobachten dass die
Schüler*innen große Freude am gemeinsamen Musizieren finden und sich in Zeiten
von Corona – und auch in den Ferien – danach sehnen wieder gemeinsam zu musizieren.
Ein weiterer Aspekt ist, dadurch, dass das soziale Lernen in den Vordergrund
rückt, verschmelzen das soziale und musikalische Lernen. Außerdem entsteht bei
den Schüler*innen durch das soziale Miteinander, eine stärkere Bindung zur Musik
und darüber hinaus auch Freundschaften.
Außerdem können die Schüler*innen die Stücke gemeinsam Üben, sodass sie auch
beim Üben nicht alleingelassen werden und das Üben selbst auch gemeinsam lernen.
In einem Alter, wo das Instrumentalspiel vermeintlich \emph{uncool} werden
könnte, lernen die Schüler*innen aber in der Gemeinschaft, dass es auch andere
Personen gibt, die ein Instrument spielen, wodurch sich der/die Schüler*in in
ihrer musikalischen Betätigung gegenseitig bestärken.
\autocite{ernst:die_zukunftsfaehige_musikschule}

Im Vergleich zum Musikunterrich in der Schule oder auch dem sogenannten
Klassenmusizieren, muss man außerdem den Vorteil sehen, dass die
Instrumentallehrer*innen sehr flexibel sind in ihrer Gestaltung, da sie sich an
keine Lehrpläne halten müssen. Dies gilt besonders, wenn das Musizieren im
Rahmen von der Musikschule stattfindet.



\subsection{Klassenmusizieren}

Eine Form des Gruppenunterrichts in Großgruppen stellt das Klassenmusizieren
dar. Es findet oft in sogenannten Bläser- oder Streicherklassen statt, sodass
die Schüler*innen je nach Angebot der Schule auf die Familie der Streich- oder
eben Blasinstrumente festgelegt werden. Das Klassenmusizieren ist allderings
nicht zu verwechseln mit dem normalen Musikunterricht, der trotzdem noch
abgehalten wird. Im Idealfall läuft es so ab, dass die Schüler*innen die
verschiedenen Instrumente, die zur Auswahl stehen, kennen und anschließend eines
auswählen dürfen. Ziel ist es auch hier wieder, dass die Kinder im gemeinsamen
Musizieren – zu dem auch das Singen, Solmisieren und Rhythmusspiele gehören –
einen ganzheitlichen, praktischen Weg zur Musik finden.
\autocite[91]{ernst:die_zukunftsfaehige_musikschule} Dadurch, dass das
Klassenmusizieren im normalen Unterrichtsalltag integriert wird, lernen die
Schüler*innen das Musizieren und das Instrument, wie jedes andere
Unterrichtsfach, als selbstverständlich kennen. Dadurch, dass es im Stundenplan
integriert ist, ist es kein besonderes außerschulisches Angebot, das nur von
wohlhabenden Eltern ermöglicht werden kann, sondern für Jeden zugänglich. So
haben die Schüler*innen die Gelegenheit, die Musik wie eine weitere Sprache
dazuzulernen. Zum Beispiel im Unterscheid zu Englisch-Unterricht, sind sie
aber nicht auf sich alleine gestellt, vor allem dann, wenn sie bei Prüfungen
ihre eigene Leistung unter Beweis stellen müssen. \enquote{Beim instrumentalen
Klassenunterricht jedoch verwandelt sich die Klasse der Schüler*innen in ein
Ensemble. Sie lernen gemeinsam und musizieren gemeinsam.}
\autocite[92]{ernst:die_zukunftsfaehige_musikschule} Wie Anselm Ernst außerdem
sehr zutreffend beschreibt, verbinden sich die Einzelleistungen und addieren
sich nicht. Dass die Schüler*innen sich als Gemeinschaft erleben ist der
zentrale Wert dieses Musizierunterrichts.

%"Es wächst die Zahl der Kinder, die über Formen des Klassenmusizierens an
%%Grundschulen (…)) gewonnen werden." %/S. 118 welches buch? evtl.
%Doerne?-nope/ anselm ernst?

%Das Klassenmusizieren lehnt sehr stakr an das Model von JeKits an, oder auch
%umgekhert. 


%JeKids: meine Annahme, dass es Quatsch ist, einem Schüler im Klassenmusizieren
%ein Isntrument beizubringen.




\subsection{Multidimensionaler Unterricht} 

Das Konzept des Multidimensionalen Unterrichts weitet die Unterrichtsformen in
alle denkbaren Strukturen aus und ist derzeit noch nicht so weit in Deutschland
verbreitet. Das Konzept stammt von Gerhard Wolters und beschreibt eine gute
Möglichkeit, wie wir in unseren Musikschulen den Multidimensionalen Unterricht
als eine wunderbare Ergänzung zum Einzelunterricht etablieren könnten. Er
bezieht sieben Dimensionen ein, die sehr unterschiedliche Richtungen beinhalten.
\autocite[86ff]{ernst:die_zukunftsfaehige_musikschule} Wolters spricht von der
Partnerschaft zwischen zwei und mehr Personen. Hier steht das \emph{voneinander Lernen}
im Vordergrung. Die zweite Dimension bezieht sich auf den Unterricht in mehreren
Räumen. Das bedeutet aber auch, dass die Lehrperson nicht immer anwesend sein
muss, während eines Unterrichtes. Egal ob es sich um eine Gruppe oder einen
Einzelunterricht handelt. Sehr revolutionär ist auch der Gedanke der flexiblen
Unterrichtszeiten. \enquote{Flexible (längere oder kürzere) Unterrichtszeiten
passen sich an die individuellen Lernbereitschaften, Lernrhythmen, Lerntempi
usw. an.} \autocite[87]{ernst:die_zukunftsfaehige_musikschule} Außerdem hat es
den positiven Nebeneffekt, dass der/die Instrumentalschüler*in am Ende sogar
längere Unterrichtszeiten hat, als in dem gewohnten Einzelunterricht. Eine neue
Dimension ist der Unterricht mit mehreren Lerhkräften. Hier können die
Interaktionen der Lehrkräfte untereinander angeregt werden, was das Unterrichten
für die Instrumentallehrer*innen durch die gegenseitigen Anregungen
interessanter gestaltet. Auch die Schüler*innen profitieren davon, da jede
Lehrkraft ihre eigenen Stärken mitbringt und die Schüler*innen von verschiedenen
Stärken profitieren können. Was es in Kinderorchestern bereits gibt, wird in
Gruppenunterrichten, wenn sie denn angeboten werden, sehr selten praktiziert,
ist das Lernen mit Partnern unterschiedlichen Alters. \enquote{Intelektuell,
körperlich, emotional und sozial sind gleichaltrige Kinder/Jugendliche sehr
unterschiedlich entwickelt, so dass auch in dieser Dimension Flexibilität
erforderlich ist.} \autocite[87]{ernst:die_zukunftsfaehige_musikschule} Die
sechste Dimension schließt sich dem unterschiedlichen Alter an, denn Wolters
schreibt, dass auch Unterricht von Schüler*innen mit unterschiedlichen
Lernniveaus stattfinden soll. Zum Einen entspricht es der Lebensrealtität und
zum Anderen kann man die Schüler*innen in diesem Zusammenhang dazu
anleiten, sich gegenseitig zu helfen, ihnen beibringen füreinander ein Vorbild
zu sein, sich gegenseitig zu motivieren und auch Verantwortung zu übernehmen.
\enquote{Von Gleichaltrigen oder Älteren lernen Kinder/Jugendliche oft besser
als von der Lehrperson.} \autocite[87]{ernst:die_zukunftsfaehige_musikschule}
Die letzte Dimension beschreibt den Unterricht mit verschiedenen Instrumenten.
Dadurch
\enquote{öffnet sich der Gruppenunterricht zum Ensembleunterricht.}
\autocite[87]{ernst:die_zukunftsfaehige_musikschule}

