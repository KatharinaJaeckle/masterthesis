\section{Musikalische Früherziehung}

Das Arbeiten mit Gruppen ist ein grundlegendes Merkmal der Elementaren
Musikpädagogik (EMP). Wir können aus der EMP auch für andere Musiziergruppen einige
wichtige Bausteine ableiten, zumal sich das Elementare Musizieren an Menschen
jeden Alters wendet. \autocite[226]{Busch:grundwissen_Instrumentalpädagogik} Oft
wird sie jedoch besonders für die musikalische Früherziehung genutzt und richtet
sich bei vielen Angeboten an junge Familien oder Kindergartenkinder. Es ist sehr wichtig, dass die jüngste Generation aus
unserer Gesellschaft schon im frühen Alter die Gruppenprozesse und die Musik kennenlernt, da
sie auch in ihren sehr jungen Jahren schon sehr aufnahmefähig sind. Wieviel sie
sich in den ersten Lebensjahren aneignen zeigt sich auch darin, dass sie
ungefähr im Alter von vier Jahren der Muttersprache mächtig sind. "Auf die Musik übertragen sind sich alle großen
Musikpädagogen von Zoltan Kodaly bis Shinichi Suzuki einig, dass ein Kind
unbedingt so früh wie möglich mit der Musik in Kontakt kommen muss. "In diesem
Sinne gilt es, ein Kontinuum der der, musikalischen Muttersprache zu
entwickeln." \autocite[45]{ernst:die_zukunftsfaehige_musikschule} Nicht nur
für die Gruppenprozesse, sondern vor allem auch für das natürliche heranführen an
die Musik, ist die Musikalische Früherziehung besonders wichtig.
Der Ort dieser musikalischen Basisbildung sollte nicht erst in der
Musikschule sein, sondern bereits im Kindergarten. \autocite[43]{ernst:die_zukunftsfaehige_musikschule}Der Vorteil ist, dass das
Musizieren im
Kindergarten zum einen für jedes Kind zugänglich wäre und zum anderen aber auch
im Alltag integriert stattfindet. Im Anschluss können die Schulkinder außerdem
durch eine bessere Orientierung an den Instrumenten, dasjenige auswählen,
welches ihnen am meisten zusagt. Diese Orientierungsphase ist nämlich nicht zu
unterschätzen und besonders heutzutage wichtig, wo die Kinder nicht mehr
vielfältig musizierend, sondern einförmig konsumierend mit der Musik aufwachsen.\autocite[37]{ernst:die_zukunftsfaehige_musikschule}

Neben dem Umgang mit Gruppen werden im Folgenden die weiteren Grundgedanken der EMP
dargelegt. Ganz vorne weg ist die wichtige entwicklungspsychologische Erkenntnis
nach Oerter und Lehmann zu betonen, die besagt, dass jeder Mensch musikalisch ist.\autocite[88]{musikalische_begabung}
Geht man von diesem Grundgedanken aus, so soll die EMP ermöglichen, dieses musikalische Potenzial zu
entwickeln, wobei die Entfaltung der musikalischen Ausdrucksfähigkeit die
Persönlichkeitsbildung des Menschen unterstützt. Das gelingt durch das musikalische Gestalten
wozu das verschiedene Handlungsbereiche zählen, wie das sensorische
Sensibilisieren, die Bewegung, das Singen, Sprechen, das Lernen die Zeit zu
strukturieren, (Elementar-  und Körper-)Instrumente spielen, das Visualisieren
sowie das Hören. \autocite[227]{Busch:grundwissen_Instrumentalpädagogik} Als
grundlegendens Musikverständnis bilden in der EMP Musik und Bewegung stets eine
Einheit.

\subsection{Aktionsformen in der Gruppe und Persönlichkeitsentwicklung}
Die Sozialform der EMP ist wie eingangs erwähnt immer die Gruppe, die aber
wiederum verschiedene Spielräume für verschiedene Aktionsformen ermöglicht. So
kann es Phasen geben, in denen jeder für sich tätig ist, Partnerarbeit oder die
Arbeit in der Gruppe als Ganzes. Alle Formen basieren auf dem beziehungsorientierten
Arbeiten, in dem "über die Beziehungen, die zur Musik aufgebaut werden sollen,
[...] auch die Kontakte mit anderen Gruppenmitgliedern im Fokus."
\autocite[10]{dartsch:kern_des_musizierens} stehen. Mit dem ganzheitliche Menschen im
Blick soll neben der angestrebten Persönlichkeitsbildung die soziale
Kommunikation ausgebaut werden.


"Dass viele dieser Kinder gut mit Musik kommunizieren können und damit sowohl im
oft notwendigen Gruppenunterricht, aber auch bei musikalischen Dialogen mit dem
Instrumentallehrer ausdrucksfähiger sind." S. 25

"(...) sind viele der damit verbundenen Techniken und Methoden für den
"Ernsthaften" Instrumentalunterricht durchaus hilfreich." S. 25
Anselm Ernst S. 37 ff

"Musikalische Kommunikation durchzog (früher) den ganzen Alltag."
\autocite[37]{ernst:die_zukunftsfaehige_musikschule}


"Für die Musikschule reklamieren wir den Erziehungs- und Bildungsauftrag, in
allen Schichten der Gesellschaft so umfassend wie möglich eine Musizierkultur
aufzubauen" (S. 37/38) Anselm Ernst

"Aus meiner Sicht liegt das wichtigste Ergebnis in der Motivation ein Instrument
zu erlernen und damit auf dem Weg des eigenständigen Musizierens
weiterzuschreiten. Ein fast ebenso wesentlicher Erfolg ist gegebenm wenn die
Kinder eine ausgeprägt positive Einstellung zum gemeinsamen Musizieren und
Lernen gewinnen." \autocite[40]{ernst:die_zukunftsfaehige_musikschule}

