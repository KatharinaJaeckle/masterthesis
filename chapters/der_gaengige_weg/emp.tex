\section{Musikalische Früherziehung}

Das Arbeiten mit Gruppen ist ein grundlegendes Merkmal der Elementaren
Musikpädagogik (EMP). Wir können aus der EMP auch für andere Musiziergruppen
einige wichtige Bausteine ableiten, zumal sich das Elementare Musizieren an
Menschen jeden Alters wendet.
\autocite[226]{busch:grundwissen_instrumentalpaedagogik} Oft wird sie jedoch
besonders für die musikalische Früherziehung genutzt und richtet sich bei vielen
Angeboten an junge Familien oder Kindergartenkinder, ähnlich wie bei Suzukis
Philosophie. Es ist  für die musikalische, soziale und psychische Entwicklung
für die jüngste Generation unserer Gesellschaft von großer Bedeutun schon im
frühen Alter die Gruppenprozesse sowie das Musizieren kennen zu lernen. Wieviel sie sich in den
ersten Lebensjahren aneignen zeigt sich alleine darin, dass sie ungefähr im
Alter von vier Jahren der Muttersprache mächtig sind. Auf die Musik übertragen
sind sich alle großen Musikpädagogen von Zoltan Kodaly bis Shinichi Suzuki
einig, dass ein Kind unbedingt so früh wie möglich mit der Musik in Kontakt
kommen muss. \enquote{In diesem Sinne gilt es, ein Kontinuum der musikalischen
Muttersprache zu entwickeln.}
\autocite[45]{ernst:die_zukunftsfaehige_musikschule} Nicht nur für die
Gruppenprozesse, sondern vor allem auch für das natürliche heranführen an die
Musik, ist die Musikalische Früherziehung besonders wichtig. Der Ort dieser
musikalischen Basisbildung sollte nicht erst in der Musikschule sein, sondern
bereits im Kindergarten oder vorher in Krabbelgruppen o.Ä.
\autocite[43]{ernst:die_zukunftsfaehige_musikschule}
Das Musizieren im Kindergarten ist zum Einen für jedes Kind zugänglich und
findet zum Anderen somit auch im Alltag integriert statt. Im Anschluss können
die Schulkinder außerdem durch eine bessere Orientierung über die Instrumente,
das Instrument auswählen, das ihnen am meisten zusagt. Diese Orientierungsphase
ist nicht zu unterschätzen und besonders heutzutage wichtig, wo die Kinder nicht
mehr vielfältig musizierend, sondern einförmig konsumierend mit der Musik
aufwachsen.\autocite[37]{ernst:die_zukunftsfaehige_musikschule}

Neben dem Umgang mit Gruppen werden im Folgenden die weiteren Grundgedanken der
EMP dargelegt. Die wichtige entwicklungspsychologische Erkenntnis nach Oerter
und Lehmann besagt, dass jeder Mensch musikalisch
ist.\autocite[88]{musikalische_begabung} Geht man von diesem Grundgedanken aus,
so soll die EMP ermöglichen, dieses musikalische Potenzial zu entwickeln, wobei
die Entfaltung der musikalischen Ausdrucksfähigkeit die Persönlichkeitsbildung
des Menschen unterstützt. Das gelingt durch das musikalische Gestalten, wozu
verschiedene Handlungsbereiche zählen, wie das sensorische Sensibilisieren, die
Bewegung, das Singen, Sprechen, das Lernen die Zeit zu strukturieren,
(Elementar-  und Körper-)Instrumente spielen, das Visualisieren sowie das Hören.
\autocite[227]{busch:grundwissen_instrumentalpaedagogik} Als grundlegendes
Musikverständnis bilden in der EMP Musik und Bewegung stets eine Einheit.

Zwei übergeordnete Erziehungsziele sind es, die psychologischen und sozialen
Ziele zu schulen. Was die beiden verbindet und vereint ist die Entwicklung von
Selbstwahrnehmung und sozialer Beziehungsfähigkeit, durch die zunehmend die
Ich-Identität ausgebildet wird. Zu den psychologischen Zielen gehört es,
Geborgenheit zu erfahren und die Selbstwahrnehmung zu fördern. Außerdem fördert
die EMP die \emph{Ich-Erweiterung} durch Identifikationsprozesse mit
verschiedenen Lebewesen, Naturerscheinungen und unbelebten Objekten. Es geht des
weiteren bei den psychologischen Zielen darum, Ideen zu entwickeln und
persönlich herausgefordert zu werden, in dem sich die Schüler*innen präsentieren
müssen und dadurch Unsicherheiten entstehen können. Es gilt zu Lernen Spannungen
auszuhalten, das Durchhaltevermögen zu steigern und Befriedigungsaufschub
ertragen zu lernen. Ein weiterer Aspekt ist es, die Initiative zu ergreifen und
Entscheidungen zu fällen, Beziehungen zur Gruppe oder einem Partner aufzubauen
und als letzten Aspekt auch das Zurücktreten und Verzichten und Selbstkontrolle
zu lernen. Dies sind alles psychologische Effekte, die im Umgang mit der Gruppe
besonders geschult und gelernt werden können; besser als im Einzel-
Instrumentalunterricht. Und doch beziehen sie sich sehr stark auf das Individuum
selbst.

Die sozialen Aspekte hingegen beziehen sich, wie schon erwähnt, auf die Gruppe.
Hier geht es darum zu lernen, die Anderen wahrzunhemen. Dazu zählt auch
aufeinander zu reagieren, in Kommunikation zueinander zu treten und aufeinander
Rücksicht zu nehmen. Das Abwarten und eine Reihenfolge einhalten, wird auch
geschult, indem sich die Gruppe selbst regulieren lernt und ein Gruppengefühl
entwickelt wird. Das funktioniert wiederum nur, wenn die Schüler*innen anfangen
sich mitverantwortlich zu fühlen und auch Ideen Anderer akzeptieren.

Um sich diesen Zielen zu nähern, geht die EMP spielerich mit Individual- und
Kollektivgestaltung um und versucht durch vielfältige Aktionsformen verschiedene
Interaktionen zu ermöglichen. Dadurch geschieht die soziale Sensibilisierung, zu
der noch die auditive Sensibilisierung hinzukommt. 

Die Sozialform der EMP ist wie eingangs erwähnt immer die Gruppe, die aber
wiederum verschiedene Spielräume für verschiedene Aktionsformen ermöglicht. Alle
Formen basieren auf dem beziehungsorientierten Arbeiten, in dem \enquote{über
die Beziehungen, die zur Musik aufgebaut werden sollen, [...] auch die Kontakte
mit anderen Gruppenmitgliedern im Fokus.}
\autocite[10]{dartsch:kern_des_musizierens} stehen. Mit dem ganzheitlichen
Menschen im Blick soll neben der angestrebten Persönlichkeitsbildung die soziale
Kommunikation ausgebaut werden. Fest steht, dass viele Kinder, die eine
musikalische Früherziehung miterlebt haben, gut mit Musik kommunizieren können
und damit sowohl im oft notwendigen Gruppenunterricht, aber auch bei
musikalischen Dialogen mit der Lehrkraft ausdrucksfähiger sind.


%S. 25%Andreas oder erstes Buch? s. 25


%"(...) sind viele der damit verbundenen Techniken und Methoden für den
%"Ernsthaften" Instrumentalunterricht durchaus hilfreich." S. 25 Anselm Ernst S.
%37 ff




%"Für die Musikschule reklamieren wir den Erziehungs- und Bildungsauftrag, in
%allen Schichten der Gesellschaft so umfassend wie möglich eine Musizierkultur
%aufzubauen" (S. 37/38) Anselm Ernst



