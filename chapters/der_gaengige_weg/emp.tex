\section{Musikalische Früherziehung}

Das Arbeiten mit Gruppen ist ein grundlegendes Merkmal der Elementaren
Musikpädagogik (EMP). Wir können aus der EMP auch für andere Musiziergruppen einige
wichtige Bausteine ableiten, auch wenn hier nicht die jungen Kinder im
Mittelpunkt stehen. Umso wichtiger ist es aber, dass die jüngste Generation aus
unserer Gesellschaft schon im frühen Alter die Gruppenprozesse und die Musik kennenlernen, da
sie auch in ihren sehr jungen Jahren schon sehr aufnahmefähig sind. Wieviel sie
sich in den ersten Lebensjahren aneignen zeigt sich auch darin, dass sie
ungefähr im Alter von vier Jahren der Muttersprache mächtig sind. "Auf die Musik übertragen sind sich alle großen
Musikpädagogen von Zoltan Kodaly bis Shinichi Suzuki einig, dass ein Kind
unbedingt so früh wie möglich mit der Musik in Kontakt kommen muss. "In diesem
Sinne gilt es, ein Kontinuum der der, musikalischen Muttersprache zu
entwickeln." \autocite[45]{ernst:die_zukunftsfaehige_musikschule} Nicht nur
für die Gruppenprozesse, sondern vor allem auch für das natürliche heranführen an
die Musik, ist die Musikalische Früherziehung besonders wichtig.
Allerdings sollte der Ort dieser musikalischen Basisbildung nicht erst in der
Musikschule sein, sondern bereits im Kindergarten. \autocite[43]{ernst:die_zukunftsfaehige_musikschule}Der Vorteil ist, dass das
Musizieren im
Kindergarten zum einen für jedes Kind zugänglich wäre und zum anderen aber auch
im Alltag integriert stattfindet. Im Anschluss können die Schulkinder außerdem
durch eine bessere Orientierung an den Instrumenten, dasjenige auswählen,
welches ihnen am meisten zusagt. Diese Orientierungsphase ist nämlich nicht zu
unterschätzen. 





"Dass viele dieser Kinder gut mit Musik kommunizieren können und damit sowohl im
oft notwendigen Gruppenunterricht, aber auch bei musikalischen Dialogen mit dem
Instrumentallehrer ausdrucksfähiger sind." S. 25

"(...) sind viele der damit verbundenen Techniken und Methoden für den
"Ernsthaften" Instrumentalunterricht durchaus hilfreich." S. 25

Anselm Ernst S. 37 ff

"Musikalische Kommunikation durchzog (früher) den ganzen Alltag."
\autocite[37]{ernst:die_zukunftsfaehige_musikschule}

"Kinder wachsen nicht mehr vielfältig musizierend, sondern einförmig
konsumierend in die Musik hinein" \autocite[37]{ernst:die_zukunftsfaehige_musikschule}

"Für die Musikschule reklamieren wir den Erziehungs- und Bildungsauftrag, in
allen Schichten der Gesellschaft so umfassend wie möglich eine Musizierkultur
aufzubauen" (S. 37/38) Anselm Ernst

"Aus meiner Sicht liegt das wichtigste Ergebnis in der Motivation ein Instrument
zu erlernen und damit auf dem Weg des eigenständigen Musizierens
weiterzuschreiten. Ein fast ebenso wesentlicher Erfolg ist gegebenm wenn die
Kinder eine ausgeprägt positive Einstellung zum gemeinsamen Musizieren und
Lernen gewinnen." \autocite[40]{ernst:die_zukunftsfaehige_musikschule}

