\section{Musikalische Früherziehung}

Mit vier Jahren ist ein Kind schon der Muttersprache mächtig, was wiederum
zeigt, dass die Kinder auch schon in den ersten vier Lebensjahren sehr
aufnahmefähig sind. Auf die Musik übertragen sind sich alle großen
Musikpädagogen von Zoltan Kodaly bis Shinichi Suzuki einig, dass ein Kind
unbedingt so früh wie möglich mit der Musik in Kontakt kommen muss. "In diesem
Sinne gilt es, ein Kontinuum der der ,musikalischen, Muttersprache zu
entwickeln." \autocite[45]{ernst:die_zukunftsfaehige_musikschule}

"Dass viele dieser Kinder gut mit Musik kommunizieren können und damit sowohl im
oft notwendigen Gruppenunterricht, aber auch bei musikalischen Dialogen mit dem
Instrumentallehrer ausdrucksfähiger sind." S. 25

"(...) sind viele der damit verbundenen Techniken und Methoden für den
"Ernsthaften" Instrumentalunterricht durchaus hilfreich." S. 25

Anselm Ernst S. 37 ff

"Musikalische Kommunikation durchzog (früher) den ganzen Alltag."
\autocite[37]{ernst:die_zukunftsfaehige_musikschule}

"Kinder wachsen nicht mehr vielfältig musizierend, sondern einförmig
konsumierend in die Musik hinein" \autocite[37]{ernst:die_zukunftsfaehige_musikschule}

"Für die Musikschule reklamieren wir den Erziehungs- und Bildungsauftrag, in
allen Schichten der Gesellschaft so umfassend wie möglich eine Musizierkultur
aufzubauen" (S. 37/38) Anselm Ernst

"Aus meiner Sicht liegt das wichtigste Ergebnis in der Motivation ein Instrument
zu erlernen und damit auf dem Weg des eigenständigen Musizierens
weiterzuschreiten. Ein fast ebenso wesentlicher Erfolg ist gegebenm wenn die
Kinder eine ausgeprägt positive Einstellung zum gemeinsamen Musizieren und
Lernen gewinnen." \autocite[40]{ernst:die_zukunftsfaehige_musikschule}


EMP super Sache -nur schon früher und zwar im Kindergarten. 
Außerdem spricht Ernst davon, dass der Ort dieser musikalischen Basisbildung
nicht erst in der Musikschule sein sollte, sondern bereits im Kindergarten. Dort
wäre es für jedes Kind zugänglich und würde bereits im früheren Alter in den
Alltag der Kinder integriert sein, sodass sie als Schulkinder dann eine Idee
haben, welches Instrument ihnen Spaß machen könnte. (s. 43)

-Orientierungsphase nach der sie entscheiden können, welches Instrument sie
selbst lernen wollen, ist nicht zu unterschätzen. 
