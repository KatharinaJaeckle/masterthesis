\section{Ensembles}

Es gibt einige Punkte, die alle Ensembles gemeinsam haben. So beudeutet das Wort
"Ensemble" zunächst einmal "eine Gruppe von Dingen, die eng
zusammengehören." (wikionary) Wenn man die Bedeutung weiter auf die Kunst
beziehen möchte, so steht in der Definition: "eine Gruppe von Künstlern -wie
Schauspieler, oder Musiker, die gemeinsam etwas vortragen."
(wikionary:ensemble) Das interessante an dieser Beschreibung ist, dass das Wort
"vortragen" impliziert, dass diese Gruppe von dem Ensemble auch auf der Bühne
oder in einem anderen öffentlichen Rahmen mit ihrem Ensemble auftreten und das
erarbeitete präsentieren. Betrachten wir den Einzel- Instrumentalunterricht, so
ist es nicht unbedingt selbstverständlich, dass der Schüler das Erarbeitete
öffentlich vorträgt. Gerade bei Privatschülern werden oft keine Vorspiele
durchgeführt, da sie ohne eine Institution im Hintergrund, die darauf Wert legt,
nicht immer dazu kommt, zumal auch nicht immer die räumlichen Möglichkeiten dafür
geschaffen sind. Wie sieht es jedoch mit dem Gruppenunterricht aus? Er findet
meist doch im Zusammenhang mit einer Institution statt, sei es dem Kindergarten, der
Schule oder einer Musikschule. Interessanterweise ist es dort, wenn auch
teilweise unausgepsrochen, selbstverständlich, dass die Gruppen bei einem
besonderen Anlass oder auch bei einem dafür geschaffenen Vorspiel, ihre Stücke
darbieten. 
Andreas Doerne beschreibt in seinem Buch \emph{Umfassend Musizieren} das Ensemble als
\enquote{eine musikalische Ganzehit, die sich
durch ein gleichberechtigtes, dialogisches Miteinander und Zusammenspiel ihrer
einzelnen Teile (Mitglieder) konstituiert.}
\autocite[62]{doerne:umfassend_musizieren} Diese Definition trifft aber meiner
Meinung nach (leider) nicht in immer in allen Punkten auf ein Ensemble zu. So
werden wir später feststellen, dass beispielsweise in einem Orchester der
Dirigent die Spielimpulse gibt und auch auch vorgibt, wie zu spielen ist,
weshalb es in diesem Moment kein gleichberechtigtes Miteinander mehr ist. Auch
das dialogische Miteinander ist nicht in jedem Ensemble erwünscht und
selbstverständlich. Wenn wir uns beispielweise die Suzuki-Talenterziehung
anschauen, werden wir sehr schnell feststellen, dass der Lehrer ganz deutlich in
der überordneten dominierenden Position ist, in der das dialogische Miteinander
auf einer sehr untergeorneten Ebene nur selten erwünscht ist. Nichts desto trotz
ist das "Ensemblespiel (ist) Kommunikation
mit musikalischen Körpern und Klängen." \autocite[62]{doerne:umfassend_musizieren}
Sich dies zu vergegenwärtigen ist sehr hilfreich, da man schnell feststellt
welche zentrale Rolle dabei die Kommunikation spielt. Die Kommunikation wiederum
ist sehr eng mit der Psyche eines jeden verknüpft, wenn er Sitautionen für sich
selbst auswertet und mit anderen abgleicht und in vielerlei anderer Hinsicht.
Die soziale Komponente ist meiner Meinung nach das Wesentliche Grundelement
eines Ensembles. Welche sozialen Eigenschaften innerhalbt eines Ensembles eine
Rolle spielen ist im folgenden Zitat sehr zutreffend geschildert:
"Das Ensemble ist durchdrungen vom gegenseitigen Zuhören und Mitteilen, Sehen
und Gesehenwerden, Agieren und Reagieren, Führen und Geführtwerden, Hervorbringen
und Abnehmen von Spielimpulsen, insgesamt also vom wechselseitigen Geben und
Nehmen." \autocite[62]{doerne:umfassend_musizieren} Diese sozialen Komponenten
finden wir auch in allen folgenden Formationen wieder. 


\subsection{Orchester} 

Ein Symphonieorchester ist, sofern man es überhaupt noch zu einem Ensemble
zählen kann, womöglich die größte Gruppierung von Musizierenden die
zusammenspielen. Aufgrund der Größe, trägt eine klare Hierarchie zur kleineren
Unterteilung innerhalb dieses großen Organismus. Auch aufgrund der Größe ist ein
Dirigent, bei dem vorne der Klang zusammenläuft auch eine wichtige
Schlüsselfigur. ... (%Andreas Doerne?!)

Orchester:
Aufmerksamkeit auf: den Dirigenten, auf die Gestaltung des eigenen Parts, auf
das Verfolgen der anderen Stimmen unda auf den Orchesterklang als Ganzes. (S. 63
Andreas Doerne)

- Hierarchie: Dirigent, Stimmführer, Tuttisten, einzelne Instrumente die
musikalisch viel beeinflussen, wie die Pauke

Orchestermöglichkeiten- und Laufbahnen von Laienmusikern:
Schulorchester, Kinderorchester, Jugendorchester, Studentenorchester, Laienorchester


\subsection{Band}
- Improvisation?


\subsection{Chor}
- technisch kann man sich nicht soviel abschauen, wie bei Band oder Orchester
- sich in den Klang einordnen


\subsection{Sonstige Formationen}
Die besondere Rolle des Klaviers
Denkt man die Instrumente die nicht für Ensembles bzw. Gruppenunterrichte geeignet sind, wie zum
Beispiel das Klavier oder die Gitarre, so kommen mir trotzdem und zum Glück doch
einige Gruppierungen aus meinem Umfeld in den Sinn, die sich über die gängigen
Formationen nicht abschrecken lassen und neugierig genug sind, sich ein Ensemble
zu suchen. Mit dem Klavier gibt es wenigstens noch die sparte der
Klavier-Kammermusik, von Klaviertrio bis hin zum Klavierquintett. Aber darüber
hinaus wird es eher wieder schwieriger und auch bei den jungen Klavierschülern,
ist das Ensemblespiel leider ein noch viel zu sehr vernachlässigter Bausteis.
Wenn man bedenkt welche ganzen sozialen und auch musikalischen Kompetenzen im
Ensemblspiel geschult und gelent werden, so muss ich sagen, dass es in Zukunft
unbedingt auch für Klavierschüler Möglichkeiten und Räume zum gemeinsamen
Musizieren geschaffen werden müssen! Betrachten wir aber die sparte der neuen
Musik, so stelle ich fest, dass gerade hier immer wieder untypische, neue und
besondere Besetzungen von Fagott über Klavier, Schlagzeug und beispielsweise
Geige immer gängiger werden und besonders beliebt sind. Allerdings sind die
Anforderungen an das Niveau in der Neuen Musik meist leider so hoch, dass sie
für Instrumentale Anfäger oft nicht spielbar sind. 
Bei dem Klavier kommt außerdem noch hinzu, dass man aufgrund der Größe des
Instrumentes leider kaum mehr als zwei Klaviere bzw. Flügel in einem Raum stehen
haben. 

Vergleichen wir das Klavier beispielsweise mit der Gitarre, die es in der
klassischen Musikszene als Ensembleinstrument auch nicht leicht hat, so kann man
wenigstens mit mehreren Schülern ein Gitarrenensemble zusammenstellen und
gemeinsam musizieren. Dass die Gitarristen aber auch Pianisten in den anderen Genres ihren Platz
meist in einer Band haben, vor allem auch im Populären Musikbereich, ist nicht
außer Acht zu lassen. 

Das Schlagzeug?





