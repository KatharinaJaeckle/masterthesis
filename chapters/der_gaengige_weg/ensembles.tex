\section{Ensembles}

Nach dem Einzelunterricht, manchmal erst im späteren Leben und manchmal schon in
der frühen Kindheit besuchen Instrumentalist*innen verschiedene Ensembles in
Ergänzung zum Instrumentalunterricht, oder auch unabhängig davon. Es gibt einige
Punkte, die alle Ensembles gemeinsam haben. So beudeutet das Wort
\emph{Ensemble} zunächst einmal \enquote{eine Gruppe von Dingen, die eng
zusammengehören.}\autocite{wikipedia:gruppe} Wenn man die Bedeutung weiter auf
die Kunst beziehen möchte, so steht in der Definition: \autocite{eine Gruppe von
Künstlern -wie Schauspieler, oder Musiker, die gemeinsam etwas vortragen.}
\autocite{wikipedia:gruppe} Das interessante an dieser Beschreibung ist, dass
das Wort 
\emph{vortragen}
impliziert, dass diese Gruppe auch auf der Bühne oder in einem anderen
öffentlichen Rahmen mit ihrem Ensemble auftritt und das Erarbeitete präsentiert.
Betrachten wir den Einzel- Instrumentalunterricht, so ist es nicht unbedingt
selbstverständlich, dass der/die Schüler*in das Erarbeitete öffentlich vorträgt.
Gerade bei Privatschüler*innen werden oft keine Vorspiele durchgeführt, da es
ohne eine Institution im Hintergrund, die darauf Wert legt, nicht immer dazu
kommt, zumal auch nicht immer die räumlichen Möglichkeiten dafür geschaffen
sind. Wie sieht es jedoch mit dem Gruppenunterricht aus? Er findet meist doch im
Zusammenhang mit einer Institution statt, sei es dem Kindergarten, der Schule
oder einer Musikschule. Interessanterweise ist es dort, wenn auch teilweise
unausgepsrochen, selbstverständlich, dass die Gruppen bei einem besonderen
Anlass oder auch bei einem dafür geschaffenen Vorspiel, ihre Stücke darbietet.
Andreas Doerne beschreibt in seinem Buch \emph{Umfassend Musizieren} das
Ensemble als \enquote{eine musikalische Ganzehit, die sich durch ein
gleichberechtigtes, dialogisches Miteinander und Zusammenspiel ihrer einzelnen
Teile (Mitglieder) konstituiert.} \autocite[62]{doerne:umfassend_musizieren}
Diese Definition trifft aber meiner Meinung nach nicht in allen Punkten auf ein
Ensemble zu. So werden wir später feststellen, dass beispielsweise in einem
Orchester der Dirigent die Spielimpulse gibt und auch auch vorgibt, wie zu
spielen ist, weshalb es in diesem Moment kein gleichberechtigtes Miteinander
mehr ist. Auch das dialogische Miteinander ist nicht in jedem Ensemble erwünscht
und selbstverständlich. Wenn wir uns beispielweise die Suzuki-Talenterziehung
anschauen, werden wir sehr schnell feststellen, dass die Lehrkraft ganz deutlich
in der überordneten dominierenden Position ist, in der das dialogische
Miteinander auf einer sehr untergeorneten Ebene nur selten erwünscht ist. Nichts
desto trotz ist das \enquote{Ensemblespiel (ist) Kommunikation mit musikalischen
Körpern und Klängen.} \autocite[62]{doerne:umfassend_musizieren} Sich dies zu
vergegenwärtigen ist sehr hilfreich, da man schnell feststellt welche zentrale
Rolle dabei die Kommunikation spielt. Die Kommunikation wiederum ist sehr eng
mit der Psyche eines jeden verknüpft, wenn er Sitautionen für sich selbst
auswertet und mit anderen abgleicht und in vielerlei anderer Hinsicht. Die
soziale Komponente ist meiner Meinung nach das wesentliche Grundelement eines
Ensembles. Welche sozialen Eigenschaften innerhalb eines Ensembles eine Rolle
spielen ist im folgenden Zitat sehr zutreffend geschildert: \enquote{Das
Ensemble ist durchdrungen vom gegenseitigen Zuhören und Mitteilen, Sehen und
Gesehenwerden, Agieren und Reagieren, Führen und Geführtwerden, Hervorbringen
und Abnehmen von Spielimpulsen, insgesamt also vom wechselseitigen Geben und
Nehmen.} \autocite[62]{doerne:umfassend_musizieren} Diese sozialen Komponenten
finden wir auch in allen folgenden Formationen wieder. Außerhalb des Musizierens
treffen sich alle Ensembles zu mehr oder weniger regelmäßigen Proben und
veranstalten auch Probenwochenenden, an denen alle über die Musik hinaus
Zusammenkommen. Bei größeren Formationen gibt es oft auch außerhalb der Musik
einen Vorstand, der für die organisatorischen Fragen verantwortlich ist und bei
Problemen versucht die Gruppe zusammen zu halten. Das ist fast vergleichbar mit
unserer Gesellschaft, in der sich das Ganze nur in anderen Dimensionen abspielt.


\subsection{Orchester} 

Ein Sinfonieorchester ist, sofern man es überhaupt noch zu einem Ensemble zählen
kann, womöglich die größte Gruppierung von Musizierenden die zusammenspielen.
Aufgrund der Größe, trägt eine klare Hierarchie zur kleineren Unterteilung
innerhalb dieses großen Organismus bei. Aufgrund der Größe ist ein Dirigent, bei
dem vorne der Klang zusammenläuft eine wichtige Schlüsselfigur des Orchesters,
da er die Musik zusammenführen und regulieren kann. Die einzelnen Spieler*innen
sind ja mit ihrer eigenen Stimme beschäftigt und sitzen teilweise sehr weit von
anderen Stimmen entfernt, sodass die Verantwortung bei dem/der Dirigent*in
liegt, dies zusammenzubringen. Sowohl im Timing als auch in der Dynamik. Das hat
zur Folge, dass die Aufmerksamkeit der einzelnen Spieler*innen neben den
Stimmführer*innen auf den/die Dirigent*in gerichtet ist, auf die Gestaltung des
eigenen Parts, auf das Verfolgen der Mitspieler*innen und auf den
Orchesterklang. \autocite[56]{doerne:umfassend_musizieren} Wie wir sehen, ist
die Aufmerksamkeit also auf sehr viele verschiedene Felder gelenkt, die sich in
ihren Schwerpunkten immer wieder verschiebt. Das ist für mich persönlich das
Spannende am Orchesterspiel, was so viel Freude bereitet und weshalb es einfach
nie langweilig wird und immer spannend ist!

Für Insrumentalist*innen aller Art und jeden Niveaus gibt es Schulorchester,
Kinderorchester, Jugendorchester, Studentenorchester und Laienorchester. Oft
sind diese jedoch in größeren Städten und in ländlicheren Gegenden nicht ganz so
weit verbreitet. Außerdem ist für ein Sinfonieorchester vorausgesetzt, dass man
ein Instrument erlernt hat, welches in der typischen Bestzung eines Orchesters
vorgesehen ist. 

\subsection{Musikverein}
Die Musikvereine sind hingegen oft vermehrt in ländlicheren Regionen zu Hause.
Es gibt bei ihnen drei verschiedene Instrumental-Kategorien. Zum einen die
Akkordeonvereine, die Blasmusikvereine, sowieo die Fasnachtsgruppierungen. In
den Akkordeonvereinen spielen zwar alle erst einmal das selbe Instrument, aber
doch in ihren unterschiedlichen Stimmen und auch leicht unterschiedlich in der
Funktionsweise, da die Bässe beispielweise verstärkt sind. Manchmal gibt es noch
Trompeter und Schlagzeuger dazu, aber das ist eher die Ausnahme. Bei den
Fasnachtsgruppen, kommen die Angehörigen einer Zunft zusammen und musizieren
gemeinsam. Die Hauptinstrumente hierbei sind Trommeln, Piccolo-Flöten und
Blechblasinstrumente. Sie arbeiten meist nur einmal im Jahr und auf den
Fasnachts-Anlass hin. Im Blasmusikverein sind ähnlich wie im Orchester
verschiedene Instrumentengruppen vertreten aus der sich heraus eine klare
Hierarchie innerhalb der Gruppen bildet. Außerdem gibt es auch eine/n
Dirigent*in, welcher die Proben anleitet. Im Vergleich zum Orchester sind aber
nur Blasinstrumente, die gespielt werden und ein Schlagzeug zuzüglich. Das
Schöne am Musikverein ist die heterogene Zusammensetzung der Mispieler*innen.
Hier kommen Jung und Alt zusammen, Anfänger und Fortgeschrittene musizieren
gemeinsam. Außer den Landespolizeiorchestern oder beim Militär, gibt es bei
Blasmusikvereinen fast nur Laiengruppierungen. Dadurch, dass also viele
Musiker*innen in ihrer Freizeit im Verein spielen, gehört neben dem eigentlichen
Musizieren auch das Vereinsleben dazu, welches ich von allen hier aufgezählten
Ensembles als das am stärksten ausgeprägteste einordnen würde. Das soziale
Miteinander entsteht somit durch die Musik, aber wird auch darüber hinaus
bewusst gepflegt und gestärkt. Da die Blasmusikvereine eher in ländlicheren
Gegenden verortet werden, spielen sie dort häufig zu besonderen Anlässen im Ort
oder aber auch im Zusammenhang mit Fasnacht zu bestimmten Umzügen. Die Konzerte
sind demnach mit einem festlichen Charakter verbunden, aber trotz allem mit
einem adequaten Grad in der Erwartung an die Leistung der Spieler*innen. Oft
sind Blasmusikvereins-Konzerte auch in einem lockeren Rahmen, wo das Publikum
parallel zum Konzert Essen und Trinken bekommt. Das lockert zum einen die
Athmosphäre und wird dadurch sowohl für das Publikum als auch für die
Musizierenden zu einem gemeinschaftlichen, geselligen Zusammenkommen, welches
von Musik begleitet wird. Wenn man in die Historie zurückblickt, gibt es viele
Hauskonzerte und auch Opernvorstellungen, während denen das Publikum essen,
trinken und laut sein durfte. In diesen Momenten nimmt die Musik eine andere
Rolle ein, als in den Konzertsälen, in der das Publikum die ernste Musik mit
klaren Applausregeln zu honorieren und ansonsten still zu sein hat. Außerdem ist
sie auf diese Art und Weise sehr natürlich in den Alltag mit eingebunden,
während ein Orchesterkonzertbesuch direkt ein spezieller besonderer Anlass ist.
Auch die Literatur ist in den Musikvereinen näher an den Bedürfnissen der
breiten Gesellschaft, da oft Filmmusik oder andere typische, bekannte
Gassenhauer gespielt werden. Durch die Athmosphäre der Musik wird das Publikum
dadurch auch anders und womöglich besser erreicht, als bspw. bei ernster
klassischer Musik. 


\subsection{Chor}
Wie beim Orchester haben wir beim Chor auch einen Chorleiter, welcher die
Aufwärmübungen und die Probenarbeit mit dem Ensemble anleitet. Parameter, wie
sich in den Klang seiner Gruppe einzuordnen sind auch ähnlich zu anderen
Ensembles, sowie das gemeinsame gleichzeitige und möglichst synchrone Agieren.
Technisch kann man sich jedoch nicht ganz soviel abschauen, da das Singen und
wie es funktioniert nicht so offensichtlich ist, wie das spielen eines
Instrumentes, wo jede Haltung imitiert werden kann. Die Körperhaltung, der Stand
an sich kann beim Singen zwar imitiert werden aber darüber hinaus gilt es seine
Stimmbänder zum klingen zu bringen, was eine sehr individuelle und wenig
sichtbare Angelegenheit darstellt. Musikalisch kann man hingegen viel dazulernen
und gerade die technische Einschränkung die Laien oftmals mit ihren Instrumenten
haben, sind beim Singen etwas weniger eine Barriere. Das ist sicher auch ein
Grund dafür, warum es sehr viele Laienchöre gibt, egal ob in einer größeren
Stadt oder auf dem Dorf. Die Voraussetzungen um in einem Chor mitzusingen sind
oft sehr offen und man muss nie aktiven Gesangsunterricht gemacht haben, um in
einem Laienchor mitwirken zu dürfen. Damit ist das Chor-Ensembles das, was den
einfachsten Zugang zur Musik ermöglicht. 


\subsection{Band}
Mit einer Band assoziiert man direkt verschiedene Genres außerhalb der
klassischen Szene und ein soziales Miteinander, welches im Vergleich zu vielen
anderen Ensemblegruppen nicht unbedingt von einer einzigen Person angeleitet
wird. Im Zusammenhang mit dieser Ensemblearbeit verbinde ich auch das kreative
Arbeiten, die Improvisation und das forschen und ausprobieren, wie man seine
eigenen Stücke mit der Gruppe schreiben, komponieren und umsetzten kann. Da eine
Band aber meist auch nur aus fünf bis zehn Personen besteht, ist diese
Interaktion und der Austausch auch möglich, ohne dass es zu komplex wird. Bei
einer Band sind die Spieler*innen meistens mit verschiedenen Instrumenten am
Musizieren, was ein wichtiger Unterschied zu Orchestern, Chören und
Blasmusikvereinen ist, wo wir zumindest immer eine Gruppering von gleichen
Instrumenten haben. Dadurch, dass jede/r für sein/ihr Instrument verantwortlich
ist und sich die Mitspieler*innen nur mit dem eigenen Instrument auskennen, kann
man sich nur über die Sprache der Musik gegenseitig weiterhelfen. Nicht aber
über konkrete technische Anweisungen. Diese Hürden muss man entweder selbst
überwinden, oder eine Lehrkraft um Hilfe bitten, wenn die Musik als Sprache
selbst nicht mehr weiterhilft. Aber es steht auch viel mehr die Musik selbst im
Vordergrund und es ist zweitrangig, wie man zu seinem Ziel gelangt. Durch die
kreative Herangehensweise müssen die Spieler*innen außerdem eine klare
Vorstellung entwickeln, was sie wie haben möchten und diese wiederum mit ihrem
Mitspielern kommunizieren. Oft haben Musizierende diese klare Vorstellung nicht
von Anfang an, aber es zeichnet Bandproben aus, dass man gemeinsam auf die Suche
danach geht. Dieser Ideenaustausch ist sehr wertvoll und besonders zum EU eine
wunderbare Ergänzung.



\subsection{Sonstige Formationen}
Denkt man an die Instrumente, die auf den ersten Blick nicht für Ensembles bzw.
Gruppenunterrichte geeignet sind, wie zum Beispiel das Klavier oder die Gitarre,
so kommen mir trotzdem und zum Glück doch einige Gruppierungen aus meinem Umfeld
in den Sinn, die sich über die gängigen Formationen nicht abschrecken lassen und
neugierig genug sind, sich ein Ensemble zu suchen. Mit dem Klavier gibt es im
klassischen Sektor wenigstens noch die sparte der Klavier-Kammermusik, von
Klaviertrio bis hin zum Klavierquintett, in anderen Genres ist es aus Bands
nicht wegzudenken. Aber darüber hinaus wird es eher wieder schwieriger und auch
bei den jungen Klavierschüler*innen, ist das Ensemblespiel leider ein noch viel
zu sehr vernachlässigter Baustein. Wenn man bedenkt welche ganzen sozialen und
auch musikalischen Kompetenzen im Ensemblspiel geschult und gelernt werden, so
muss ich sagen, dass es in Zukunft unbedingt auch für Klavierschüler
Möglichkeiten und Räume zum gemeinsamen Musizieren geschaffen werden müssen!
Betrachten wir aber die sparte der neuen Musik, so stelle ich fest, dass gerade
hier immer wieder untypische, neue und besondere Besetzungen von Fagott über
Klavier, Schlagzeug und beispielsweise Geige immer gängiger werden und besonders
beliebt sind. Allerdings sind die Anforderungen an das Niveau in der Neuen Musik
meist leider so hoch, dass sie für Instrumentale Anfänger*innen oft nicht
spielbar sind. Bei dem Klavier kommt außerdem noch hinzu, dass man aufgrund der
Größe des Instrumentes leider kaum mehr als zwei Klaviere bzw. Flügel in einem
Raum stehen haben. Wenn man aber an die Familie der Keyboards denkt, sind diese
wiederum sehr vielseitig einsetzbar.

Vergleichen wir das Klavier beispielsweise mit der Gitarre, die es in der
klassischen Musikszene als Ensembleinstrument auch nicht leicht hat, so kann man
wenigstens mit mehreren Schülern ein Gitarrenensemble zusammenstellen und
gemeinsam musizieren. Dass die Gitarristen aber auch Pianisten in den anderen
Genres ihren Platz meist in einer Band haben, vor allem auch im Populären
Musikbereich, ist nicht außer Acht zu lassen. Auch in anderen Kulturen spielt
die Gitarre eine sehr ausgeprägte Rolle. Das Schlagzeug ist ein weiteres
Instrument, welches es im Ensemble nicht immer leicht hat. Immerhin hat jedes
Orchester, jede Band und jeder Musikverein einen Schlagzeuger. Aber mit seinen
eigenen Schlagzeugkolleg*innen zusammen zu spielen, ist außerhalb des
Musikstudiums nicht unbedingt üblich. 

Wir können daraus mitnehmen, dass die Instrumente in den verschiedenen Genres
andere Funktionen und Einsatzmöglichkeiten haben. Deshalb sollten
Instrumentalschüler*innen an verschiedene Genres herangeführt werden um einen
möglichst breiten Horizont zu haben und durch diesen die Möglichkeiten sehen,
die ihr Instrument auch im Zusammenhang von Ensembles bietet. 





