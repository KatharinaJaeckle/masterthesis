\section{Ensembles}

Es gibt einige Punkte, die alle Ensembles gemeinsam haben. So beudeutet das Wort
\emph{Ensemble} zunächst einmal \enquote{eine Gruppe von Dingen, die eng zusammengehören.}
%(wikionary) 
Wenn man die Bedeutung weiter auf die Kunst beziehen möchte, so
steht in der Definition: \autocite{eine Gruppe von Künstlern -wie Schauspieler, oder
Musiker, die gemeinsam etwas vortragen.} %Wiki. ensemble 
Das interessante
an dieser Beschreibung ist, dass das Wort "vortragen" impliziert, dass diese
Gruppe von dem Ensemble auch auf der Bühne oder in einem anderen öffentlichen
Rahmen mit ihrem Ensemble auftreten und das erarbeitete präsentieren. Betrachten
wir den Einzel- Instrumentalunterricht, so ist es nicht unbedingt
selbstverständlich, dass der Schüler das Erarbeitete öffentlich vorträgt. Gerade
bei Privatschülern werden oft keine Vorspiele durchgeführt, da es ohne eine
Institution im Hintergrund, die darauf Wert legt, nicht immer dazu kommt, zumal
auch nicht immer die räumlichen Möglichkeiten dafür geschaffen sind. Wie sieht
es jedoch mit dem Gruppenunterricht aus? Er findet meist doch im Zusammenhang
mit einer Institution statt, sei es dem Kindergarten, der Schule oder einer
Musikschule. Interessanterweise ist es dort, wenn auch teilweise
unausgepsrochen, selbstverständlich, dass die Gruppen bei einem besonderen
Anlass oder auch bei einem dafür geschaffenen Vorspiel, ihre Stücke darbieten.
Andreas Doerne beschreibt in seinem Buch \emph{Umfassend Musizieren} das
Ensemble als \enquote{eine musikalische Ganzehit, die sich durch ein
gleichberechtigtes, dialogisches Miteinander und Zusammenspiel ihrer einzelnen
Teile (Mitglieder) konstituiert.} \autocite[62]{doerne:umfassend_musizieren}
Diese Definition trifft aber meiner Meinung nach nicht in immer in
allen Punkten auf ein Ensemble zu. So werden wir später feststellen, dass
beispielsweise in einem Orchester der Dirigent die Spielimpulse gibt und auch
auch vorgibt, wie zu spielen ist, weshalb es in diesem Moment kein
gleichberechtigtes Miteinander mehr ist. Auch das dialogische Miteinander ist
nicht in jedem Ensemble erwünscht und selbstverständlich. Wenn wir uns
beispielweise die Suzuki-Talenterziehung anschauen, werden wir sehr schnell
feststellen, dass der Lehrer ganz deutlich in der überordneten dominierenden
Position ist, in der das dialogische Miteinander auf einer sehr untergeorneten
Ebene nur selten erwünscht ist. Nichts desto trotz ist das "Ensemblespiel (ist)
Kommunikation mit musikalischen Körpern und Klängen."
\autocite[62]{doerne:umfassend_musizieren} Sich dies zu vergegenwärtigen ist
sehr hilfreich, da man schnell feststellt welche zentrale Rolle dabei die
Kommunikation spielt. Die Kommunikation wiederum ist sehr eng mit der Psyche
eines jeden verknüpft, wenn er Sitautionen für sich selbst auswertet und mit
anderen abgleicht und in vielerlei anderer Hinsicht. Die soziale Komponente ist
meiner Meinung nach das Wesentliche Grundelement eines Ensembles. Welche
sozialen Eigenschaften innerhalb eines Ensembles eine Rolle spielen ist im
folgenden Zitat sehr zutreffend geschildert: \enquote{Das Ensemble ist durchdrungen vom
gegenseitigen Zuhören und Mitteilen, Sehen und Gesehenwerden, Agieren und
Reagieren, Führen und Geführtwerden, Hervorbringen und Abnehmen von
Spielimpulsen, insgesamt also vom wechselseitigen Geben und Nehmen.}
\autocite[62]{doerne:umfassend_musizieren} Diese sozialen Komponenten finden wir
auch in allen folgenden Formationen wieder. 


\subsection{Orchester} 

Ein Symphonieorchester ist, sofern man es überhaupt noch zu einem Ensemble
zählen kann, womöglich die größte Gruppierung von Musizierenden die
zusammenspielen. Aufgrund der Größe, trägt eine klare Hierarchie zur kleineren
Unterteilung innerhalb dieses großen Organismus. Auch aufgrund der Größe ist ein
Dirigent, bei dem vorne der Klang zusammenläuft auch eine wichtige
Schlüsselfigur. ... (%Andreas Doerne?!)

Orchester: Aufmerksamkeit auf: den Dirigenten, auf die Gestaltung des eigenen
Parts, auf das Verfolgen der anderen Stimmen unda auf den Orchesterklang als
Ganzes. (S. 63 Andreas Doerne)

- Hierarchie: Dirigent, Stimmführer, Tuttisten, einzelne Instrumente die
musikalisch viel beeinflussen, wie die Pauke

Orchestermöglichkeiten- und Laufbahnen von Laienmusikern: Schulorchester,
Kinderorchester, Jugendorchester, Studentenorchester, Laienorchester

\subsection{Musikverein}
Bei Musikvereinen gibt es drei Kategorien zum einen die Akkordeonvereine, die
Blasmusikvereine, sowieo die Fasnachtsgruppierungen. In den Akkordeonvereinen spielen zwar alle
erst einmal das selbe Instrument, aber doch in ihren unterschiedlichen
Stimmen und auch leicht unterschiedlich in der Funktionsweise, da die Bässe
beispielweise verstärkt sind. Manchmal gibt es noch Trompeter und Schlagzeuger
dazu, aber das ist eher die Ausnahme. Bei den Fasnachtsgruppen, kommen die
Angehörigen einer Zunft zusammen und musizieren gemeinsam. Bei ihnen ist es
meist nur einmal im Jahr und zu diesem Anlass. 
Im Blasmusikverein sind ähnlich wie im Orchester verschiedene
Instrumentengruppen vertreten aus der sich heraus eine klare Hierarchie
innerhalb der Gruppen bildet. Außerdem gibt es den Dirigenten, welcher die
Proben anleitet. Im Vergleich zum Orchester sind es aber nur Blasinstrumente, die
gespielt werden und ein Schlagzeug zuzüglich. Das Schöne am Musikverein ist die
heterogene Zusammensetzung der Mispieler*innen. Hier kommen Jung und Alt
zusammen, Anfänger und Fortgeschrittene musizieren gemeinsam. Außer den Landespolizeiorchestern oder
beim Militär, gibt es bei
Blasmusikvereinen fast nur Laiengruppierungen. Dadurch, dass also viele Musiker*innen in ihrer Freizeit im Verein
spielen, gehört neben dem eigentlichen Musizieren auch das Vereinsleben dazu,
welches ich von allen hier aufgezählten Ensembles als das am stärkste ausgeprägteste einordnen
würde. Das soziale Miteinander entsteht somit durch die Musik, aber wird auch
darüber hinaus bewusst gepflegt und gestärkt. Da die Blasmusikvereine eher in ländlicheren Gegenden verortet werden,
spielen sie dort häufig zu besonderen Anlässen im Ort oder aber auch im Zusammenhang
mit Fasnacht zu bestimmten Umzügen. Die Konzerte sind demnach mit einem festlichen
Charakter verbunden, aber trotz allem mit einem adequaten Grad in der Erwartung
an die Leistung der Spieler. Oft sind Blasmusikvereins-Konzerte auch in einem
lockeren Rahmen, wo das Publikum parallel zum Konzert Essen und Trinken bekommt.
Das lockert zum einen die Athmosphäre und wird dadurch sowohl für das Publikum
als auch für die Musizierenden zu einem gemeinschaftlichen, geselligen
Zusammenkommen, welches von Musik begleitet wird. Wenn man in die Historie
zurücknlickt, gibt es viele Hauskonzerte und auch Opernvorstellungen, während
denen das Publikum essen und trinken und laut sein durfte. In diesen Momenten
nimmt die Musik eine andere Rolle ein, als in den Konzertsälen, in der das
Publikum die ernste Musik mit klaren Applausregeln zu honorieren und ansonsten
still zu sein hat. 


\subsection{Chor}
Wie beim Orchester haben wir beim Chor auch einen Chorleiter, welcher die
Aufwärmübungen und die Probenarbeit mit dem Ensemble anleitet. Parameter, wie
sich in den Klang seiner Gruppe einzuordnen sind auch ähnlich zum Orchester,
sowie das gemeinsame gleichzeitige und möglichst synchrone Agieren. Technisch
kann man sich jedoch nicht ganz soviel abschauen, da das Singen und wie es
funktioniert nicht so offensichtlich sind, wie das spielen eines Instrumentes,
wo jede Haltung imitiert werden kann. Die Körperhaltung, der Stand an sich kann
beim Singen zwar imitiert werden aber darüber hinaus gilt es seine Stimmbänder zum
klingen zu bringen, was eine sehr individuelle und wenig sichtbare Angelegenheit
darstellt. Musikalisch kann man hingegen viel dazulernen und gerade die
technische Einschränkung die Laien oftmals mit ihren Instrumenten haben, ist
beim Singen etwas weniger eine Barriere.


\subsection{Band}
Mit einer Band assoziiert man direkt verschiedene Genres außerhalb der
klassischen Szene und ein soziales Miteinander, welches im Vergleich zu vielen
anderen Ensemblegruppen nicht unbedingt von einer einzigen Person angeleitet wird. 
Im Zusammenhang mit dieser Ensemblearbeit verbinde ich auch das kreative
Arbeiten, die Improvisation und das forschen und ausprobieren, wie man seine
eigenen Stücke mit der Gruppe schreiben und komponieren kann. Da eine Band aber
meist auch nur aus fünf bis zehn Personen besteht, ist diese Interaktion und der
Austausch auch möglich, ohne dass es zu komplex wird.
Bei einer Band sind die Mitspieler meistens mit verschiedenen Instrumenten am
Musizieren, was ein wichtiger Unterschied zu Orchestern, Chören und
Blasmusikvereinen ist. Dadurch, dass jeder für sein Instrument verantwortlich
ist und die Mitspieler sich mit dem eigenen nicht auskennen, muss man selbst
herausfinden wie man technische Hürden, die auftreten können, überwindet. Es
steht viel mehr die Musik selbst im Vordergrund und es ist zweitrangig, wie man
zu seinem Ziel gelangt. Durch das die kreative Herangehensweise müssen die
Spieler außerdem eine klare Vorstellung entwickeln, was sie wie haben möchten
und diese wiederum mit ihrem Mitspielern kommunizieren. Dieser Ideenaustausch
ist sehr wertvoll und besonders zum EU eine wunderbare Ergänzung. Da der Lehrer doch meist anleitet, wie
man etwas technisch lösen kann um zu dem, vom Lehrer vorgegebenen, Ziel zu
gelangen. 


\subsection{Sonstige Formationen}
Denkt man an die Instrumente, die auf den ersten Blick nicht für
Ensembles bzw. Gruppenunterrichte geeignet sind, wie zum Beispiel das Klavier
oder die Gitarre, so kommen mir trotzdem und zum Glück doch einige Gruppierungen
aus meinem Umfeld in den Sinn, die sich über die gängigen Formationen nicht
abschrecken lassen und neugierig genug sind, sich ein Ensemble zu suchen. Mit
dem Klavier gibt es im klassischen Sektor wenigstens noch die sparte der Klavier-Kammermusik, von
Klaviertrio bis hin zum Klavierquintett, in anderen Genres ist es aus Bands
nicht wegzudenken. Aber darüber hinaus wird es eher wieder
schwieriger und auch bei den jungen Klavierschülern, ist das Ensemblespiel
leider ein noch viel zu sehr vernachlässigter Baustein. Wenn man bedenkt welche
ganzen sozialen und auch musikalischen Kompetenzen im Ensemblspiel geschult und
gelernt werden, so muss ich sagen, dass es in Zukunft unbedingt auch für
Klavierschüler Möglichkeiten und Räume zum gemeinsamen Musizieren geschaffen
werden müssen! Betrachten wir aber die sparte der neuen Musik, so stelle ich
fest, dass gerade hier immer wieder untypische, neue und besondere Besetzungen
von Fagott über Klavier, Schlagzeug und beispielsweise Geige immer gängiger
werden und besonders beliebt sind. Allerdings sind die Anforderungen an das
Niveau in der Neuen Musik meist leider so hoch, dass sie für Instrumentale
Anfänger oft nicht spielbar sind. Bei dem Klavier kommt außerdem noch hinzu, dass
man aufgrund der Größe des Instrumentes leider kaum mehr als zwei Klaviere bzw.
Flügel in einem Raum stehen haben. 

Vergleichen wir das Klavier beispielsweise mit der Gitarre, die es in der
klassischen Musikszene als Ensembleinstrument auch nicht leicht hat, so kann man
wenigstens mit mehreren Schülern ein Gitarrenensemble zusammenstellen und
gemeinsam musizieren. Dass die Gitarristen aber auch Pianisten in den anderen
Genres ihren Platz meist in einer Band haben, vor allem auch im Populären
Musikbereich, ist nicht außer Acht zu lassen. Auch in anderen Kulturen spielt
die Gitarre eine sehr ausgeprägte Rolle.
Das Schlagzeug ist ein weiteres Instrument, welches es im Ensemble nicht immer
leicht hat. Immerhin hat jedes Orchester, jede Band und jeder Musikverein einen
Schlagzeuger. Aber mit seinen eigenen Schlagzeugkollegen zusammen zu spielen,
ist außerhalb des Musikstudiums nicht unbedingt üblich. 





